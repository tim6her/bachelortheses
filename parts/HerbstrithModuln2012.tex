\newpage

\thispagestyle{empty}

\section*{Vorwort}
Den Inhalt dieser Bachelorarbeit bildet eine kurze Einf\"uhrung in die Modultheorie. Im ersten Abschnitt werden die grundlegenden Begriffe der Modultheorie eingef\"uhrt und die ersten elementaren Eigenschaften bewiesen. Wir werden sehen, dass sowohl Ringe mit 1 als auch Vektorr\"aume eine Modulstruktur tragen. Besonders deutlich tritt dieser Zusammenhang zwischen Moduln und Ringen hervor, wenn wir bereits bekannte Ei\-gen\-schaf\-ten von Ringen -- wie den Homomorphiesatz -- auf Moduln verallgemeinern. Der zweite Teil dieser Arbeit widmet sich Er\-zeug\-en\-den\-sys\-tem\-en, Basen und linearer Unabh\"angigkeit und versteht sich als Vergleich zwischen Moduln und Vektorr\"aumen. 
\vspace{1.5cm}
\section*{Abstract}
The content of this bachelor thesis is a brief introduction to module theory. In the first section the basic definitions are introduced and some elementary properties are proven. We will see that unitary rings as well as vector spaces carry a module structure and will generalize well-known results of rings on modules. The second section deals with generating sets, bases and linear independence and can be viewed as a comparison between vector spaces and modules.

\newpage

\chapter{Definitions and Elementary Properties}

	Closely following \cite[Chap.VII]{jantzen2005algebra} we will start with a series of definitions and notations.
	
	\section{Basic Definitions}
	
		\begin{defin} Let $(M,+)$ be an Abelian group and $(R,+,\cdot)$ be a ring with 1. We call the triple $(M,+,\ast)$, where $\ast\colon R\times M\ra M$, $(a,x)\mt a\ast x$ is an operation, that fulfils the following four axioms:
\begin{multienumerate}
\mitemxx{$1\ast x=x$,}{$a\ast(b\ast x)=(a\cdot b)\ast x$,}
\mitemxx{$a\ast(x+y)=a\ast x+a\ast y$ and}{$(a+b)\ast x=a\ast x+b\ast x$}
\end{multienumerate}
for all $a,b \in R$ and all $x,y \in M$, a \emph{left $R$-module} or a \textit{left module} over $R$.
\end{defin}

Analogously, one can define a right $R$-module. More precisely: Given an unital ring $(R,+,\cdot)$ an Abelian group $(M,+)$ endowed by an operation $\ast\colon M \times R \ra G$, $(x,a)\mt x\ast a$ such that for all $x,y\in M$ and all $a,b\in R$ we have
\begin{multienumerate}
\mitemxx{$x\ast 1=x$,}{$(x\ast a)\ast b=x\ast (a\cdot b)$,}
\mitemxx{$(x+y)\ast a=x\ast a+y\ast a$ and}{$x\ast (a+b)=x\ast a+x\ast b$,}
\end{multienumerate}
is called a \emph{right $R$-module.}

\begin{rem}
\begin{exlist}
\item For all rings $R:=(S,+,\cdot)$ we define the \textit{opposite Ring} $R^{op}:=(S,+,\circ)$ of $R$ as the ring with the operation $\circ$, defined by $a\circ b:= b\cdot a$ for all $a,b \in S$. Let $(M,+,\ast)$ be a right $R$-module.
By defining $x\tilde{\ast}a:=a\ast x$ for all $a\in S$ and all $x\in M$, the module $(M,+,\tilde{\ast})$ becomes a left $R^{op}$-module.
Hence, all statements on left modules can be translated into equivalent findings on right modules and vice versa.
In the following we shall therefore exclusively deal with left modules and will simply write module instead of left module.
\item When it is clear which operations are meant, we will note $M$ instead of $(M,+,\ast)$ for such a left module over the ring $R)$.
Furthermore, we will write $ax$ instead of $a\ast x$ and $ab$ instead of $a\cdot b$.
\item In analogy to vector spaces we call the operation $\ast$ \emph{scalar multiplication}.
\end{exlist}
\end{rem}

\begin{exam}\label{ex.module}
\begin{exlist}
\item Every field $F$ is especially a unital ring. Hen\-ce an $F$-vector space is an $F$-module.
Therefore one can view modules as a generalization of vector spaces.
\item Every unital ring $(R,+,\cdot)$ is a module over itself, if we define $a\ast b:= a\cdot b$. \label{ex.RingAsModule}
\item Consider the trivial group $\lbrace 0\rbrace$. This Abelian group is a module over any unital ring. Therefore, this module is called \emph{trivial}. See \cref{lm.basicProp}.(ii) for details.
%NOTE: ^ I can't find a way to reference multienum.
\item Given a vector space $V$ over a field $F$, let $\End_F(V)$ denote the set of all $F$-linear maps of $V$ into itself. Then $\End_F(V)$ is a unital ring with respect to point-wise addition and composition of mappings. We define the operation $\Phi:\End_F(V)\times V \ra V$ by $(\varphi,x) \mt \varphi(x)$, which satisfies
\begin{enumerate}
\item $\Phi(\id, x)=\id(x)=x$,
\item $\Phi(\psi,\Phi(\varphi,x))=\Phi(\psi,\varphi (x))=\\
\psi(\varphi(x))=(\psi\circ\varphi)(x)=\Phi(\psi\circ\varphi,x)$,
\item $\Phi(\varphi,x+y)=\varphi(x+y)=\varphi(x)+\varphi(y)=\Phi (\varphi,x)+\Phi (\varphi,y)$ and
\item $\Phi(\varphi+\psi,x)=(\varphi+\psi)(x)=\varphi(x)+\psi(x)=\Phi(\varphi,x)+\Phi (\varphi,y)$
\end{enumerate}
for all $x,y\in V$ and all $\varphi, \psi \in \End_F(V)$. Hence, $(V,+,\Phi)$ is an $\End_F(V)$-module. This example originated in a comment by \cite{lang2002algebra} on page 118.\label{exam.Vector Space as EndFV-module}
\item \label{ex.Z module} Let $(G,+)$ be an Abelian group. We define an operation $\ast : \Z\times G \ra G$ by
\begin{equation*}
r\ast x:=\left. \begin{cases}
\overbrace{x+x+\ldots+x}^{r \textnormal{ addends}} & \textnormal{if } r>0 \\ 
0 & \textnormal{if } r=0\\ 
-(\underbrace{x+x+\ldots+x}_{\lvert r\rvert \textnormal{ addends}}) & \textnormal{if } r<0
\end{cases} \right.
\end{equation*}
This operation satisfies all module axioms. Hence, every Abelian group can be viewed as a $\Z$-module.
\item \label{ex.genModule} Let $(M,+)$ be an Abelian group. The set $\End(M)$ of all group-endomorphisms of $M$ is a unital ring, with respect to point-wise addition and composition of mappings. For an arbitrary unital ring $R$ and a ring-homomorphism $\varphi\colon R\to\End(M)$, which satisfies $\varphi(1)=\id$, we define the operation $\ast\colon R\times M\rightarrow M$ by $a\ast x:=(\varphi (a))(x)$. This operation has the following properties:
\begin{enumerate}
\item $1\ast x=(\varphi(1))(x)=\id(x)=x$,
\item $a\ast(b\ast x)=a\ast(\varphi(b))(x)=\varphi(a)((\varphi(b))(x))=$\\
$(\varphi(a)\circ \varphi(b))(x)=(\varphi(ab))(x)=(ab)\ast x$,
\item $a\ast(x+y)=(\varphi(a))(x+y)=(\varphi(a))(x)+(\varphi(a))(y)=$\\
$a\ast x+a\ast y$ and
\item $(a+b)\ast x= (\varphi(a+b))(x)=(\varphi(a))(x)+(\varphi(b))(x)=$\\
$a\ast x+b\ast x$
\end{enumerate}
for all $a,b\in R$ and all $x,y\in M$. Hence $(M,+,\ast)$ is a module.
\end{exlist}
\end{exam}

In \cref{ex.genModule} we have seen that every ring-ho\-mo\-morph\-ism $\varphi\colon R\rightarrow\End(M)$ induces exactly one scalar multiplication, such that $a\ast x=(\varphi(a))(x)$. On the other hand, one verifies easily that every scalar multiplication induces one and only one ring-ho\-mo\-morph\-ism $\varphi\colon R\rightarrow\End(M)$, such that $(\varphi(a))(x)=a\ast x$. Let now $M$ be an $R$-module. For an arbitrary $a\in R$ consider the mapping $\ell_a\colon M\rightarrow M$ defined by $x\mapsto a\ast x$. As a result of module axiom (iii) we see $\ell_a$ is a group-homomorphism. We define a mapping $\varphi\colon R\rightarrow \End(M)$ by $a \mapsto \ell_a$. At first, we note that module axiom (i) implies $\varphi(1)=\id$. Let $a,b$ be two arbitrary elements in $R$, then
\begin{align*}
(\varphi(a+b))(x)=\ell_{a+b}(x)=(a+b)x\overset{\text{Ax. (iv)}}{=}\\
=ax+bx=\ell_a(x)+\ell_b(x)=(\varphi(a)+\varphi(b))(x)\\
\intertext{and}
(\varphi(ab))(x)=\ell_{ab}(x)=(ab)x\overset{\text{Ax. (ii)}}{=}\\
=a(bx)=\ell_a(\ell_b(x))=(\varphi(a)\circ \varphi(b))(x).
\end{align*}
Hence, $\varphi$ is a ring-ho\-mo\-morph\-ism and we have found the desired form. Due to the construction of $\varphi$ this ring-homomorphism is  unique. 

\begin{lem}\label{lm.basicProp}For any $R$-module $M$, the following identities hold for all elements $a,\seq{a}\in R$ and all $x,\seq[m]{x}\in M$:
\begin{multienumerate}
\mitemxx{$0x=0$,}{$a0=0$,}
\mitemxx{$(-1)x=-x$,}{$(-a)x=a(-x)=-(ax)$,}
\mitemxx{$(a_1-a_2)x=a_1x-a_2x$,}{$a(x_1-x_2)=ax_1-ax_2$ and}
\mitemx{$\sum\limits^n_{i=1}a_i\sum\limits^m_{j=1}x_i=\sum\limits^n_{i=1}\sum\limits^m_{j=1}a_ix_j$.}
\end{multienumerate} 
\end{lem}
\begin{proof}
For all $x\in M$ we have
\begin{equation*}
0=0x-0x=(0+0)x-0x=0x+0x-0x=0x,
\end{equation*}
which implies (i). The second identity can be shown analogously. For an arbitrary $x\in M$ we find 
\begin{equation*}
0\overset{\text{(i)}}{=}0x=(1+(-1))x=x+(-1)x.
\end{equation*}
The uniqueness of the inverse implies (iii). Claim (iv) follows easily from (iii). To prove (v), one notes
\begin{equation*} 
(a_1-a_2)x=(a_1+(-a_2))x=a_1x+(-a_2)x\overset{\text{(iv)}}{=}a_1x-a_2x.
\end{equation*}
One proves (vi) analogously. The generalization of the last two module axioms which can be found in (vii) is shown via induction on $n$ and $m$.
\end{proof}

\begin{defin}Let $M$ be a module over a unital Ring $R$ and $N$ be a non-empty subset of $M$. We call $N$ a \emph{submodule} of $M$ and note $N\leq M$, if $N$ is a module itself, with respect to the restrictions of the operations of $M$ to $N$.
\end{defin}  

\begin{lem}\label{lm.SubMod}
Let $M$ be an $R$-module and $N\subseteq M$ be a non-empty subset, then the following statements are equivalent:
\begin{thmlist}
\item $N$ is a submodule of $M$.
\item $x+y\in N$ and $ax\in N$ for all $a\in R$ and for all $x,y\in N$
\end{thmlist}
\end{lem}
\begin{proof}
Implication (i)$\Rightarrow$(ii) is clear. To prove the other direction, we consider two elements $x,y\in N$. We know $-1\in R$ which implies $-y=(-1)y\in N$ and therefore $x-y\in N$. Hence, $(N,+)$ is a group. Since the addition is commutative on the whole set $M$, the group $(N,+)$ is Abelian. The same argument can be applied to verify the four module axioms. Therefore $N$ is a submodule of $M$. 
\end{proof}

\begin{exam}\begin{exlist}
\item Every module $M$ contains the submodules $M$ and $\lbrace 0\rbrace$. Therefore, these submodules are called \emph{trivial}.
\item Given a vector space $V$, the linear subspaces of $V$ are its submodules.
\item If we view a unital ring $R$ as a left module over itself, being a submodule of $R$ is equivalent to being a left ideal of $R$.
\item An $F$-vector space $V$ viewed as a module over the ring of its endomorphisms $\End_F(V)$ contains only the trivial submodules. See \cref{exam. Generating set of V over EndFV} for further details.
\item In the case of an Abelian group $G$ viewed as a $\Z$-module, being a submodule is equivalent to being a subgroup.
\end{exlist}\end{exam}

\begin{lem} \label{lem.submodules}
Let $M$ be an $R$-module, then the following statements hold:
\begin{thmlist}
\item Given a non-empty index set $I$ and $N_i\leq M\;\forall i\in I$, then the intersection $\bigcap\limits_{i\in I}N_i$ is a submodule of $M$. \label{lm.IntersectionOfSubmods}
\item Given a positive integer $n\in\Z$ and $N_i\leq M\;\forall i\in \lbrace 0,1,\ldots,n\rbrace$, then the  set of sums 
\begin{equation*}
\sum\limits_{i=0}^n N_i:=\left\lbrace x\in M\middle\arrowvert\forall i\in\lbrace 0,1,\ldots ,n\rbrace ,\exists x_i\in N_i\colon x=\sum_{i=0}^nx_i\right\rbrace
\end{equation*}
is a submodule of $M$.
\item Let $N_1$ and $N_2$ be submodules of $M$, then $N_1\cup N_2\leq M$ is equivalent to $N_1\subseteq N_2$ or $N_2\subseteq N_1$.
\end{thmlist}
\end{lem}
\begin{proof}
\begin{plist}
\item Let $x,y$ be in $\bigcap_{i\in I}N_i$, then $x,y\in N_i$ for all $i\in I$. It follows $x+y\in N_i$ for all $i\in I$, which implies $x+y\in \bigcap_{i\in I}N_i$. Furthermore, we have $a x \in N_i$ for all $i\in I$ and therefore $a x \in\bigcap_{i\in I}N_i$ for an arbitrary $a$ in $R$. \cref{lm.SubMod} implies now, that $\bigcap_{i\in I}N_i$ is a submodule of $M$.
\item Suppose $x,y\in \sum_{i=0}^n N_i$ then there exist $x_i,y_i\in N_i$ for $0\leq i\leq n$ such that $x=\sum_{i=0}^{n}x_i$ and $y=\sum_{i=0}^{n}y_i$. We have $x+y=\sum_{i=0}^{n}x_i+y_i$ and $x_i+y_i\in N_i$ for all $0\leq i\leq n$. Hence, $x+y\in \sum_{i=0}^n N_i$. Given an $a\in R$ we notice $ax=a\sum_{i=0}^{n}x_i=\sum_{i=0}^{n}ax_i$ and $ax_i\in N_i$ for $0\leq i\leq n$.
\item ($\Rightarrow$) Suppose $N_1\subsetneq N_2$ and $N_2\subsetneq N_1$, then there exists an $x\in N_1\setminus N_2$ and there exists a $y\in N_2\setminus N_1$. This implies, that $x,y\in N_1\cup N_2$ and if $N_1\cup N_2$ were a submodule then $x+y$ is an element of $N_1\cup N_2$. Assuming $x+y\in N_1$ it follows that $y=-x+(x+y)$ is an element of $N_1$, which contradicts our premises. Analogously the assumption $x+y\in N_2$, implies $x=(x+y)-x\in N_2$ and again we are lead to a contradiction. The other direction of the equivalence is trivial.
\end{plist}
\end{proof}

	\section{Module-Homomorphisms}
		
		% !TeX encoding = UTF-8
% !TeX spellcheck = en_GB
 \begin{defin}
Let $M,M'$ be two modules over a unital ring $R$ and let $\varphi : M \ra M'$ be a mapping. 
\begin{thmlist}
\item We call $\varphi$ a \emph{module-homomorphism} or \textit{$R$-linear} if it satisfies
\begin{exlist}
\item $\varphi(x+y)=\varphi(x)+\varphi(y)$ and
\item $\varphi(ax)=a\varphi(x)$
\end{exlist}
for all $x,y \in M$ and all $a \in R$.
\item A (module-)homomorphism is called \textit{monomorphism}, if it is injective. An \emph{epimorphism} is a surjective homomorphism. A bijective homomorphism is referred to as \textit{isomorphism}. If $M=M'$ we call $\varphi$ an \textit{endomorphism}. A bijective endomorphism is called \textit{automorphism}.
\item For a homomorphism $\varphi$ we define the subset $\ker(\varphi)\subseteq M$ by $\ker(\varphi):=\lbrace x\in M \mid \varphi(x)=0\rbrace$ and call it the \emph{kernel of $\varphi$}. The subset $\im(\varphi)\subseteq M'$ is defined by $\im(\varphi):=\lbrace y\in M' \mid \exists x\in M\colon \varphi(x)=y\rbrace$. This set is called \emph{image of $\varphi$}.
\end{thmlist}
\end{defin}

Note that the first postulate in the definition of a module\-/homomorphism implies that $\varphi$ is a group-homomorphism. We can therefore apply statements of group-homomorphisms to mo\-dule-ho\-mo\-morph\-isms. For example, every mo\-dule-ho\-mo\-morph\-ism $\varphi\colon M\rightarrow M'$ satisfies $\varphi(0)=0$ and $\varphi(-x)=-\varphi(x)$.

\begin{exam}
\begin{exlist}
\item Given a modul $M$, the map $\id_M\colon M \rightarrow M$ defined by $x\mapsto x$ is an automorphism. We  call this mapping the \emph{identity} (mapping) on $M$.
\item Let $M$ and $M'$ be $R$-modules. The mapping $\varphi\colon M\rightarrow M'$ defined by $x\mapsto 0$ is a homomorphism with $\ker(\varphi)=M$. 
\item Let $G$ and $H$ be two $\Z$-modules like the one defined in Example~\ref{ex.Z module} then every group\-/homomorphism $\varphi\colon G\rightarrow H$ is a module\-/homomorphism. %NOTE: Sloppy ref
To prove this one verifies $\varphi(rx)=r\varphi(x)$ for all $r\in\Z$, $z\geq 0$ per induction on $r$. If $r<0$, we note that 
\begin{equation*}
\varphi(rx)=\varphi(-1(\lvert r\rvert x))=-\varphi(\lvert r\rvert x)=-\lvert r\rvert\varphi(x)=r\varphi(x).
\end{equation*}
\end{exlist}
\end{exam}

\begin{lem}\label{lm.Composition of Homo} 
Let $M, N$ and $O$ be $R$-modules and
     \[\varphi\colon M\rightarrow N, \quad \psi\colon N\rightarrow O\] 
homomorphisms then $\psi\circ\varphi\colon M\rightarrow O$ is a homomorphism.
\end{lem}
\begin{proof}
When $\varphi$ and $\psi$ are viewd as homomorphisms of groups, the statement above holds. Hence, it is enough to prove the compatibility with scalar multiplication. Let therefore $a$ be in $R$ and $x$ be in $N$ then $\psi(\varphi(ax))=\psi(a\varphi(x))=a\psi(\varphi(x))$.
\end{proof}

One verifies easily that \cref{lm.Composition of Homo} holds for monomorphisms as well, \ie, let $M, N, O$ be $R$-modules and $\varphi\colon M\rightarrow N$, $\psi\colon N\rightarrow O$ monomorphisms then $\psi\circ\varphi\colon N\rightarrow O$ is a monomorphism. Moreover, the statement remains true for epi- and isomorphisms.

\begin{lem}
Let $M,M'$ be $R$-modules and $\varphi\colon M\rightarrow M'$ a homomorphism then the following statements hold:
\begin{thmlist}
\item Let $N$ be a submodule of $M$ and $N'$ a submodule of $M'$ then $\varphi(N)\leq M'$ and $\varphi^{-1}(N') \leq M$.
\item $\ker(\varphi)\leq M$ and $\im(\varphi)\leq M'$.
\item The mapping $\varphi$ is a monomorphism if and only if $\ker(\varphi)={0}$.
\end{thmlist}
\end{lem}
\begin{proof}
\begin{plist}
\item We notice that $\varphi(N)$ is an abelian subgroup of $M'$ and $\varphi^{-1}(N')$ is an abelian subgroup of $M$. Hence, it is enough to verify that these groups are closed under scalar multiplication. For this purpose, let $a$ be an arbitrary element in $R$ and $y$ be in $\varphi(N)$ then there exists an $x\in N$ such that $\varphi(x)=y$. We note $a\varphi(x)=\varphi(ax)\in\varphi(N)$, because $ax\in N$. If $x$ is an element of $\varphi^{-1}(N')$ then there exists a $y\in N'$ such that $\varphi(x)=y$ and we have $\varphi(ax)=a\varphi(x)=ay\in N'$, from which follows $ax\in \varphi^{-1}(N')$.
\item Considering $N=M$ and $N'=\lbrace 0\rbrace$ statement (ii) follows from claim~(i).
\item Since $\varphi$ viewed as a group-homomorphism is injective if and only if $\ker(\varphi)={0}$, this statement is clear.
\end{plist}
\end{proof}

\begin{lem}
Let $M,M'$ be $R$-modules and $\varphi\colon M\rightarrow M'$ a homomorphism then the following statements hold:
\begin{thmlist}
\item If $\varphi$ is an isomorphism then $\varphi^{-1}$ is an isomorphism.
\item The mapping $\varphi$ is an isomorphism if and only if there exits a homomorphism $\psi\colon M'\rightarrow M$ such that $\psi\circ\varphi=\id_M$ and $\varphi\circ\psi=\id_{M'}$.
\end{thmlist}
\end{lem}
\begin{proof}
\begin{plist}
\item Owing to the fact that $\varphi^{-1}$ is a group-isomorphism, it is enough to verify the compatibility with scalar multiplication. Let therefore $a$ be an arbitrary element in $R$ and $y$ in $M'$ then there exists an $x\in M$ such that $\varphi(x)=y$. Hence, we have
\begin{equation*}
\varphi^{-1}(ay)=\varphi^{-1}(a\varphi(x))=\varphi^{-1}(\varphi(ax))=ax=a\varphi^{-1}(y).
\end{equation*}
\item Since $\varphi$ is bijective viewed as a group-homomorphism iff this condition is met, statement (ii) is clear.
\end{plist}
\end{proof}

We will continue with a brief study of the algebraic structure of module\-/homomorphisms.

\begin{pro}\label{prop.Homomorphisms are Module}
Let $M,M'$ be modules over the unital ring $R$ and let $\Hom_R(M,M')$ denote the set of all homomorphisms $\varphi\colon M\to M'$ then
\begin{thmlist}
\item \label{prop.Homomorphisms are Group} $\Hom_R(M,M')$ is an abelian group, with respect to the operation \[+\colon\Hom_R(M,M')\times\Hom_R(M,M')\to\Hom_R(M,M')\]
defined by $(\varphi+\psi)(x)=\varphi(x)+\psi(x)$.
\item If $R$ is commutative then $\Hom_R(M,M')$ is an $R$-module, with respect to the operations $+$ and 
\[\ast\colon R\times\Hom_R(M,M')\to\Hom_R(M,M')\]
defined by $(a\ast\varphi)(x)=a\varphi(x)$.
\end{thmlist}
\end{pro}
\begin{proof}
\begin{plist}
\item \sloppypar Let $\Hom(M,M')$ denote the set of all group-ho\-mo\-morph\-isms of $M$ to $M'$ then $(\Hom(M,M'),+)$ is an abelian group. Hence, it is enough to verify that $\Hom_R(M,M')$ is a subgroup of the homomorphism group $\Hom(M,M')$.

Given two homomorphisms $\varphi,\psi\in\Hom_R(M,M')$ and an element $a\in R$, we find
\begin{align*}
(\varphi-\psi)(ax)&=\varphi(ax)-\psi(ax)=a\varphi(x)-a\psi(x)=\\
            &=a(\varphi(x)-\psi(x))=a(\varphi-\psi)(x)
\end{align*}
and the claim follows.
\item The mapping $a\ast\varphi$ is a homomorphism for all $a\in R$ since the following identities hold for arbitrary $x,y\in M$ and an arbitrary $b\in R$:
\begin{enumerate}
\item $(a\ast\varphi)(x+y)=a\varphi(x+y)=a\varphi(x)+a\varphi(y)=(a\ast\varphi)(x)+(a\ast\varphi)(y)$ and
\item $(a\ast\varphi)(bx)=a\varphi(bx)=ab\varphi(x)=ba\varphi(x)=b(a\ast\varphi)(x)$.
\end{enumerate} 


Secondly, we need to prove the four module-axioms, for this purpose we consider two homomorphisms $\varphi,\psi\in\Hom_R(M,M')$ and two arbitrary elements $a,b\in R$. We find
\begin{enumerate}
\item $(1\ast\varphi)(x)=1\varphi(x)=\varphi(x)$,
\item $(a\ast(b\ast\varphi))(x)=a(b\ast\varphi)(x)=a(b\varphi(x))=(ab)\varphi(x)=((ab)\ast\varphi(x))$,
\item $(a\ast(\varphi+\psi))(x)=a(\varphi+\psi)(x)=a(\varphi(x)+\psi(x))=a\varphi(x)+a\psi(x)=(a\ast\varphi + a\ast\psi)(x)$ and
\item $((a+b)\ast\varphi)(x)=(a+b)\varphi(x)=a\varphi(x)+b\varphi(x)=(a\ast\varphi+b\ast\varphi)(x)$.
\end{enumerate}
\end{plist}
\end{proof}

\begin{rem}
Let $M$ a module over a commutative unital ring $R$. If we view $R$ as a module over itself like we did in \cref{ex.RingAsModule}, \cref{prop.Homomorphisms are Module} implies that $\Hom_R(R,M)$ is an $R$-module. The mapping 
\[ F\colon\Hom_R(R,M)\to M\;\text{ defined by }\;\varphi\mapsto\varphi(1) \]
is a homomorphism, because 
\begin{equation*}
F(\varphi+\psi)=(\varphi+\psi)(1)=\varphi(1)+\psi(1)=F(\varphi)+F(\psi)
\end{equation*}
and $F(a\varphi)=a\varphi(1)=aF(\varphi)$ holds for all $\varphi,\psi\in\Hom_R(R,M)$ and for all $a\in R$. Given an $x\in M$, consider the mapping $\varphi_x\colon R\to M$ defined by $a\mapsto ax$. This mapping is a homomorphism, because of module axiom (ii) and (iv). We note, that $F(\varphi_x)=x$ and see that $F$ is surjective. The mapping $F$ is injective, because for all $\varphi\in\ker(F)$, we have $\varphi(a)=\varphi(a1)=a\varphi(1)=0$ and therefore $\varphi=0$. It follows that $\ker(F)=\lbrace 0\rbrace$, which implies the claim. The argumentation above shows
\begin{equation*}
\Hom_R(R,M)\cong M.
\end{equation*}
\end{rem}

	\section{Quotient Modules and Isomorphism Theorems}

		% !TeX encoding = UTF-8
% !TeX spellcheck = en_GB
Given an $R$-module $M$ which contains a submodule $N$, then by commutativity $(N,+)$ is a normal subgroup of $(M,+)$. Hence, the relation $\equiv$ defined by $x\equiv y:\Leftrightarrow x-y\in N$ is an equivalence relation on $M$, which is compatible with addition, therefore the set of all equivalence classes $\MtN$ is an abelian group called quotient group $M$ modulo $N$. The following theorem shows, that the quotient group $\MtN$ carries a module structure over $R$.

\begin{thm} \label{theo.quotient module}
Let $M$ be an $R$-module which contains the submodule $N$ and define the operation $\ast\colon R\times \MtN\rightarrow \MtN$ by $a(x+N)=ax+N$. This operation is well-defined and $(\MtN,+,\ast)$ is a module over $R$, where + is the addition on the quotient group $\MtN$.
\end{thm}

\begin{proof}
At first we verify that the mapping $\ast$ is well-defined. Let therefore $x\equiv x'$. We need to prove that $ax\equiv ax'$. Since $x\equiv x'$ there exists an $n\in N$ such that $x=x'+n$ and therefore $ax-ax'=ax-a(x-n)=an\in N$ since $N$ is a submodule of $M$. Hence, $ax\equiv ax'$ and $ax+N=ax'+N$. 

It remains to prove that the four module axioms are satisfied. For all $a,b\in R$ and all $x+N,y+N\in \MtN$ we have
\begin{enumerate}
\item $1(x+N)=x+N$,
\item $a(b(x+N))=a(bx+N)=abx+N=(ab)(x+N)$,
\item $a((x+N)+(y+N))=a(x+y+N)=a(x+y)+N=ax+ay+N=(ax+N)+(ay+N)=a(x+N)+a(y+N)$ and
\item $(a+b)(x+N)=(a+b)x+N=ax+bx+N=(ax+N)+(bx+N)=a(x+N)+b(x+N)$.
\end{enumerate}
Hence, $(\MtN,+,\ast)$ is a module.
\end{proof}

\begin{defin}
We call the module $(\MtN,+,\ast)$ introduced in \cref{theo.quotient module} the \emph{quotient module $M$ modulo $N$}.
\end{defin}

\begin{rem}
\begin{exlist}
\item In \cref{theo.quotient module} we have seen that every $R$-sub\-mod\-ule $N$ of $M$ induces an equivalence relation $\equiv$ on $M$  such that $\equiv$ is compatible with addition and scalar multiplication, \ie $w\equiv x$ and $y\equiv z$ implies $w+y\equiv x+z$ and $aw\equiv ax$ for all $a\in R$. 

Let on the other hand $\equiv$ be an equivalence relation on $M$ that is compatible with addition and scalar multiplication, then there exists a submodule $N\leq M$ such that $x\equiv y\Leftrightarrow x-y\in N$. To prove this we consider two elements $x,y\in M$ with $x\equiv y$. This is equivalent to $x-y\equiv 0$ since $-y\equiv -y$ and $\equiv$ is compatible with addition. Let $N_0$ denote the equivalence class of 0, \ie $N_0:=\lbrace z\in M|z\equiv 0\rbrace$. This set is not empty because $0\equiv 0$. Furthermore, $N_0$ is a submodule of $M$ because for all $z ,w\in N_0$ and all $a\in R$ the identities $z +w\equiv 0+0$ and $az\equiv a0$ hold. We conclude
\begin{equation*}
x\equiv y\Leftrightarrow x-y\in N_0\leq M \;\text{and}\; \faktor{M}{\equiv}=\faktor{M}{N_0}.
\end{equation*}
\item Consider the mapping $\pi\colon M\rightarrow \MtN$ defined by $x\mapsto x+N$. We notice that $\pi$ is a group-epimorphism. Moreover, it is a module-epimorphism. Given an arbitrary $a\in R$ and $x\in M$ we have $\pi(ax)=ax+N=a(x+N)=a\pi(x)$. We call the mapping $\pi$ \emph{canonical projection of $M$ to $\MtN$}. The kernel of $\pi$ is $N$.
\end{exlist}
\end{rem}

\begin{thm}[Universal Property]\label{theo.Universal Property}
Let $\varphi\colon M \to M'$ be a homomorphism of $R$-modules and let $N$ be a submodule of $M$. The canonical projection of $M$ to $\MtN$ is denoted by $\pi$. If $N$ is a submodule of $\ker(\varphi)$ then there exists a unique homomorphism $\overline{\varphi}\colon \MtN\rightarrow M'$ such that $\overline{\varphi}\circ\pi=\varphi$. Moreover, $\ker(\overline{\varphi})=\ker(\varphi)/N$ and the following equivalence holds: $\overline{\varphi}$ is an isomorphism iff $\varphi$ is an epimorphism and $\ker(\varphi)=N$.
\end{thm}

The statement of \cref{theo.Universal Property} can be visualized by the following commutative diagram.

\begin{center}
\begin{tikzpicture}
  \node (M) at ( 0,1.5)  {$M$};
  \node (M2) at (3,1.5) {$M'$}
    	edge [<-] node[above,comment] {$\varphi$} (M);
  \node (MtN) at (0,0) {$\MTN$}
  	edge [->, dashed] node[below,comment] {$\exists !\overline{\varphi}$} (M2)
  	edge [<-] node[left,comment] {$\pi$} (M);
\end{tikzpicture}
\end{center}


\begin{proof}
We notice that there exists a unique group-homomorphism $\overline{\varphi}$ such that $\overline{\varphi}\circ\pi=\varphi$. The mapping $\overline{\varphi}$ is defined by $\overline{\varphi}(x+N)=\varphi(x)$. It remains to verify that $\overline{\varphi}$ is a module-homomorphism. Let therefore be $a\in R$ and $x+N\in \MtN$, then $\overline{\varphi}(a(x+N))=\overline{\varphi}(ax+N)=\varphi(ax)=a\varphi(x)=a\overline{\varphi}(x+N)$. The other statements are clear, since they hold for every group-homomorphism.
\end{proof}

\begin{cor}[Homomorphism Theorem] \label{cor.Homomorphism Theorem}
Let $\varphi\colon M\to M'$ be $R$-linear, then we have
\begin{equation*}
\faktor{M}{\ker(\varphi)}\cong\im(\varphi)
\end{equation*}
\end{cor}
\begin{proof}
Let $N=\ker(\varphi)$ then the corollary follows from \cref{theo.Universal Property}.
\end{proof}

\begin{thm}[Comparison Principle] \label{theo.Comparison Principle}
Let $M$ be an $R$-module which contains the submodule $N$ and $\pi\colon M\to\MtN$ the canonical projection. The set of all submodules of $M$ which contain $N$ shall be denoted by $S(M,N):=\lbrace M'\leq M \mid M'\supseteq N\rbrace$ and the set of all submodules of $\MtN$ by $S(\MtN)$. Furthermore, $\mathcal{P}(\MtN)$ denotes the power set of $\MtN$. We define a mapping $\varphi\colon S(M,N)\to \mathcal{P}(\MtN)$ by
\begin{equation*}
\varphi(M'):=\pi(M')=\left\lbrace x+N\in \MTN \middle\arrowvert x\in M'\right\rbrace.
\end{equation*}
Then $\varphi$ is injective, $\im(\varphi)=S(\MtN)$ and $\varphi(M')\cong\nicefrac{M'}{N}$ for all $M'\in S(M,N)$.
\end{thm}
\begin{proof}
At first we verify that $x+N\in \varphi(M')$ is equivalent to $x\in M'$. Assume $x+N$ is an element of $\varphi(M')$ then there exists an $x'\in M'$ such that $x+N=x'+N$. This implies the existence of an $n\in N$ such that $x'=x+n$. It follows $x+n\in M'$ and $x\in M'$ because $M'\supseteq N$. The other direction is trivial.

To prove that $\varphi$ is injective we consider two submodules $M_1$ and $M_2\in S(M,N)$ such that $\varphi(M_1)=\varphi(M_2)$. Then the following equivalence holds:
\begin{equation*}
x\in M_1\Leftrightarrow x+N\in \varphi(M_1)\Leftrightarrow x+N\in\varphi(M_2)\Leftrightarrow x\in M_2.
\end{equation*}

Next we verify $\im(\varphi)=S(\MtN)$. Since $\pi$ is a homomorphism $\varphi(M')=\pi(M')$ is a submodule of $\MtN$ for all submodules $M'$ of $M$. On the other hand, if $S$ is a submodule of $\MtN$, \ie $S\in S(\MtN)$, then $\pi^{-1}(S)\leq M$. Seeing that $N\in S$ we find $N\subseteq\pi^{-1}(S)$. Therefore there exists an $M'\in S(M,N)$ such that $\varphi(M')$ equals $S$, which was to be shown.

In order to prove $\varphi(M')\cong\nicefrac{M'}{N}$ consider the natural embedding $\iota\colon \nicefrac{M'}{N}\to\MtN$ defined by $x+N\mapsto x+N$. It is a monomorphism with $\im(\iota)=\lbrace x+N\in\MtN \mid x\in M'\rbrace=\varphi(M')$.
\end{proof}

\cref{theo.Comparison Principle} implies directly the following corollary.

\begin{cor}
Let $M$ be an $R$-module which contains the submodule $N$, then the following statements hold:
\begin{thmlist}
\item Given a submodule $M'$ such that $N\leq M'\leq M$, then $\nicefrac{M'}{N}$ is a submodule of $\MtN$.
\item Given a quotient module $S$ such that $S\leq\MtN$, then there exists a submodule $M'$ with $N\leq M'\leq M$ such that $\nicefrac{M'}{N}=S$.
\end{thmlist}
\end{cor}

\begin{cor}[Isomorphism Theorems]
Given an $R$-module $M$ which contains the submodules $N_1,N_2$, then we have
\begin{thmlist}
\item $$\faktor{N_1}{\left(N_1\cap N_2\right)}\cong\faktor{\left(N_1+N_2\right)}{N_2}$$
\item and if $N_2\subseteq N1$ $$\faktor{\left(\faktor{M}{N_2}\right)}{\left(\faktor{N_1}{N_2}\right)}\cong\faktor{M}{N_1}.$$
\end{thmlist}
\end{cor}
\begin{proof}
\begin{plist}
\item Consider the mapping $\alpha\colon N_1\to \nicefrac{\left(N_1+N_2\right)}{N_2}$ def\-ined by $\alpha(x)=x+N_2$. It is the composition of the embedding $N_1\to N_1+N_2$, $x\mapsto x$ and the canonical projection $N_1+N_2\to\nicefrac{\left(N_1+N_2\right)}{N_2}$, $x\mapsto x+N_2$, hence $\alpha$ is a homomorphism. The kernel of $\alpha$ is $N_1\cap N_2$ and $\alpha$ is surjective, because for an arbitrary $x+N_2\in\nicefrac{\left(N_1+N_2\right)}{N_2}$ there exist $n\in N_1$ and $m\in N_2$ such that $x+N_2=n+m+N_2=n+N_2=\alpha(n)$. Now \cref{cor.Homomorphism Theorem} implies the claim. 
\item Consider the canonical projection $\pi\colon M\to \nicefrac{M}{N_1}$, $x\mapsto x+N_1$. Because $N_2\leq N_2=\ker(\pi)$ and $\nicefrac{M}{N_1}=\im(\pi)$, \cref{theo.Universal Property} implies the existence of an epimorphism $\overline{\pi}\colon \nicefrac{M}{N_2}\to\nicefrac{M}{N_1}$ with $\ker(\overline{\pi})=\nicefrac{\ker(\pi)}{N_2}=\nicefrac{N_1}{N_2}$. Again \cref{cor.Homomorphism Theorem} implies the claim.
\end{plist}
\end{proof}
		
	\section{Direct Sums and Products}
	
		Given $(M_i)_{i\in I}$ a family of $R$-modules, indexed by $I$. The set of all indexed families $(x_i)_{i\in I}$ where $x_i\in M_i$ for all $i\in I$ is called the \emph{Cartesian product of the modules $(M_i)_{i\in I}$} and is denoted by $\prod\limits_{i\in I} M_i$. If $M_i=M$ for all $i\in I$ we rather note $M^I$ for their Cartesian product.

\begin{pro}\label{prop.DirectProduct}
The Cartesian product $\prod_{i\in I} M_i$ of the family of $R$-modules $(M_i)_{i\in I}$ is an $R$-module with respect to component-wise addition and component-wise scalar-multiplication, \ie 
\[(x_i)_{i\in I}+(y_i)_{i\in I}:=(x_i+y_i)_{i\in I}\text{ and }a\ast(x_i)_{i\in I}:=(a\ast x_i)_{i\in I}.\]
\end{pro}

To prove the proposition above one has to to verify all module axioms in a straight forward way.

\begin{defin}
We call the $R$-module $\left(\prod_{i\in I} M_i,+,\ast\right)$ introduced in Proposition~\ref{prop.DirectProduct} the \emph{direct product} of the family of modules $(M_i)_{i\in I}$.
\end{defin}

For every $j\in I$ we call the mapping $\pi_j\colon \prod_{i\in I} M_i\to M_j$ defined by $\pi_j((x_i)_{i\in I})=x_j$ \emph{projection} on the factor $M_j$. One verifies easily that $\pi_j$ is a homomorphism for every $j$ in $I$. These projections fulfil the following universal property:

\begin{pro}[Universal Property of the Direct Product]
Let $(\varphi_i)_{i\in I}$ be an indexed family of module-homomorphisms $\varphi_i\colon M\to M_i$. Then there exists a unique homomorphism $\varphi\colon M\to \prod_{i\in I} M_i$ such that $\varphi_i=\pi_i\circ\varphi$ for all $i\in I$.
\end{pro}

One can visualize the proposition above with the following commutative diagram.

\begin{center}
\begin{tikzpicture}
  \node (N) at ( 0,1.5)  {$M$};
  \node (Mi) at (3,1.5) {$Mi$}
    	edge [<-] node[above,comment] {$\varphi_i$} (N);
  \node (MMi) at (3,0) {$\prod\limits_{i\in I}M_i$}
  	edge [<-, dashed] node[below,comment] {$\exists !\varphi$} (N)
  	edge [->] node[right,comment] {$\pi_i$} (Mi);
\end{tikzpicture}
\end{center}

\begin{proof}
Consider the mapping 
\[\varphi\colon M\to \prod_{i\in I} M_i, \; x \mapsto (\varphi_i(x))_{i\in I}.\]
It is a homomorphism and fulfils $\varphi_i=\pi_i\circ\varphi$ for all $i\in I$. The uniqueness follows from the construction of $\varphi$.
\end{proof}

\begin{thm}\label{theo.DirectSum}
Let $\prod_{i\in I} M_i$ be the direct product of the indexed family of $R$-modules $(M_i)_{i\in I}$. The set
\begin{equation*}
\bigoplus_{i\in I}M_i=\left\lbrace (x_i)_{i\in I}\in \prod_{i\in I} M_i \middle\arrowvert x_i=0 \textnormal{ for all but finitely many } i\right\rbrace
\end{equation*}
is a submodule of the direct product.
\end{thm}

\begin{proof}
At first, we notice $0=(0)_{i\in I}\in \bigoplus_{i\in I}M_i$. Hence, $\bigoplus_{i\in I}M_i$ is not empty. Let $(x_i)_{i\in I}$ and $(y_i)_{i\in I}$ be in $\bigoplus_{i\in I}M_i$ then the sets
\begin{equation*}
J:=\lbrace i\in I\mid x_i\neq 0\rbrace \textnormal{ and } K:=\lbrace i\in I\mid y_i\neq 0\rbrace
\end{equation*} 
are finite per definition. Considering $(x_i)_{i\in I}+(y_i)_{i\in I}$, we find that the identity $x_l+y_l=0+0=0$ holds for all $l\in I\setminus (J\cup K)$. The set $J\cup K$ is finite since it is the union of two finite sets. It follows that $(x_i)_{i\in I}+(y_i)_{i\in I}\in \bigoplus_{i\in I}M_i$. We find $a(x_i)_{i\in I}\in \bigoplus_{i\in I}M_i$ for every $a$ in $R$, because for all $l\in I\setminus J$ the identity $ax_l=a0=0$ holds and $J$ is finite.
\end{proof}

\begin{defin}
\begin{thmlist}
\item The submodule $\bigoplus_{i\in I}M_i$ introduced in Theorem~\ref{theo.DirectSum} is called \emph{direct sum} of the modules $(M_i)_{i\in I}$.
\item If $I$ is finite, i.e. by renumbering we have $I=\lbrace 1,2,\ldots r\rbrace$, then $\bigoplus_{i\in I}M_i=\prod_{i\in I} M_i$ and the direct sum is rather denoted by $M_1\oplus M_2\oplus\ldots\oplus M_r$.
\item If $M_i=M$ for all $i\in I$ we note $M^{(I)}$ for the direct sum $\bigoplus_{i\in I}M_i$. When $I$ is finite, i.e. $|I|=r\in\N$, then $M^r=M^I=M^{(I)}$.
\end{thmlist}
\end{defin}

For every $i\in I$ we define a mapping $\iota_i\colon M_i\to \bigoplus_{i\in I}M_i$ by $\iota_i(x)=(x_j)_{j\in I}$ such that $x_i=x$ and $x_j=0$ if $j\neq i$. The mapping $\iota_i$ is homomorphism and is called \textit{embedding} in the $i$-th component. The embeddings fulfil the following universal property:

\begin{pro}[Universal Property of the Direct Sum]\label{prop.UniversalPropertyDirectSum}
Let $(\varphi_i)_{i\in I}$ be an indexed family of module-homomorphisms $\varphi_i\colon M_i\to M$, then there exists a unique homomorphism $\varphi\colon\bigoplus_{i\in I}M_i\to M$ such that $\varphi_i=\varphi\circ\iota_i$.
\end{pro}

One can visualize the proposition above with the following commutative diagram.
\begin{center}
\begin{tikzpicture}
  \node (Mi) at ( 0,1.5)  {$Mi$};
  \node (M) at (3,1.5) {$M$}
    	edge [<-] node[above,comment] {$\varphi_i$} (Mi);
  \node (SMi) at (0,0) {$\bigoplus\limits_{i\in I}M_i$}
  	edge [->, dashed] node[below,comment] {$\exists !\varphi$} (M)
  	edge [<-] node[left,comment] {$\iota_i$} (Mi);
\end{tikzpicture}
\end{center}


\begin{proof}
Consider the mapping 
\[\varphi\colon\bigoplus_{i\in I}M_i\to M, \; (x_i)_{i\in I}\mapsto\sum_{i\in I}x_i.\]
Note that the sum on the right side is well-defined because only a finite number of the addends $x_i$ is unequal to 0. Furthermore, $\varphi$ is a homomorphism that fulfils $\varphi_i=\varphi\circ\iota_i$ for all $i\in I$. The uniqueness follows from the construction of $\varphi$.
\end{proof}

%\todo{Hier k\"onnte noch eine Abhandlung dar\"uber stehen, dass die universelle Eigenschaft direkte Summen und Produkte bis auf Isomorphie eindeutig bestimmt}
%\begin{tikzpicture}
%  \node (Mi) at (0,0)  {$M_i$};
%  \node (P) at (1.5,-2.5) {$P$}
%    	edge [->] node[right,comment] {$p_i$} (Mi);
%  \node (PMi) at (-1.5,-2.5) {$\prod\limits_{i\in I}M_i$}
%  	edge [->, bend left] node[above,comment] {$\exists !\psi$} (P)
%  	edge [<-, bend right] node[below,comment] {$\exists !\varphi$} (P)
%  	edge [->] node[left,comment] {$\pi_i$} (Mi);
%\end{tikzpicture}
%
%\begin{tikzpicture}
%  \node (Mi) at (0,0)  {$M_i$};
%  \node (S) at (1.5,-2.5) {$S$}
%    	edge [<-] node[right,comment] {$j_i$} (Mi);
%  \node (SMi) at (-1.5,-2.5) {$\sum\limits_{i\in I}M_i$}
%  	edge [->, bend left] node[above,comment] {$\exists !\varphi$} (S)
%  	edge [<-, bend right] node[below,comment] {$\exists !\psi$} (S)
%  	edge [<-] node[left,comment] {$\iota_i$} (Mi);
%\end{tikzpicture}

		
\chapter{Generating Sets and Bases of Modules} \label{sec.GenSets}

	\section{Basic Definitions}

		% !TeX encoding = UTF-8
% !TeX spellcheck = en_GB
\begin{defin}
Let $M$ be an $R$-Module and $S\subseteq M$ a subset.
\begin{thmlist}
\item The submodul \begin{equation*}
\langle S \rangle_R := \bigcap\limits_{\substack{N\leq M\\S\subseteq N}}N
\end{equation*}  is called the \emph{submodule generated by $S$ in $M$}. This is the smallest submodule of $M$ that contains $S$.
\item We define $\langle\emptyset\rangle_R:=\lbrace 0\rbrace$.
\item If $\langle S \rangle_R=M$, we say $M$ is \emph{generated} by $S$ or \emph{$S$ is a generating set of $M$}.
\item If $S$ is finite, i.\,e., $S=\lbrace \seq{x}\rbrace\subseteq M$, we write $\langle\seq{x}\rangle_R$ instead of $\langle S \rangle_R$.
\item If there exist $\seq{x}\in M$ such that $\langle\seq{x}\rangle_R=M$, we say \emph{$M$ is finitely generated}.
\end{thmlist}
\end{defin}
Note that this definition is valid because of the findings in \cref{lm.IntersectionOfSubmods}. The following lemma provides a different characterization of generating sets.

\begin{thm}\label{lem.Generating Set Homomorphism}
Given an arbitrary index set $I$, such that $(x_i)_{i\in I}$ is a family of elements in an $R$-module $M$, then the mapping
\begin{equation*}
\gamma\colon R^{(I)}\to M \text{ defined by } (a_i)_{i\in I}\mapsto \sum_{i\in I} a_i x_i
\end{equation*}
is a module homomorphism with $\im(\gamma)=\langle ((x_i)_{i\in I}\rangle_R$.
\end{thm}
In the theorem above the ring $R$ is viewed as a module over itself.
\begin{proof}
Consider the family of mappings $\varphi_{x_i}\colon R\to M$ defined by $a\mapsto ax_i$. Module axiom (ii) and (iv) imply, that $\varphi_{x_i}$ is a homomorphism. The mapping $\gamma\colon R^{(I)}\to M$ defined by $(a_i)_{i\in I}\mapsto\sum_{i\in I} a_ix_i$ is a homomorphism, such that $\varphi_{x_i}=\gamma\circ\iota_i$, where $\iota_i\colon R\to R^{(I)}$ is the embedding of $R$ in the $i$-th component of $R^{(I)}$, for all $i\in I$, by \cref{prop.UniversalPropertyDirectSum}.

To prove $\im(\gamma)=\langle (x_i)_{i\in I}\rangle_R$ we note that $\im(\gamma)$ is a submodule of $M$ that contains $x_i$ for all $i\in I$. Hence, $\im(\gamma)\supseteq\langle (x_i)_{i\in I}\rangle_R$. On the other hand, let $N$ be a submodule of $M$ that contains $\lbrace x_i\mid i\in I\rbrace$. Then $\sum_{i\in J} a_i x_i\in N$ for all finite subsets $J\subseteq I$, which implies $\im(\gamma)=\langle (x_i)_{i\in I}\rangle_R$ because $\im(\gamma)=\lbrace\sum_{i\in J} a_i x_i| J\subseteq I, J \text{ finite}\rbrace$.
\end{proof}

We note that $(x_i)_{i\in I}$ is a generating set of the $R$-module $M$ iff $\gamma$ is surjective. This together with \cref{cor.Homomorphism Theorem} implies that 
\begin{equation*}
M\cong\faktor{R^{(I)}}{\ker(\gamma)}.
\end{equation*}


\begin{defin}
Let $(x_i)_{i\in I}$ be an indexed family in an $R$-mo\-dule $M$ and $\gamma\colon R^{(I)}\to M$ the homomorphism introduced in \cref{lem.Generating Set Homomorphism}.
\begin{thmlist}
\item We say the family $(x_i)_{i\in I}$ is \emph{linearly independent over $R$} if $\gamma$ is injective.
\item The family $(x_i)_{i\in I}$ is called a \emph{basis of $M$ over $R$} if $\gamma$ is bijective.
\end{thmlist}
\end{defin}
The following Corollary gives alternative characterizations of linear independence. All statements are a rephrase of $\gamma$ being injective.
\begin{cor}\label{lm.Linear Independence}
Let $(x_i)_{i\in I}$ be an indexed family in an $R$-module $M$ then the following statements are equivalent.
\begin{thmlist}
\item $(x_i)_{i\in I}$ is linear independent over $R$.
\item If $\sum_{i\in I}a_i x_i=0$, where only a finite number of $a_i\in R$ is unequal to 0, then $a_i=0$ for all $i\in I$.
\item If 
\[\sum_{i\in I}a_i x_i=\sum_{i\in I}b_i x_i,\]
where only a finite number of $a_i,b_i\in R$ is unequal to 0, then $a_i=b_i$ for all $i\in I$.\label{lm.All Scalars are 0}
\item For every strict subset $J\subsetneq I$ the submodule $\langle (a_i)_{i\in J}\rangle_R$ is a strict subset of $\langle (a_i)_{i\in I}\rangle_R$.
\end{thmlist}
\end{cor}

Before we give examples of finitely generated modules and bases of modules we will briefly study homomorphisms of finitely generated modules.

\begin{thm}\label{theo.Homomorphisms of free modules}
Let $\varphi\colon M\to M'$ be a homomorphism of $R$-modules and let $(v_i)_{i\in I}$ be a generating set of $M$ then $\varphi$ is defined uniquely by the images $\varphi(v_i)$, $i\in I$.
\end{thm}
\begin{proof}
Given an arbitrary $x\in M$ we find a finite subset $J\subseteq I$ and elements $a_j\in R$ for all $j\in J$ such that $\sum_{j\in J}a_j v_j=x$. Considering $\varphi(x)$ we notice
\begin{equation*}
\varphi(x)=\varphi\left(\sum_{j\in J}a_j v_j\right)=\sum_{j\in J}\varphi(a_jv_j)=\sum_{j\in J}a_j\varphi(v_j).
\end{equation*}
\end{proof}

\begin{defin}
\begin{thmlist}
\item A family of module-homomorphisms
\[(\varphi_i\colon M_i\to M_{i+1})_{i\in I},\]
where $I\subseteq \Z$ is an interval, is called \emph{exact sequence} if $\im(\varphi_i)=\ker(\varphi_{i+1})$ for all $i\in I$ (excluding the maximum of $I$ if it exists).
\item A \emph{short exact sequence} is an exact sequence where $I=\lbrace0,1,2,3\rbrace$ and $M_0=M_4=\lbrace 0\rbrace$.
We will write
\begin{equation*}
0\longrightarrow M_1\overset{\varphi_1}{\longrightarrow} M_2\overset{\varphi_2}{\longrightarrow} M_3\longrightarrow 0
\end{equation*}
for such  a sequence.
\end{thmlist}
\end{defin}

Note that $\varphi_0$ and $\varphi_3$ in the second part of the definition above are uniquely defined by their domain or their image, respectively. Furthermore, $\varphi_1$ is injective and $\varphi_2$ surjective.

\begin{lem}\label{lem.Short Exact Sequence of Finitely Generated Modules}
Let
\begin{equation*}
0\longrightarrow M'\overset{\varphi_1}{\longrightarrow} M\overset{\varphi_2}{\longrightarrow} M''\longrightarrow 0
\end{equation*}
be a short exact sequence of $R$-modules, then
\begin{thmlist}
\item $M''$ is finitely generated if $M$ is finitely generated and
\item if $M'$ and $M''$ are finitely generated, then $M$ is finitely generated.\label{lem.M' M'' implies M}
\end{thmlist}
\end{lem}

\begin{proof}
\begin{plist}
\item Let $\seq{v}\in M$ be a generating set of $M$ over $R$ then \cref{theo.Homomorphisms of free modules} implies that every $x\in M''$ can be written as 
\[x=\sum_{i=1}^n a_i\varphi_2(v_i)\]
for some $a_i\in R$ because $\varphi_2$ is surjective. Therefore, $\varphi_2(v_1),\ldots\varphi_2(v_n)$ is a finite generating set of $M''$.
\item Let $\seq[r]{u}\in M'$ be a generating set of $M'$ and let $\seq[s]{w}\in M''$ be a generating set of $M''$.
We will prove that
\[\varphi_1(u_1),\ldots,\varphi_1(u_r),v_1,\ldots,v_s\in M\]
where $v_i\in\varphi_2^{-1}(w_i)$ for $i\in\lbrace 1,\ldots,s\rbrace$, is a generating set of $M$ over $R$.
Let therefore $x$ be an arbitrary element of $M$.
Since $\seq[s]{w}$ is a generating set of $M''$ we can write 
\begin{equation*}
\varphi_2(x)=\sum_{i=1}^s a_i w_i=\varphi_2\left(\vphantom{\sum_{i=1}^s}\right.
\underbrace{\sum_{i=1}^s a_i v_i}_{\in M}
\left. \vphantom{\sum_{i=1}^s}\right)
\end{equation*}
for some $\seq[s]{a}\in R$. Since $\nicefrac{M}{\ker(\varphi_2)}\cong M''$ (with respect to $\bar{\varphi}_2$ see \cref{theo.Universal Property} and \cref{cor.Homomorphism Theorem}) we find $x\in \sum_{i=1}^s a_i v_i+\ker(\varphi_2)$ and therefore $x=\sum_{i=1}^s a_i v_i+k$ where $k\in\ker(\varphi_2)$. Because $\im(\varphi_1)=\ker(\varphi_2)$ part (i) implies that $\varphi_1(u_1),\ldots,\varphi_1(u_r)$ is a generating set of $\ker(\varphi_2)\subseteq M$. Hence, $x=\sum_{i=1}^s a_i v_i+\sum_{j=1}^s b_j \varphi_1(u_j)$ for some $\seq[s]{b}\in R$. Which proves the claim.
\end{plist}
\end{proof}

One of the main results in linear algebra is that every vector space over a field has a basis over that field. \cref{exam.basis of G} shows that a general $R$-module does not have a basis.

\begin{exam}\label{exam.Bases of Modules}
\begin{exlist}
\item Let $R$ be a unital ring, then $\lbrace 1\rbrace$ is a basis of $R$, because $a=a1$ for all $a\in R$ and this decomposition is unique.
\item Let $V$ be an $F$-vector space. In \cref{exam.Vector Space as EndFV-module} we have seen, that $V$ is an $\End_F(V)$-module. If $\seq{b}\in V$ is a basis of $V$ over $F$, then $\lbrace b_1 \rbrace$ is a generating set of $V$ over $\End_F(V)$, because for every $v\in V$ we can define a linear mapping 
\begin{equation*}
\varphi_v\colon V\to V \text{ by } \varphi_v(b_i):= \begin{cases}v & \text{if } i=1\\ 0 & \text{if } i\neq 1\end{cases}.
\end{equation*}
Therefore the mapping $\gamma\colon \End_F(V)\to V$ defined by $\varphi\mapsto \varphi(b_1)$ is surjective and not injective if $n>1$.

Note that this implies that $V$ contains only the trivial submodules because for every $x\in V\setminus\lbrace 0\rbrace$ we can find a basis $\seq{\hat{b}}$ such that $x=\hat{b}_1$ and $\langle x\rangle_{\End_F(V)}=V$ is the smallest submodule that contains $x$. \label{exam. Generating set of V over EndFV}
\item Let $G\neq\lbrace 0\rbrace$ be a finite group viewed as a $\Z$-module, then $G$ does not have a basis over $\Z$. Let $n=|G|$. We assume that $G$ has a basis $B$. If $B$ is empty then $\langle B\rangle_{\Z} =\lbrace 0\rbrace$ which is a contradiction to $G\neq\lbrace 0\rbrace$. If, on the other hand, $B$ is non-empty with $x\in B$ we consider the $n+1$ elements $x,2x,\ldots,(n+1)x$. Since $n=| G|$ there exist $1\leq k<\ell\leq n+1$ such that $kx=\ell x$. Which is a contradiction to the assumption that $B$ is a basis of $G$ over $\Z$.\label{exam.basis of G}
\end{exlist}
\end{exam}

Finally, we will see that the reverse implication of \cref{lem.M' M'' implies M} does not hold in general, i.\,e., there exist finitely generated modules which contain non finitely generated submodules. The following example can be found in \cite[p.197]{jantzen2005algebra}.

\begin{exam}
Let $F$ be a field. We consider $R=F[X_1,X_2,\ldots]$ the polynomial ring with countable infinitely many variables $X_i$ over $K$. We consider $R$ as a module over itself and find that $\lbrace 1\rbrace$ is a generating set of $R$ over itself.

The set $\mathfrak{I}$ of all polynomials in $R$ with constant term $a_0=0$ is an ideal in $R$ and therefore a submodule. But there is no finite generating set of $\mathfrak{I}$. If we assume that $\lbrace\seq{p}\rbrace\subseteq \mathfrak{I}$ is a generating set of $\mathfrak{I}$ we note that only a finite number $m\in\N$ of the variables $(X_i)_{i\in\N}$ appear in $\lbrace\seq{p}\rbrace$. We denote these $m$ variables by $X_{i_1},\ldots,X_{i_m}$. Hence, there exists a natural number $k\in\N$ such that $X_k$ does not appear in $\lbrace\seq{p}\rbrace$ and so there do not exist $\seq{a}\in R$ such that $X_k=\sum_{i=1}^n a_ip_i$ because only polynomials containing at least one $X_{i_\ell}$ ($1\leq\ell\leq k$) can be represented by such a sum. It follows that$\langle\seq{p}\rangle_R\subsetneq \mathfrak{I}$.
\end{exam}

	\section{Free Modules}
	
		Example~\secref{exam.Bases of Modules}{exam.basis of G} leads directly to the following definition.
\begin{defin}
An $R$-module $M$ is called \emph{free} over $R$ if it has a base over $R$.
\end{defin}

Theorem~\ref{theo.Homomorphisms of free modules} implies a different approach of characterising mono-, epi- and isomorphisms of free modules.
\begin{cor}
Let $\varphi\colon M\to M'$ be a homomorphism of $R$-modules and $B:=(v_i)_{i\in I}$ a basis of $M$ then the following statements hold.
\begin{compactenum}
\item $\varphi$ is injective iff $\varphi(B):=(\varphi(v_i))_{i\in I}$ is linear independent in $M'$ (over $R$).
\item $\varphi$ is surjective iff $\varphi(B)$ is a generating set of $M'$.
\item $\varphi$ is bijective iff $\varphi(B)$ is a basis of $M'$.
\end{compactenum}
\end{cor}
\begin{proof}
We consider the following diagram where $\gamma_{\varphi(B)}$ denotes the homomorphism $\gamma_{\varphi(B)}(a_i)_{i\in I}:=\sum_{i\in I}a_i\varphi(v_i)$.
\begin{center}
\begin{tikzpicture}
  \node (M) at ( 0,1.7)  {$M$};
  \node (M2) at (2,1.7) {$M'$}
    	edge [<-] node[above,comment] {$\varphi$} (M);
  \node (RI) at (1,0) {$R^{(I)}$}
  	edge [->] node[left,comment] {$\gamma_B$} (M)
  	edge [->] node[right,comment] {$\gamma_{\varphi(B)}$} (M2);
\end{tikzpicture}
\end{center}
This diagram commutates because of the findings in Theorem~\ref{theo.Homomorphisms of free modules}. Since $\gamma_B$ is bijective $\gamma_{\varphi(B)}$ is injective or surjective respectively if and only if $\varphi$ is injective or surjective, which proves the claim.
\end{proof}
The proposition above implies that if $M'$ is isomorphic to the free module $M$ then $M'$ is free.


The following example, which can be found in \cite[p.194]{jantzen2005algebra}, shows that the concept of dimensions as we know it for vector spaces can not be generalized for free modules over arbitrary rings. We will see that the cardinality of a basis of a free $R$-module is in general not unique.
\begin{exam}
Let $F$ be a field then the polynomial ring $F[X]$ is an $F$-vector space and the set $R:=\End_F(F[X])$ of all endomorphisms in $F[X]$ is a ring with respect to point-wise addition and composition of mappings. We view $R$ as a module over itself and find that $\id_{F[X]}$ is a basis of $R$ over $R$, since for every $\varphi\in R$ the identity $\varphi=\varphi\circ id_{F[X]}$ holds and this decomposition is unique.

On the other hand we consider the endomorphisms $f_1,f_2\in R$ defined by
\begin{eqnarray*}
f_1(X^i):=\begin{cases} X^{\frac{i}{2}} & \text{if } 2\mid i\\ 0&\text{if } 2\nmid i\\\end{cases}, &  f_2(X^i):=\begin{cases} 0 & \text{if } 2\mid i\\ X^{\frac{i-1}{2}}&\text{if } 2\nmid i\\\end{cases}.
\end{eqnarray*}
The set $\lbrace f_1,f_2\rbrace$ is linearly independent over $R$ because $g_1\circ f_1+g_2\circ f_2=\mathbf{0}$ iff $g_1=g_2=\mathbf{0}$. To prove this we note that $\mathbf{0}$ denotes the mapping $F[X]\to F[X],\; p\mapsto 0$ and assume $g_1\neq\mathbf{0}$. Hence, there exists an $i\in\N$ such that $g_1(X^i)\neq 0$. Therefore, $(g_1\circ f_1+g_2\circ f_2)(X^{2i})=(g_1\circ f_1)(X^{2i})+(g_2\circ f_2)(X^{2i})=g_1(X^i)+0\neq 0$ which contradicts $g_1\circ f_1+g_2\circ f_2=\mathbf{0}$. The assumption $g_2\neq \mathbf{0}$ leads to the same contradiction. Eventually, we verify that $\lbrace f_1,f_2\rbrace$ is a basis of $R$. Let therefore $\varphi$ be an arbitrary endomorphism in $R$. We consider $g_1,g_2\in R$ defined by $g_1(X^i)=\varphi(X^{2i})$ and $g_2(X^i)=\varphi(X^{2i+1})$. These endomorphisms fulfil $g_1\circ f_1+g_2\circ f_2=\varphi$.

Consequently, $R$ has a basis containing one element and a basis containing two elements which implies $R^1\cong R^2$. It follows per induction on $n$ that $R\cong R^n$ for all $n\in\N$.
\end{exam} 

We will now show that the cardinality of a finite basis of an $R$-module is unique, if $R\neq\lbrace 0\rbrace$ and $R$ is commutative.

\begin{lemma}\label{lm.SigmaModule}
Let $M$ be an $R$-module and $\mathfrak{I}$ a left ideal in $R$ then the following statements hold.
\begin{compactenum}
\item The subset \label{lm.Sigma}
\begin{equation*}
S:=\left\lbrace\sum_{i=1}^n \mathfrak{a}_ix_i\mid n\in\N, \mathfrak{a}_i\in \mathfrak{I}, x_i\in M, 1\leq i\leq n\right\rbrace
\end{equation*}
is a submodule of $M$.
\item If $\mathfrak{I}$ is a (two-sided) ideal in $R$ then $\nicefrac{M}{S}$ is an $\nicefrac{R}{\mathfrak{I}}$-module, with respect to the scalar multiplication
\begin{equation*}
(a+\mathfrak{I})(x+S):=ax+S.
\end{equation*}
\end{compactenum}
\end{lemma}
\begin{proof}
\begin{enumerate}
\item Let $J,K$ be finite subsets of $\N$ and $\mathfrak{a}_j\in \mathfrak{I}$ for all $j\in J$, $\mathfrak{b}_k\in \mathfrak{I}$ for all $k\in K$ and  $x_i\in M$ for all $i\in J\cup K$. We define $\mathfrak{a}_j:=0$ if $j\in K \setminus J$ and $\mathfrak{b}_k:=0$ if $k\in J\setminus K$. It follows that $\sum_{j\in J}\mathfrak{a}_jx_j\in S$, $\sum_{k\in K}\mathfrak{b}_kx_k\in S$ and therefore
\begin{equation*}
\sum_{j\in J}\mathfrak{a}_jx_j+\sum_{k\in K}\mathfrak{a}_kx_k=\sum_{i\in J\cup K} \mathfrak{a}_ix_i+\sum_{i\in J\cup K} \mathfrak{b}_ix_i=\sum_{i\in J\cup K} (\mathfrak{a}_i+\mathfrak{b}_i)x_i\in S.
\end{equation*} 
 Given an arbitrary $c\in R$ we find $c\sum_{j\in J}\mathfrak{a}_jx_j=\sum_{j\in J}c\mathfrak{a}_jx_j\in S$ since $\mathfrak{I}$ is a left ideal in $R$.
\item At first we notice that $(\nicefrac{M}{S}, +)$ is an Abelian group since $\nicefrac{M}{S}$ is an $R$-module.

Secondly, we need to verify the validity of the definition of the scalar multiplication. Let $a\equiv a'$ with respect to the equivalence relation on $R$ induced by $\mathfrak{I}$. We show that $ax\equiv a'x$ with respect to the equivalence relation on $M$ induced by $S$ for all $x\in M$. Since $a\equiv a'$ there exists an element $\mathfrak{b}\in\mathfrak{I}$ such that $a=a'+\mathfrak{b}$, therefore $ax-a'x=ax-(a-\mathfrak{b})x=\mathfrak{b}x\in S$ and the claim is proven. On the other hand, let $x\equiv x'$. One needs to verify $ax\equiv ax'$ for all $a\in R$. Since $x\equiv x'$ there exists a sum $\sum_{i=1}^n\mathfrak{b}_iy_i\in S$ such that $x=x'+\sum_{i=1}^n\mathfrak{b}_iy_i$ and therefore $ax-ax'=a\sum_{i=1}^n\mathfrak{b}_iy_i=\sum_{i=1}^na\mathfrak{b}_iy_i\in S$ since $\mathfrak{b}_i\in\mathfrak{I}$ and $\mathfrak{I}$ is an ideal.

The four module axioms are now proven in a straight forward way.
\end{enumerate}
\end{proof}

\begin{lemma}\label{lm.basis of M to Sigma}
Let $M$ be a free $R$-module and $\mathfrak{I}$ an ideal in $R$. If we consider a basis $\seq{v}$ of $M$ over $R$ and the submodule $S$ like in Lemma~\secref{lm.SigmaModule}{lm.Sigma}, then $v_1+S,\ldots,v_n+S$ is a basis of $\nicefrac{M}{S}$ over $\nicefrac{R}{\mathfrak{I}}$.
\end{lemma}
\begin{proof}
The family $v_1+S,\ldots,v_n+S$ is a generating set of $\nicefrac{M}{S}$ over $\nicefrac{R}{\mathfrak{I}}$ because for an arbitrary $x+S\in\nicefrac{M}{S}$ there exists a decomposition $x=\sum_{i=1}^na_iv_i$ which leads us to 
\begin{equation*}
\sum_{i=1}^n(a_i+\mathfrak{I})(v_i+S)=\sum_{i=1}^n(a_iv_i+S)=\sum_{i=1}^na_iv_i+S=x+S.
\end{equation*}

Let $a_1+\mathfrak{I},\ldots,a_n+\mathfrak{I}$ be elements of $\nicefrac{R}{\mathfrak{I}}$ that fulfil $\sum_{i=1}^n(a_i+\mathfrak{I})(v_i+S)=S$. We will prove $a_i\in\mathfrak{I}$ for $1\leq i\leq n$ then Corollary~\ref{lm.Linear Independence} implies the claim. We note that 
\begin{equation*}
0+S=S=\sum_{i=1}^n(a_i+\mathfrak{I})(v_i+S)=\left(\sum_{i=1}^n a_iv_i\right)+S.
\end{equation*}
It follows that there exists an $s\in S$ such that $0+s=\sum_{i=1}^n a_iv_i$. Let $s=\sum_{j=1}^m\mathfrak{b}_jx_j$ for $\mathfrak{b}_j\in \mathfrak{I}$ and $x_j\in M$, then there exist $c_{ij}\in R$ for $1\leq i\leq n$ and $1\leq j\leq m$ such that $x_j=\sum_{i=1}^mc_{ij}v_i$. It follows that
\begin{equation*}
\sum_{i=1}^n a_iv_i=s=\sum_{j=1}^m\mathfrak{b}_j\sum_{i=1}^mc_{ij}v_i=\sum_{i=1}^n\left(\sum_{j=1}^m\mathfrak{b}_jc_{ij}\right)v_i.
\end{equation*}
Now Lemma~\ref{lm.Linear Independence} implies $a_i=\sum_{j=1}^m\mathfrak{b}_jc_{ij}\in \mathfrak{I}$ for $1\leq i\leq n$.
\end{proof}

\begin{theo}
Let $M$ be a free module over the commutative ring $R\neq\lbrace 0\rbrace$ then every basis of $M$ over $R$ has the same cardinality.
\end{theo}
\begin{proof}
Let $\seq{v}$ and $\seq[m]{w}$ be bases of $M$. Since $R$ is commutative $R$ contains a maximal ideal $\mathfrak{I}$. (see \cite[Thm.III.3.6]{jantzen2005algebra}) Therefore, $\nicefrac{R}{\mathfrak{I}}$ is a field (see \cite[p.93]{lang2002algebra}) and $\nicefrac{M}{S}$ is a vector space, if $S$ denotes the submodule of Lemma~\secref{lm.SigmaModule}{lm.Sigma}. Lemma~\ref{lm.basis of M to Sigma} implies that $v_1+S,\ldots,v_n+S$ as well as $w_1+S,\ldots,w_m+S$ are bases of $\nicefrac{M}{S}$ over $\nicefrac{R}{\mathfrak{I}}$. Since the cardinality of bases of vector spaces is unique it follows $n=m$.
\end{proof}
		
	\section{Homomorphisms of Free Modules and Matrices}
	
		% !TeX encoding = UTF-8
% !TeX spellcheck = en_GB
We will continue with a brief study of the connection between homomorphisms of free modules and matrices.

\begin{pro}\label{prop.Commutative Diagram of Free Modules}
Let $\varphi\colon M\to M'$ be a homomorphism of free $R$-modules, let the family $B:=(v_i)_{i\in I}$ be a base of $M$ and let $C:=(w_j)_{j\in J}$ be a base of $M'$. Then there exists a unique homomorphism $\overline{\varphi}\colon R^{(I)}\to R^{(J)}$ such that the following diagram, where $\gamma_B\colon R^{(I)}\to M$ and $\gamma_C\colon R^{(J)}\to M'$ denote mappings like the one in \cref{lem.Generating Set Homomorphism}, commutates.
\begin{center}
\begin{tikzpicture}
  \node (M) at ( 0,2)  {$M$};
  \node (M2) at (2,2) {$M'$}
    	edge [<-] node[above,comment] {$\varphi$} (M);
  \node (RI) at (0,0) {$R^{(I)}$}
  	edge [->] node[left,comment] {$\gamma_B$} (M);
  \node (RJ) at (2,0) {$R^{(J)}$}
  	edge [<-, dashed] node[below,comment] {$\exists !\overline{\varphi}$} (RI)
  	edge [->] node[right,comment] {$\gamma_C$} (M2);
\end{tikzpicture}
\end{center}
\end{pro}
\begin{proof}
Since $B$ is a base of $M$ and $C$ is a base of $M'$ the mappings $\gamma_B$ and $\gamma_C$ are isomorphisms. We define a homomorphism by $\overline{\varphi}:=\gamma_C^{-1}\circ\varphi\circ\gamma_B$. Clearly this homomorphism satisfies $\gamma_C\circ\overline{\varphi}\circ\gamma_B^{-1}=\varphi$ which proves the existence of such a homomorphism. Moreover, $\overline{\varphi}$ is unique by construction.
\end{proof}

\begin{cor}\label{cor.Matrix implies Homomorphism}
Let $R$ be a commutative unital ring and $R^n$ a direct sum of $R$ then every matrix $A\in\mathrm{M}_{m\times n}(R)$ induces a homomorphism defined by
\begin{equation*}
\varphi_A\colon R^n\to R^m,\; x\mapsto Ax.
\end{equation*}
\end{cor}
\begin{proof}
Given an arbitrary pair $x,y\in R^n$ and an $a\in R$ the following identities hold
\begin{exlist}
\item $\varphi_A(x+y)=A(x+y)=Ax+Ay=\varphi_A(x)+\varphi_A(y)$,
\item $\varphi_A(ax)=A(ax)=aAx=a\varphi_A(x)$.
\end{exlist}
\end{proof}

\begin{rem}
The set $\End_R(M)$ of all endomorphisms in a free $R$-mod\-ule $M$ is a ring with respect to point-wise addition and composition  by \cref{lm.Composition of Homo} and \cref{prop.Homomorphisms are Group}.
If $M$ has a finite base over $R$ there exists an $n\in\N$ such that $M\cong R^n$. In \cref{prop.Commutative Diagram of Free Modules} we have seen that there exists a bijection $\Gamma\colon\End_R(M)\to\End_R(R^n)$ defined by $\varphi\mapsto\gamma^{-1}\circ\varphi\circ\gamma$.
This mapping is a isomorphism of rings because for all $\varphi,\psi\in\End_R(M)$ the following identities are satisfied:
\begin{exlist}
\item $\Gamma(\varphi+\psi)=\gamma^{-1}\circ(\varphi+\psi)\circ\gamma =\gamma^{-1}\circ(\varphi\circ\gamma+\psi\circ\gamma)= (\gamma^{-1}\circ\varphi\circ\gamma)+(\gamma^{-1}\circ\psi\circ\gamma)=\Gamma(\varphi)+\Gamma(\psi)$ and
\item $\Gamma(\varphi\circ\psi)=\gamma^{-1}\circ(\varphi\circ\psi)\circ\gamma =\gamma^{-1}\circ\varphi\circ\gamma^{-1}\circ\gamma\circ\psi\circ\gamma =\Gamma(\varphi)\circ\Gamma(\psi)$.
\end{exlist}
\end{rem}

\begin{pro}\label{prop.Matrix isomorph Homomorphism}
Let $R$ be a commutative ring with 1, then $R^n$ is a module and the following isomorphism of rings exists:
\begin{equation*}
\End_R(R^n)\cong \mathrm{M}_n(R).
\end{equation*}
\end{pro}
\begin{proof}
We study the mapping $\left[\;\cdot\;\right]\colon\End_R(R^n)\to\mathrm{M}_n(R)$ defined by
\begin{equation*}
[\varphi]:=
\begin{pmatrix}
\pi_1(\varphi(e_1)) & \pi_1(\varphi(e_2)) & \ldots & \pi_1(\varphi(e_n))\\
\pi_2(\varphi(e_1)) & \pi_2(\varphi(e_2)) & \ldots & \pi_2(\varphi(e_n))\\
\vdots &\vdots & &\vdots\\
\pi_n(\varphi(e_1)) & \pi_n(\varphi(e_2)) & \ldots & \pi_n(\varphi(e_n))
\end{pmatrix},
\end{equation*}
where $e_i\in R^n$ denotes the family $\left(e_{i1}\;\ldots\;e_{in}\right)^t$, $e_{ii}=1$, $e_{ij}=0$ if $i\neq j$ and $\pi_i$ is the projection on the $i$-th component. Since the $j$-th column of the matrix $[\varphi]$ equals $\varphi(e_j)$ we define the abbreviation $[\varphi]=\left(\varphi(e_1)\;\ldots\varphi(e_n)\;\right)$ for this matrix. The mapping $\left[\;\cdot\;\right]$ is a ring-homomorphism, because for every pair of module-homomorphisms $\varphi,\psi\in\End_R(R^n)$ the following identities hold:
\begin{align*}
[\varphi+\psi]&=\begin{pmatrix}(\varphi+\psi)(e_1) & \dots & (\varphi+\psi)(e_n)\end{pmatrix}=\\
    &=\begin{pmatrix}\varphi(e_1)+\psi(e_1) & \dots & \varphi(e_n)+\psi(e_n)\end{pmatrix}=\\
    &=\begin{pmatrix}\varphi(e_1) & \ldots & \varphi(e_n)\end{pmatrix}+\begin{pmatrix}\psi(e_1) & \ldots & \psi(e_n)\end{pmatrix}=\\
    &=[\varphi]+[\psi]
\end{align*}
and
\begin{align*}
&[\varphi\circ\psi]=\begin{pmatrix}(\varphi\circ\psi)(e_1) & \ldots & (\varphi\circ\psi)(e_n)\end{pmatrix}=\\
&=\begin{pmatrix}\varphi(\psi(e_1)) & \ldots & \varphi(\psi(e_n))\end{pmatrix}=\\
&=\begin{pmatrix}\varphi(\sum_{j=1}^n\underbrace{\pi_j(\psi(e_1))}_{\in R}e_j) & \ldots & \varphi(\sum_{j=1}^n\underbrace{\pi_j(\psi(e_n))}_{\in R}e_j)\end{pmatrix}=\\
&=\begin{pmatrix}\sum_{j=1}^n\pi_j(\psi(e_1))\varphi(e_j) & \ldots & \sum_{j=1}^n\pi_j(\psi(e_n))\varphi(e_j)\end{pmatrix}=\\
&=\left(\begin{smallmatrix}
\pi_1(\sum_{j=1}^n\pi_j(\psi(e_1))\varphi(e_j)) & \ldots & \pi_1(\sum_{j=1}^n\pi_j(\psi(e_n))\varphi(e_j))\\
\vdots & &\vdots\\
\pi_n(\sum_{j=1}^n\pi_j(\psi(e_1))\varphi(e_j)) & \ldots & \pi_n(\sum_{j=1}^n\pi_j(\psi(e_n))\varphi(e_j))
\end{smallmatrix}\right)=\\
&=\left(\begin{smallmatrix}
\sum_{j=1}^n\pi_j(\psi(e_1))\pi_1(\varphi(e_j)) & \ldots & \sum_{j=1}^n\pi_j(\psi(e_n))\pi_1(\varphi(e_j))\\
\vdots & &\vdots\\
\sum_{j=1}^n\pi_j(\psi(e_1))\pi_n(\varphi(e_j)) & \ldots & \sum_{j=1}^n\pi_j(\psi(e_n))\pi_n(\varphi(e_j))
\end{smallmatrix}\right)=\\
&=[\varphi][\psi].
\end{align*}
The homomorphism $[\;\cdot\;]$ is injective because $\ker([\;\cdot\;])=\lbrace 0\rbrace$. \cref{cor.Matrix implies Homomorphism} implies the surjectivity.
\end{proof}

\cref{prop.Matrix isomorph Homomorphism} implies that we can identify endomorphisms of free modules over commutative rings with matrices over these rings. Note that this proposition can not be generalized to free modules over non-commutative rings because  \cref{cor.Matrix implies Homomorphism} does not hold for non-commutative rings.




\vspace{\fill}
\printbibliography[heading=subbibliography]
