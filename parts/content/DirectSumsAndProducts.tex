% !TeX encoding = UTF-8
% !TeX spellcheck = en_GB
Given $(M_i)_{i\in I}$ a family of $R$-modules, indexed by $I$. The set of all indexed families $(x_i)_{i\in I}$ where $x_i\in M_i$ for all $i\in I$ is called the \emph{Cartesian product of the modules $(M_i)_{i\in I}$} and is denoted by $\prod\limits_{i\in I} M_i$. If $M_i=M$ for all $i\in I$ we rather note $M^I$ for their Cartesian product.

\begin{pro}\label{prop.DirectProduct}
The Cartesian product $\prod_{i\in I} M_i$ of the family of $R$-modules $(M_i)_{i\in I}$ is an $R$-module with respect to component-wise addition and component-wise scalar-multiplication, \ie 
\[(x_i)_{i\in I}+(y_i)_{i\in I}:=(x_i+y_i)_{i\in I}\text{ and }a\ast(x_i)_{i\in I}:=(a\ast x_i)_{i\in I}.\]
\end{pro}

To prove the proposition above one has to to verify all module axioms in a straight forward way.

\begin{defin}\sloppypar
We call the $R$-module $\left(\prod_{i\in I} M_i,+,\ast\right)$ introduced in \cref{prop.DirectProduct} the \emph{direct product} of the family of modules $(M_i)_{i\in I}$.
\end{defin}

For every $j\in I$ we call the mapping $\pi_j\colon \prod_{i\in I} M_i\to M_j$ defined by $\pi_j((x_i)_{i\in I})=x_j$ \emph{projection} on the factor $M_j$. One verifies easily that $\pi_j$ is a homomorphism for every $j$ in $I$. These projections fulfil the following universal property:

\begin{pro}[universal prop.\ of the direct product]
Let $(\varphi_i)_{i\in I}$ be an indexed family of module-homomorphisms $\varphi_i\colon M\to M_i$. Then there exists a unique homomorphism $\varphi\colon M\to \prod_{i\in I} M_i$ such that $\varphi_i=\pi_i\circ\varphi$ for all $i\in I$.
\end{pro}

One can visualize the proposition above with the following commutative diagram.

\begin{center}
\begin{tikzpicture}
  \node (N) at ( 0,1.5)  {$M$};
  \node (Mi) at (3,1.5) {$Mi$}
    	edge [<-] node[above,comment] {$\varphi_i$} (N);
  \node (MMi) at (3,0) {$\prod\limits_{i\in I}M_i$}
  	edge [<-, dashed] node[below,comment] {$\exists !\varphi$} (N)
  	edge [->] node[right,comment] {$\pi_i$} (Mi);
\end{tikzpicture}
\end{center}

\begin{proof}
Consider the mapping 
\[\varphi\colon M\to \prod_{i\in I} M_i, \; x \mapsto (\varphi_i(x))_{i\in I}.\]
It is a homomorphism and fulfils $\varphi_i=\pi_i\circ\varphi$ for all $i\in I$. The uniqueness follows from the construction of $\varphi$.
\end{proof}

\begin{thm}\label{theo.DirectSum}
Let $\prod_{i\in I} M_i$ be the direct product of the indexed family of $R$-modules $(M_i)_{i\in I}$. The set
\begin{equation*}
\bigoplus_{i\in I}M_i=\left\lbrace (x_i)_{i\in I}\in \prod_{i\in I} M_i \middle\arrowvert x_i=0 \textnormal{ for all but finitely many } i\right\rbrace
\end{equation*}
is a submodule of the direct product.
\end{thm}

\begin{proof}
At first, we notice $0=(0)_{i\in I}\in \bigoplus_{i\in I}M_i$. Hence, $\bigoplus_{i\in I}M_i$ is not empty. Let $(x_i)_{i\in I}$ and $(y_i)_{i\in I}$ be in $\bigoplus_{i\in I}M_i$ then the sets
\begin{equation*}
J:=\lbrace i\in I\mid x_i\neq 0\rbrace \textnormal{ and } K:=\lbrace i\in I\mid y_i\neq 0\rbrace
\end{equation*} 
are finite per definition. Considering $(x_i)_{i\in I}+(y_i)_{i\in I}$, we find that the identity $x_l+y_l=0+0=0$ holds for all $l\in I\setminus (J\cup K)$. The set $J\cup K$ is finite since it is the union of two finite sets. It follows that $(x_i)_{i\in I}+(y_i)_{i\in I}\in \bigoplus_{i\in I}M_i$. We find $a(x_i)_{i\in I}\in \bigoplus_{i\in I}M_i$ for every $a$ in $R$, because for all $l\in I\setminus J$ the identity $ax_l=a0=0$ holds and $J$ is finite.
\end{proof}

\begin{defin}
\begin{thmlist}
\item\sloppypar The submodule $\bigoplus_{i\in I}M_i$ introduced in \cref{theo.DirectSum} is called \emph{direct sum} of the modules $(M_i)_{i\in I}$.
\item If $I$ is finite, \ie by renumbering we have $I=\lbrace 1,2,\ldots r\rbrace$, then $\bigoplus_{i\in I}M_i=\prod_{i\in I} M_i$ and the direct sum is rather denoted by $M_1\oplus M_2\oplus\ldots\oplus M_r$.
\item If $M_i=M$ for all $i\in I$ we note $M^{(I)}$ for the direct sum $\bigoplus_{i\in I}M_i$. When $I$ is finite, \ie $|I|=r\in\N$, then $M^r=M^I=M^{(I)}$.
\end{thmlist}
\end{defin}

For every $i\in I$ we define a mapping $\iota_i\colon M_i\to \bigoplus_{i\in I}M_i$ by $\iota_i(x)=(x_j)_{j\in I}$ such that $x_i=x$ and $x_j=0$ if $j\neq i$. The mapping $\iota_i$ is homomorphism and is called \textit{embedding} in the $i$-th component. The embeddings fulfil the following universal property:

\begin{pro}[universal property of the direct sum]\label{prop.UniversalPropertyDirectSum}
Let $(\varphi_i)_{i\in I}$ be an indexed family of module-homomorphisms $\varphi_i\colon M_i\to M$, then there exists a unique homomorphism $\varphi\colon\bigoplus_{i\in I}M_i\to M$ such that $\varphi_i=\varphi\circ\iota_i$.
\end{pro}

One can visualize the proposition above with the following commutative diagram.
\begin{center}
\begin{tikzpicture}
  \node (Mi) at ( 0,1.5)  {$Mi$};
  \node (M) at (3,1.5) {$M$}
    	edge [<-] node[above,comment] {$\varphi_i$} (Mi);
  \node (SMi) at (0,0) {$\bigoplus\limits_{i\in I}M_i$}
  	edge [->, dashed] node[below,comment] {$\exists !\varphi$} (M)
  	edge [<-] node[left,comment] {$\iota_i$} (Mi);
\end{tikzpicture}
\end{center}


\begin{proof}
Consider the mapping 
\[\varphi\colon\bigoplus_{i\in I}M_i\to M, \; (x_i)_{i\in I}\mapsto\sum_{i\in I}x_i.\]
Note that the sum on the right side is well-defined because only a finite number of the addends $x_i$ is unequal to 0. Furthermore, $\varphi$ is a homomorphism that fulfils $\varphi_i=\varphi\circ\iota_i$ for all $i\in I$. The uniqueness follows from the construction of $\varphi$.
\end{proof}

%\todo{Hier k\"onnte noch eine Abhandlung dar\"uber stehen, dass die universelle Eigenschaft direkte Summen und Produkte bis auf Isomorphie eindeutig bestimmt}
%\begin{tikzpicture}
%  \node (Mi) at (0,0)  {$M_i$};
%  \node (P) at (1.5,-2.5) {$P$}
%    	edge [->] node[right,comment] {$p_i$} (Mi);
%  \node (PMi) at (-1.5,-2.5) {$\prod\limits_{i\in I}M_i$}
%  	edge [->, bend left] node[above,comment] {$\exists !\psi$} (P)
%  	edge [<-, bend right] node[below,comment] {$\exists !\varphi$} (P)
%  	edge [->] node[left,comment] {$\pi_i$} (Mi);
%\end{tikzpicture}
%
%\begin{tikzpicture}
%  \node (Mi) at (0,0)  {$M_i$};
%  \node (S) at (1.5,-2.5) {$S$}
%    	edge [<-] node[right,comment] {$j_i$} (Mi);
%  \node (SMi) at (-1.5,-2.5) {$\sum\limits_{i\in I}M_i$}
%  	edge [->, bend left] node[above,comment] {$\exists !\varphi$} (S)
%  	edge [<-, bend right] node[below,comment] {$\exists !\psi$} (S)
%  	edge [<-] node[left,comment] {$\iota_i$} (Mi);
%\end{tikzpicture}
