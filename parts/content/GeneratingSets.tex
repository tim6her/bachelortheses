% !TeX encoding = UTF-8
% !TeX spellcheck = en_GB
\begin{defin}
Let $M$ be an $R$-Module and $S\subseteq M$ a subset.
\begin{thmlist}
\item The submodul \begin{equation*}
\langle S \rangle_R := \bigcap\limits_{\substack{N\leq M\\S\subseteq N}}N
\end{equation*}  is called the \emph{submodule generated by $S$ in $M$}. This is the smallest submodule of $M$ that contains $S$.
\item We define $\langle\emptyset\rangle_R:=\lbrace 0\rbrace$.
\item If $\langle S \rangle_R=M$, we say $M$ is \emph{generated} by $S$ or \emph{$S$ is a generating set of $M$}.
\item If $S$ is finite, i.\,e., $S=\lbrace \seq{x}\rbrace\subseteq M$, we write $\langle\seq{x}\rangle_R$ instead of $\langle S \rangle_R$.
\item If there exist $\seq{x}\in M$ such that $\langle\seq{x}\rangle_R=M$, we say \emph{$M$ is finitely generated}.
\end{thmlist}
\end{defin}
Note that this definition is valid because of the findings in \cref{lm.IntersectionOfSubmods}. The following lemma provides a different characterization of generating sets.

\begin{thm}\label{lem.Generating Set Homomorphism}
Given an arbitrary index set $I$, such that $(x_i)_{i\in I}$ is a family of elements in an $R$-module $M$, then the mapping
\begin{equation*}
\gamma\colon R^{(I)}\to M \text{ defined by } (a_i)_{i\in I}\mapsto \sum_{i\in I} a_i x_i
\end{equation*}
is a module homomorphism with $\im(\gamma)=\langle ((x_i)_{i\in I}\rangle_R$.
\end{thm}
In the theorem above the ring $R$ is viewed as a module over itself.
\begin{proof}
Consider the family of mappings $\varphi_{x_i}\colon R\to M$ defined by $a\mapsto ax_i$. Module axiom (ii) and (iv) imply, that $\varphi_{x_i}$ is a homomorphism. The mapping $\gamma\colon R^{(I)}\to M$ defined by $(a_i)_{i\in I}\mapsto\sum_{i\in I} a_ix_i$ is a homomorphism, such that $\varphi_{x_i}=\gamma\circ\iota_i$, where $\iota_i\colon R\to R^{(I)}$ is the embedding of $R$ in the $i$-th component of $R^{(I)}$, for all $i\in I$, by \cref{prop.UniversalPropertyDirectSum}.

To prove $\im(\gamma)=\langle (x_i)_{i\in I}\rangle_R$ we note that $\im(\gamma)$ is a submodule of $M$ that contains $x_i$ for all $i\in I$. Hence, $\im(\gamma)\supseteq\langle (x_i)_{i\in I}\rangle_R$. On the other hand, let $N$ be a submodule of $M$ that contains $\lbrace x_i\mid i\in I\rbrace$. Then $\sum_{i\in J} a_i x_i\in N$ for all finite subsets $J\subseteq I$, which implies $\im(\gamma)=\langle (x_i)_{i\in I}\rangle_R$ because $\im(\gamma)=\lbrace\sum_{i\in J} a_i x_i| J\subseteq I, J \text{ finite}\rbrace$.
\end{proof}

We note that $(x_i)_{i\in I}$ is a generating set of the $R$-module $M$ iff $\gamma$ is surjective. This together with \cref{cor.Homomorphism Theorem} implies that 
\begin{equation*}
M\cong\faktor{R^{(I)}}{\ker(\gamma)}.
\end{equation*}


\begin{defin}
Let $(x_i)_{i\in I}$ be an indexed family in an $R$-mo\-dule $M$ and $\gamma\colon R^{(I)}\to M$ the homomorphism introduced in \cref{lem.Generating Set Homomorphism}.
\begin{thmlist}
\item We say the family $(x_i)_{i\in I}$ is \emph{linearly independent over $R$} if $\gamma$ is injective.
\item The family $(x_i)_{i\in I}$ is called a \emph{basis of $M$ over $R$} if $\gamma$ is bijective.
\end{thmlist}
\end{defin}
The following Corollary gives alternative characterizations of linear independence. All statements are a rephrase of $\gamma$ being injective.
\begin{cor}\label{lm.Linear Independence}
Let $(x_i)_{i\in I}$ be an indexed family in an $R$-module $M$ then the following statements are equivalent.
\begin{thmlist}
\item $(x_i)_{i\in I}$ is linear independent over $R$.
\item If $\sum_{i\in I}a_i x_i=0$, where only a finite number of $a_i\in R$ is unequal to 0, then $a_i=0$ for all $i\in I$.
\item If 
\[\sum_{i\in I}a_i x_i=\sum_{i\in I}b_i x_i,\]
where only a finite number of $a_i,b_i\in R$ is unequal to 0, then $a_i=b_i$ for all $i\in I$.\label{lm.All Scalars are 0}
\item For every strict subset $J\subsetneq I$ the submodule $\langle (a_i)_{i\in J}\rangle_R$ is a strict subset of $\langle (a_i)_{i\in I}\rangle_R$.
\end{thmlist}
\end{cor}

Before we give examples of finitely generated modules and bases of modules we will briefly study homomorphisms of finitely generated modules.

\begin{thm}\label{theo.Homomorphisms of free modules}
Let $\varphi\colon M\to M'$ be a homomorphism of $R$-modules and let $(v_i)_{i\in I}$ be a generating set of $M$ then $\varphi$ is defined uniquely by the images $\varphi(v_i)$, $i\in I$.
\end{thm}
\begin{proof}
Given an arbitrary $x\in M$ we find a finite subset $J\subseteq I$ and elements $a_j\in R$ for all $j\in J$ such that $\sum_{j\in J}a_j v_j=x$. Considering $\varphi(x)$ we notice
\begin{equation*}
\varphi(x)=\varphi\left(\sum_{j\in J}a_j v_j\right)=\sum_{j\in J}\varphi(a_jv_j)=\sum_{j\in J}a_j\varphi(v_j).
\end{equation*}
\end{proof}

\begin{defin}
\begin{thmlist}
\item A family of module-homomorphisms
\[(\varphi_i\colon M_i\to M_{i+1})_{i\in I},\]
where $I\subseteq \Z$ is an interval, is called \emph{exact sequence} if $\im(\varphi_i)=\ker(\varphi_{i+1})$ for all $i\in I$ (excluding the maximum of $I$ if it exists).
\item A \emph{short exact sequence} is an exact sequence where $I=\lbrace0,1,2,3\rbrace$ and $M_0=M_4=\lbrace 0\rbrace$.
We will write
\begin{equation*}
0\longrightarrow M_1\overset{\varphi_1}{\longrightarrow} M_2\overset{\varphi_2}{\longrightarrow} M_3\longrightarrow 0
\end{equation*}
for such  a sequence.
\end{thmlist}
\end{defin}

Note that $\varphi_0$ and $\varphi_3$ in the second part of the definition above are uniquely defined by their domain or their image, respectively. Furthermore, $\varphi_1$ is injective and $\varphi_2$ surjective.

\begin{lem}\label{lem.Short Exact Sequence of Finitely Generated Modules}
Let
\begin{equation*}
0\longrightarrow M'\overset{\varphi_1}{\longrightarrow} M\overset{\varphi_2}{\longrightarrow} M''\longrightarrow 0
\end{equation*}
be a short exact sequence of $R$-modules, then
\begin{thmlist}
\item $M''$ is finitely generated if $M$ is finitely generated and
\item if $M'$ and $M''$ are finitely generated, then $M$ is finitely generated.\label{lem.M' M'' implies M}
\end{thmlist}
\end{lem}

\begin{proof}
\begin{plist}
\item Let $\seq{v}\in M$ be a generating set of $M$ over $R$ then \cref{theo.Homomorphisms of free modules} implies that every $x\in M''$ can be written as 
\[x=\sum_{i=1}^n a_i\varphi_2(v_i)\]
for some $a_i\in R$ because $\varphi_2$ is surjective. Therefore, $\varphi_2(v_1),\ldots\varphi_2(v_n)$ is a finite generating set of $M''$.
\item Let $\seq[r]{u}\in M'$ be a generating set of $M'$ and let $\seq[s]{w}\in M''$ be a generating set of $M''$.
We will prove that
\[\varphi_1(u_1),\ldots,\varphi_1(u_r),v_1,\ldots,v_s\in M\]
where $v_i\in\varphi_2^{-1}(w_i)$ for $i\in\lbrace 1,\ldots,s\rbrace$, is a generating set of $M$ over $R$.
Let therefore $x$ be an arbitrary element of $M$.
Since $\seq[s]{w}$ is a generating set of $M''$ we can write 
\begin{equation*}
\varphi_2(x)=\sum_{i=1}^s a_i w_i=\varphi_2\left(\vphantom{\sum_{i=1}^s}\right.
\underbrace{\sum_{i=1}^s a_i v_i}_{\in M}
\left. \vphantom{\sum_{i=1}^s}\right)
\end{equation*}
for some $\seq[s]{a}\in R$. Since $\nicefrac{M}{\ker(\varphi_2)}\cong M''$ (with respect to $\bar{\varphi}_2$ see \cref{theo.Universal Property} and \cref{cor.Homomorphism Theorem}) we find $x\in \sum_{i=1}^s a_i v_i+\ker(\varphi_2)$ and therefore $x=\sum_{i=1}^s a_i v_i+k$ where $k\in\ker(\varphi_2)$. Because $\im(\varphi_1)=\ker(\varphi_2)$ part (i) implies that $\varphi_1(u_1),\ldots,\varphi_1(u_r)$ is a generating set of $\ker(\varphi_2)\subseteq M$. Hence, $x=\sum_{i=1}^s a_i v_i+\sum_{j=1}^s b_j \varphi_1(u_j)$ for some $\seq[s]{b}\in R$. Which proves the claim.
\end{plist}
\end{proof}

One of the main results in linear algebra is that every vector space over a field has a basis over that field. \cref{exam.basis of G} shows that a general $R$-module does not have a basis.

\begin{exam}\label{exam.Bases of Modules}
\begin{exlist}
\item Let $R$ be a unital ring, then $\lbrace 1\rbrace$ is a basis of $R$, because $a=a1$ for all $a\in R$ and this decomposition is unique.
\item Let $V$ be an $F$-vector space. In \cref{exam.Vector Space as EndFV-module} we have seen, that $V$ is an $\End_F(V)$-module. If $\seq{b}\in V$ is a basis of $V$ over $F$, then $\lbrace b_1 \rbrace$ is a generating set of $V$ over $\End_F(V)$, because for every $v\in V$ we can define a linear mapping 
\begin{equation*}
\varphi_v\colon V\to V \text{ by } \varphi_v(b_i):= \begin{cases}v & \text{if } i=1\\ 0 & \text{if } i\neq 1\end{cases}.
\end{equation*}
Therefore the mapping $\gamma\colon \End_F(V)\to V$ defined by $\varphi\mapsto \varphi(b_1)$ is surjective and not injective if $n>1$.

Note that this implies that $V$ contains only the trivial submodules because for every $x\in V\setminus\lbrace 0\rbrace$ we can find a basis $\seq{\hat{b}}$ such that $x=\hat{b}_1$ and $\langle x\rangle_{\End_F(V)}=V$ is the smallest submodule that contains $x$. \label{exam. Generating set of V over EndFV}
\item Let $G\neq\lbrace 0\rbrace$ be a finite group viewed as a $\Z$-module, then $G$ does not have a basis over $\Z$. Let $n=|G|$. We assume that $G$ has a basis $B$. If $B$ is empty then $\langle B\rangle_{\Z} =\lbrace 0\rbrace$ which is a contradiction to $G\neq\lbrace 0\rbrace$. If, on the other hand, $B$ is non-empty with $x\in B$ we consider the $n+1$ elements $x,2x,\ldots,(n+1)x$. Since $n=| G|$ there exist $1\leq k<\ell\leq n+1$ such that $kx=\ell x$. Which is a contradiction to the assumption that $B$ is a basis of $G$ over $\Z$.\label{exam.basis of G}
\end{exlist}
\end{exam}

Finally, we will see that the reverse implication of \cref{lem.M' M'' implies M} does not hold in general, i.\,e., there exist finitely generated modules which contain non finitely generated submodules. The following example can be found in \cite[p.197]{jantzen2005algebra}.

\begin{exam}
Let $F$ be a field. We consider $R=F[X_1,X_2,\ldots]$ the polynomial ring with countable infinitely many variables $X_i$ over $K$. We consider $R$ as a module over itself and find that $\lbrace 1\rbrace$ is a generating set of $R$ over itself.

The set $\mathfrak{I}$ of all polynomials in $R$ with constant term $a_0=0$ is an ideal in $R$ and therefore a submodule. But there is no finite generating set of $\mathfrak{I}$. If we assume that $\lbrace\seq{p}\rbrace\subseteq \mathfrak{I}$ is a generating set of $\mathfrak{I}$ we note that only a finite number $m\in\N$ of the variables $(X_i)_{i\in\N}$ appear in $\lbrace\seq{p}\rbrace$. We denote these $m$ variables by $X_{i_1},\ldots,X_{i_m}$. Hence, there exists a natural number $k\in\N$ such that $X_k$ does not appear in $\lbrace\seq{p}\rbrace$ and so there do not exist $\seq{a}\in R$ such that $X_k=\sum_{i=1}^n a_ip_i$ because only polynomials containing at least one $X_{i_\ell}$ ($1\leq\ell\leq k$) can be represented by such a sum. It follows that$\langle\seq{p}\rangle_R\subsetneq \mathfrak{I}$.
\end{exam}