\begin{defin}
A group $G$ is called \emph{periodic,} if for each element $a\in G$ there exists a positive integer $n>0$, such that
\begin{equation*}
a^n=e.
\end{equation*}
\end{defin}
Clearly every finite group is periodic, since $\lvert G\rvert=n$ implies $a^n=e$ for each group element $a\in G$. However, note that the exponent $n$ in the definition of a periodic group my depend on the group element $a\in G$. For example consider the direct sum
\begin{equation*}
G=\bigoplus_{i=2}^\infty \Z_i,
\end{equation*}
where $\Z_i=\nicefrac{\Z}{i\Z}$ denotes the cyclic group of order $i\geq 2$. For each sequence $(a_i)_{i=2}^\infty$ contained in $G$ one finds finitely many indices $\seq[k]{i}$, such that $a_i=0$ if $i\not\in\lbrace\seq[k]{i}\rbrace$. Hence, the least common multiple $\lcm(\seq[k]{i})$ satisfies
\begin{equation*}
\lcm(\seq[k]{i})\cdot(a_i)=(\overline{0})_{i=1}^\infty.
\end{equation*}
Therefore, $G$ is periodic. Howbeit, for each integer $M\geq 2$ the group element $(\delta_{i,M+1})_{i=1}^\infty\in G$, where
\begin{equation*}
\delta_{i,M+1}=\begin{cases} \overline{1} &\text{if }i=M+1\\ \overline{0}&\text{else}\end{cases},
\end{equation*}
is of order $M+1$.

If there exists a positive integer $N$, such that
\begin{equation*}
a^N=e
\end{equation*}
for all elements $a$ of a group $G$, we say $G$ \emph{is of exponent $N$.}

In 1902 \textcite{burnside1902} published his paper \emph{On an unsettled question in the theory of discontinuous groups}, where he wrote:
\begin{quotation}
\itshape%
A still undecided problem in the theory of discontinuous groups is whether the order of a group may be not finite while the order of every operation it contains is finite.
\end{quotation}

\textcite{newman2004still} points out, that by `discontinuous groups' Burnside either meant specific subgroups of $\mathrm{GL}_n(\C)$ or assumed that the groups are finitely generated. However, in this bachelor thesis, we will only discuss the latter suggestion, so that Burnside's question is equivalent to:
\begin{quotation}
Is every finitely generated, periodic group finite?
\end{quotation}

The search for a proof of this hypothesis or a counterexample to it is commonly referred to as the \emph{unbounded Burnside problem.}\footnote{Some authors speak of the \emph{general Burnside Problem}.}

At first let us take a look at some special cases, where the answer to Burnside's question is affirmative. If the group $G$ is Abelian, then by the fundamental theorem of finitely generated Abelian groups the following holds
\begin{equation*}
G\cong\Z^n \oplus \Z_{q_1} \oplus \cdots \oplus \Z_{q_t},
\end{equation*}
where the rank $n$ is non-negative and the numbers $\seq[t]{q}$ are powers of (not necessarily distinct) prime numbers. Assume $n$ to be greater than zero, then $(1,0,\ldots,0)\in\Z^n \oplus \Z_{q_1} \oplus \cdots \oplus \Z_{q_t}$ is of infinite order. Thus, a finitely generated, periodic, Abelian group has finite order.

In his paper \textcite{burnside1902} introduces a special case of his initial question:
\begin{quotation}
\itshape%
Let $\seq[m]{A}$ be a set of independent operations finite in number, and suppose that they satisfy the system of relations given by
\begin{equation*}
S^n=1
\end{equation*}
where $n$ is a given integer, while $S$ represents in turn any and every operation which can be generated from the m given operations $A$.

Is the group thus defined one of finite order, and if so what is its order?
\end{quotation}
This special case is called the \emph{bounded Burnside problem}. It is rather easy to answer if the exponent $n$ equals 2 since such a group $G$ must be Abelian. To see this consider two arbitrary elements $a,b$ of a group $G$, which by assumption fulfil $(ab)^2=e$. Hence, $ab=(ab)^{-1}=ba$ and the order of $G$ is $2^m$. In his paper \textcite{burnside1902} proves that the answer is affirmative for the exponent $n=3$ and arbitrary $m$ as well as for the exponent $n=4$ and $m=2$ generators.

It was proven by Schur in 1911, that each finitely generated periodic subgroup of a linear group $\mathrm{GL}_n(\C)$ is finite. This result was generalized by Kaplansky in 1965 for arbitrary fields.

Around 1930 a third question---the \emph{restricted Burnside problem}---arose in this context:
\begin{quotation}
If you fix two integers $m,n>1$ are there up to isomorphy only finitely many finitely groups generated by $m$ elements of exponent $n$?
\end{quotation}

It took more than 30 years until significant progress on Burnside's problems was made. In 1964 Evgeny Golod and Igor Shafarevich were able to construct a counterexample to the unbounded Burnside problem. Pyotr Novikov and Sergei Adian proved the existence of infinite, finitely generated groups of odd exponent $n\geq 4381$. Thus, the answer to the bounded Burnside problem is negative as well. In 1975 Adian generalized the statement of the \emph{Novikov-Adian-theorem} on finitely generated groups of odd exponent $n\geq 665$.

The first Grigorchuk group, which proves to be a counterexample to the bounded Burnside problem as well, was introduced by Rostislav Grigorchuk in 1980.

\sloppypar
Alexander Olśhanski\u\i's proof of the Novikov-Adian-theorem, which was based on geometric arguments, was published in 1982. The techniques he introduced in his work proved to be very powerful in the whole field of group theory.

After more than 60 years of research the restricted Burnside problem was affirmatively answered by Efim Zelmanov in 1994. He was awarded the Fields-medallion for this work.

The broadest generalization of the Novikov-Adian-theorem is due to Lys\"{e}nok. He published his proof of the existence of infinite, finitely generated groups of general exponent $n\geq 8000$ in 1996.