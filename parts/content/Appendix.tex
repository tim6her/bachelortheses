% !TeX encoding = UTF-8
% !TeX spellcheck = en_GB
The program code for \textsc{Wolfram Mathematica$^\circledR$ 9} in \cref{lst.SubgroupSn} defines the procedure \verb+SubgroupSn+, which finds the subgroup \verb+SG+ of the symmetric group $\mathfrak{S}_n$ that is generated by a given list of generators \verb+GenIn+.

The permutations $\pi\in\mathfrak{S}_n$ are implemented as lists of length $n$, where the $i$-th entry contains the image $\pi(i)$. The function \verb+Comp+ returns the list corresponding to the composition of two permutations \verb+p1+ and \verb+p2+. It is defined on line 4. The idea of the procedure \verb+SubgroupSn+ is to compose the generators with the group elements from left and right until there are no new elements left.

To initialize this process we append the first generator to \verb+SG+ (see line 7). During each iteration of the \verb+While+-loop (line 11--28) we store the first element of \verb+Gen+ in \verb+g+ and drop it from \verb+Gen+. The permutation \verb+g+ will be composed with each group element from right (line 16) and from left (line 21). If the composition \verb+c+ is not already contained in temporary storage \verb+TempList+ of the subgroup, the permutation \verb+c+ is appended to the list of generators \verb+Gen+ and the temporary storage \verb+TempList+. The algorithm terminates, when there are no generators left.

\begin{lstlisting}[float, caption=Finding a subgroup of $\mathfrak S_n$,
                   label=lst.SubgroupSn,
                   language=Mathematica, 
                   numbers=left,
                   numberstyle=\tiny,
                   stepnumber=2,
                   numbersep=5pt]
SubgroupSn[GenIn_, n_] :=
 Module[{g, c, Gen, SG, TempList},
  
  Comp[p1_, p2_] := Table[p1[[p2[[i]]]], {i, 1, n}];
  
  Gen = GenIn;
  SG = {Gen[[1]]};
  
  TempList = SG;
  
  While[Length[Gen] > 0,
   g = Gen[[1]];
   Gen = Rest[Gen];
   
   For[i = 1, i <= Length[SG], i++,
    c = Comp[SG[[i]], g];
    If[Count[TempList, c] == 0,
     AppendTo[Gen, c];
     AppendTo[TempList, c];
     ];
    c = Comp[g, SG[[i]]];
    If[Count[TempList, c] == 0,
     AppendTo[Gen, c];
     AppendTo[TempList, c];
     ];
    ];
   SG = TempList
   ];
  Return[SG];
  ]
\end{lstlisting}

The program code in \cref{lst.pi 3(B)} calls the procedure \verb+SubgroupSn+ to compute the subgroup $\varpi_3(B)\subseteq\mathfrak{S}_8$. The images of the generators $b,u,v$ and $w$ under $\varpi_3$ are stored in \verb+pib+, \verb+piu+, \verb+piv+ and \verb+piw+. The integers contained in these lists correspond to the numbers below the diagrams in \cref{g:pi b,g:pi u,g:pi v,g:pi w}.
\begin{lstlisting}[float, caption=Computing $\varpi_3(B)\subseteq\mathfrak{S}_8$,
                   label=lst.pi 3(B),
                   language=Mathematica, 
                   numbers=left,
                   numberstyle=\tiny,
                   stepnumber=2,
                   numbersep=5pt]
pib = {3, 4, 1, 2, 6, 5, 7, 8};
piu = {2, 1, 3, 4, 7, 8, 5, 6};
piv = {2, 1, 4, 3, 5, 6, 7, 8};
piw = {1, 2, 3, 4, 6, 5, 8, 7};
Gen = {pib, piu, piv, piw};

SubgroupSn[Gen, 8]
\end{lstlisting}
We find that the image of $B$ under $\varpi_3$ is the following subgroup
\begin{align*}
\varpi_3(B)=\lbrace	&\epsilon,(1\, 3)(2\, 4)(5\, 6),   (1\, 4)(2\, 3)(5\, 6), (1\, 2)(3\, 4),\\
                    & (1\, 4\, 2\, 3)(5\, 7\, 6\, 8),	(1\, 3\, 2\, 4)(5\, 8\, 6\, 7),   (1\, 2)(5\, 7)(6\, 8),\\
                    & (1\, 3\, 2\, 4)(5\, 7\, 6\, 8),	(1\, 4\, 2\, 3)(5\, 8\, 6\, 7),   (3\, 4)(5\, 7)(6\, 8),\\
                    & (1\, 3)(2\, 4)(7\, 8),   (5\, 6)(7\, 8), 
                    (1\, 4)(2\, 3)(7\, 8),\\
                    & (1\, 2)(3\, 4)(5\, 6)(7\, 8),   (1\, 2)(5\, 8)(6\, 7), (3\, 4)(5\, 8)(6\, 7)\rbrace.
\end{align*}
The image of $t=(ab)^2$ under $\psi_1$ is $(ca,ac)$. Therefore, the images of the vertices on level 3 under $t$ compute as follows
\begin{align*}
t(0,0,0)&=(0,ca(0,0))=(0,c(1,0))=(0,1,b(0))=(0,1,0),\\
t(0,0,1)&=(0,ca(0,1))=(0,c(1,0))=(0,1,b(1))=(0,1,1),\\
t(0,1,0)&=(0,ca(1,0))=(0,c(0,0))=(0,1,b(0))=(0,0,0),\\
t(0,1,1)&=(0,ca(1,1))=(0,c(0,1))=(0,0,b(1))=(0,0,1),\\[2em]
t(1,0,0)&=(1,ac(0,0))=(1,a(0,a(0)))=(1,a(0,1)=\\
        &=(1,1,1),\\
t(1,0,1)&=(1,ac(0,1))=(1,a(0,a(1)))=(1,a(0,0)=\\
        &=(1,1,0),\\
t(1,1,0)&=(1,ac(1,0))=(1,a(1,d(0)))=(1,a(1,0)=\\
        &=(1,0,0)\\
\intertext{ and}
t(1,1,1)&=(1,ac(1,1))=(1,a(1,d(0)))=(1,a(1,0)=\\
        &=(1,0,1).
\end{align*}
As a result, automorphism $t$ is mapped to the permutation 
\begin{equation*}
(1\,4\,2\,3)(5\,7\,6\,8)\in\mathfrak{S}_8,
\end{equation*}
so that we may apply the algorithm described above to find $\varpi_3(K)$.
\begin{lstlisting}[language=Mathematica, numbers=left, numberstyle=\tiny, stepnumber=2, numbersep=5pt]
pit = {4, 3, 1, 2, 7, 8, 6, 5};
piu = {2, 1, 4, 3, 5, 6, 7, 8};
piw = {1, 2, 3, 4, 6, 5, 8, 7};
Gen = {pit, piu, piw};

SubgroupSn[Gen, 8];
\end{lstlisting}
We conclude, that the image of $K$ under $\varpi_3$ is
\begin{align*}
\varpi_3(K)=\lbrace	&\epsilon,(1\, 3)(2\, 4)(5\, 6), (1\, 4)(2\, 3)(5\, 6), (1\, 2)(3\, 4),   \\
									&(1\, 4\, 2\, 3)(5\, 7\, 6\, 8), (1\, 3\, 2\, 4)(5\, 8\, 6\, 7), \\
									&(1\, 2)(5\, 7)(6\, 8),   (1\, 3\, 2\, 4)(5\, 7\, 6\, 8)\rbrace.
\end{align*}