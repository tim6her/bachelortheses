In this final section we will discuss the growth of the first Grigorchuk group, i.e. we will study the relation of the length of a word $w$ in the alphabet $\lbrace a,b,c,d\rbrace$ and the order of the automorphism represented by $w$. More formally, let
\begin{equation*}
O_k:=\lbrace \ord(\gamma)\vert\gamma\in\Gamma,\ell(\gamma)\leq k\rbrace
\end{equation*}
denote the set of the orders of all automorphisms which are at most of length $k$. Since there exist only a finite number of  words in the alphabet $\lbrace a,b,c,d\rbrace$ that are shorter than $k$, the set $O_k$ is finite. By \cref{thm:Gamma is a 2 group} it contains a  finite maximum.  We define the \emph{growth function} $o\colon \N_0\to\N_0$ by 
\begin{equation*}
o(k):=\max(O_k).
\end{equation*}
Since $O_k\subseteq O_{k+1}$ the function $o$ is monotonically increasing.\\
The main result of this section is that the first Grigorchuk group is not a counterexample to the bounded Burnside's problem, i.e. for each non-negative integer $M$, there exists an automorphism $\gamma\in\Gamma$, such that the order of $\gamma$ exceeds $M$.\\
Under close inspection of the proof of \cref{thm:Gamma is a 2 group} one can derive an upper bound for the growth of $\Gamma$.
\begin{cor}\label{thm:Upper bound for the growth of Gamma}
The growth of $\Gamma$ is bound by an exponential function. More precisely
\begin{equation}\label{eq:Upper bound for the growth of Gamma}
o(k)\leq 2^{n(k)},
\end{equation}
where $n$ denotes the affine function $n(k):=\frac{1}{2}k+3$ for all $k\in\N_0$.
\end{cor}

\begin{proof}
\hyperlink{thm:Base Case}{Analogously to the proof of \cref*{thm:Gamma is a 2 group}} we conclude
\begin{itemize}
\item $o(0)=0$ since $\ell(\gamma)=0$ iff $\gamma=\id$,
\item $o(1)=2$ by \cref{thm:Gammas Generators are Self-Inverse} and
\item $o(2)=16$ by \cref{thm:Dihedral groups in Gamma}.
\end{itemize}
We proceed by induction on the maximal length $k$. To this end, let $k>2$ and assume the claim to be proven for all orders of automorphisms of length shorter than $k$. Given an arbitrary automorphism $\gamma\in\Gamma$, satisfying $\ell(\gamma)=k$, we will prove
\begin{equation*}
\ord(\gamma)\leq 2^{n(k)}.
\end{equation*}
This suffices because
\begin{equation*}
O_k=O_{k-1}\cup \lbrace\ord(\gamma)\vert \gamma\in\Gamma,\ell(\gamma)=k\rbrace
\end{equation*} 
and the right hand side of Inequality~(\ref{eq:Upper bound for the growth of Gamma}) is monotonically increasing.\\
Firstly, suppose $k=2m+1$ is odd. If this is the case, \hyperlink{thm:k is odd}{it was shown} that $\gamma$ is conjugate to an automorphism $\eta$ with $\ell(\eta)\leq k-1$. It follows, $o(2m+1)\leq o(2m)$ and by the monotonicity of $o$
\begin{equation}\label{eq:o(2m+1)=o(2m)}
o(2m+1)=o(2m).
\end{equation}
Thus, the statement is proven for odd length $k$, since the right hand side of \cref{eq:Upper bound for the growth of Gamma} is monotonically increasing.\\
Secondly, assume four divides $k$, i.e. there exists an integer $m$, such that $k=4m$. In the proof of \cref{thm:Gamma is a 2 group} we have seen, that the automorphisms $\gamma_1$ and $\gamma_2$ satisfying $(\gamma_1,\gamma_2)=\psi(\gamma)$, are at most of length $\frac{k}{2}=2m$. The injectivity of $\psi$ implies, that
\begin{equation*}
\ord(\gamma)=\max(\ord(\gamma_1),\ord(\gamma_2))
\end{equation*}
since the orders of $\gamma_1$ and $\gamma_2$ are powers of 2 by \cref{thm:Gamma is a 2 group}. We conclude
\begin{equation}
o(4m)=o(2m)\label{eq:o(4m)=o(2m)}
\end{equation}
for the growth function $o$, which proves the claim, because the right hand side of Inequality \ref{eq:Upper bound for the growth of Gamma} is monotonically increasing.\\
Finally, assume $k=4m-2$ for some integer $m\geq 2$. Analogously to the proof of  \cref{thm:Gamma is a 2 group} we may suppose the word $w$ of length $k$ representing $\gamma$ to start with automorphism $a$, i.e.
\begin{equation*}
w=au_1 a u_2 a\ldots u_l.
\end{equation*}
Again we finish the proof by distinguishing three cases for the pair of automorphisms $(\alpha,\beta):=\psi(\gamma^2)$.
\begin{clist}
\item\label{p:Upper Bound for the Growth of Gamma 1} If there exists an index $1\leq j\leq l$, such that $u_j=d$, we have seen, that $\alpha$ and $\beta$ are at most of length $k-1$. The injectivity of $\psi$ implies, that $\ord(\gamma^2)=\max(\ord(\alpha),\ord(\beta))$, which is equivalent to
\begin{equation}\label{eq:Upper Bound for the Growth of Gamma 1}
\ord(\gamma)=2\max(\ord(\alpha), \ord(\beta)).
\end{equation}
\item\label{p:Upper Bound for the Growth of Gamma 2} If there exists an index $1\leq j\leq l$, such that $u_j=c$, we have already shown, that one of the following is the case:
\begin{itemize}
\item Both automorphisms $\alpha$ and $\beta$ are shorter than $k$.
\item Both $\alpha$ and $\beta$ are automorphisms of length $k$ containing the automorphism $d$
\end{itemize}
If the first is the case, the induction hypothesis proves the claim. If the latter is true, case \ref{p:Upper Bound for the Growth of Gamma 1} applies. Thus,
\begin{equation}\label{eq:Upper Bound for the Growth of Gamma 2}
\ord(\gamma)=4\max(\ord(\alpha), \ord(\beta)).
\end{equation}
\item Finally, if neither \ref{p:Upper Bound for the Growth of Gamma 1} nor \ref{p:Upper Bound for the Growth of Gamma 2} are the case, then $\gamma\in\langle a,b\rangle\cong D_{16}$. It follows,
\begin{equation}\label{eq:Upper Bound for the Growth of Gamma 3}
\ord(\gamma)\leq 16
\end{equation}
by \cref{thm:D16}.
\end{clist}
\cref{eq:Upper Bound for the Growth of Gamma 1,eq:Upper Bound for the Growth of Gamma 2,eq:Upper Bound for the Growth of Gamma 3} implicate
\begin{equation*}
\ord(\gamma)\leq4\max(\ord(\alpha), \ord(\beta), 4).
\end{equation*}
As for the growth function, it follows
\begin{align*}
o(4m-2)	&\leq 4\max(o(4m-3),4)\overset{\text{\cref{eq:o(2m+1)=o(2m)}}}{=}\\
		&= 4\max(o(4m-4),4)\overset{\text{\cref{eq:o(4m)=o(2m)}}}{=}\\
		&=4\max(o(2m-2),4)\overset{\text{IH}}{\leq}\\
		&\leq \max\left(2^{n(2m-2)+2}, 2^4\right)=\\
		&=\max\left(2^{m+4},2^4\right)\leq 2^{\frac{4m-2}{2}+3}.
\end{align*}
The inequality marked with ``IH" holds by induction hypothesis while the last inequality holds by the assumption $m\geq 2$.
\end{proof}

Let as above $n\colon\N_0\to\N_0$ denote a function that serves as an upper bound for the logarithmic growth of $\Gamma$, i.e.
\begin{equation*}
\log_2(o(k))\leq n(k)
\end{equation*}
for all integers $k\geq 0$. In the proof of the corollary we have derived sufficient recursive conditions for $n$, namely the initial conditions
\begin{equation*}
n(0)\geq 0,\quad n(1)\geq 1\quad \text{and}\quad n(2)\geq 4\\
\end{equation*}
as well as
\begin{align*}
n(k)\geq n(k-1)&\quad\text{if } k>2,\\
n(4m)\geq n(2m)&\quad\text{if } m\geq 1\text{ and}\\
n(4m-2)\geq n(2m-2)+2&\quad\text{if } m\geq 1.
\end{align*}
\cref{thm:Upper bound for the growth of Gamma} shows, that $n_1(k)=\frac{k}{2}+3$ serves as an upper bound for the logarithmic growth of $\Gamma$. However, note that this is far from optimal. \textcite{2006math} recently published a paper stating, that one may choose $n_2(k):=\log_2(4k^2+1)$ as such a function. Their result, which was obtained by studying $\Gamma$ as a group defined by finite state automata, implies a polynomial growth of $\Gamma$, i.e.
\begin{equation*}
o(k)\leq 4k^2+1.
\end{equation*}
In \cref{g:Logarithmic growth} the upper bounds $n_1$ (dashed line) and $n_2$ (continuous line) as well as the minimal solution to the recursion derived from the proof of \cref{thm:Upper bound for the growth of Gamma} are plotted. 

\begin{figure}
\begin{centering}
\includegraphics[width=\textwidth]{./res/LogarithmicGrowth}
%\mbox{
%\beginpgfgraphicnamed{LogarithmicGrowth}
%\begin{tikzpicture}[line cap=round,line join=round,>=triangle 45,x=0.2cm,y=0.4cm, every node=[draw=none]]
%\draw[->,color=black] (-0.5,0) -- (50.5,0);
%\foreach \x in {5,10,15,20,25,30,35,40,45}
%\draw[shift={(\x,0)},color=black] (0pt,2pt) -- (0pt,-2pt) node[below, draw=none] {\footnotesize $\x$};
%\draw[->,color=black] (0,-0.5) -- (0,15.5);
%\foreach \y in {2,4,6,8,10,12,14}
%\draw[shift={(0,\y)},color=black] (2pt,0pt) -- (-2pt,0pt) node[left, draw=none] {\footnotesize $\y$};
%\clip(-0.5,-0.5) rectangle (50.5,15.5);
%\draw[thick,dashed,smooth,samples=100,domain=0:50.5] plot(\x,{\x/2+3});
%\draw[thick,smooth,samples=100,domain=0:50.5] plot(\x,{ln(4*\x^2+1)/ln(2)});
%
%\begin{scriptsize}
%\fill  (0,0) circle (1.5pt);
%\fill  (1,1) circle (1.5pt);
%\fill  (2,4) circle (1.5pt);
%\fill  (3,4) circle (1.5pt);
%\fill  (4,4) circle (1.5pt);
%\fill  (5,4) circle (1.5pt);
%\fill  (6,4) circle (1.5pt);
%\fill  (7,4) circle (1.5pt);
%\fill  (8,4) circle (1.5pt);
%\fill  (9,6) circle (1.5pt);
%\fill  (10,6) circle (1.5pt);
%\fill  (11,6) circle (1.5pt);
%\fill  (12,6) circle (1.5pt);
%\fill  (13,6) circle (1.5pt);
%\fill  (14,6) circle (1.5pt);
%\fill  (15,6) circle (1.5pt);
%\fill  (16,6) circle (1.5pt);
%\fill  (17,6) circle (1.5pt);
%\fill  (18,6) circle (1.5pt);
%\fill  (19,6) circle (1.5pt);
%\fill  (20,6) circle (1.5pt);
%\fill  (21,8) circle (1.5pt);
%\fill  (22,8) circle (1.5pt);
%\fill  (23,8) circle (1.5pt);
%\fill  (24,8) circle (1.5pt);
%\fill  (25,8) circle (1.5pt);
%\fill  (26,8) circle (1.5pt);
%\fill  (27,8) circle (1.5pt);
%\fill  (28,8) circle (1.5pt);
%\fill  (29,8) circle (1.5pt);
%\fill  (30,8) circle (1.5pt);
%\fill  (31,8) circle (1.5pt);
%\fill  (32,8) circle (1.5pt);
%\fill  (33,8) circle (1.5pt);
%\fill  (34,8) circle (1.5pt);
%\fill  (35,8) circle (1.5pt);
%\fill  (36,8) circle (1.5pt);
%\fill  (37,8) circle (1.5pt);
%\fill  (38,8) circle (1.5pt);
%\fill  (39,8) circle (1.5pt);
%\fill  (40,8) circle (1.5pt);
%\fill  (41,8) circle (1.5pt);
%\fill  (42,8) circle (1.5pt);
%\fill  (43,8) circle (1.5pt);
%\fill  (44,8) circle (1.5pt);
%\fill  (45,10) circle (1.5pt);
%\fill  (46,10) circle (1.5pt);
%\fill  (47,10) circle (1.5pt);
%\fill  (48,10) circle (1.5pt);
%\fill  (49,10) circle (1.5pt);
%\fill  (50,10) circle (1.5pt);
%\end{scriptsize}
%\end{tikzpicture}
%\endpgfgraphicnamed
%}
\caption{Upper bounds for the logarithmic growth of $\Gamma$ as in \cref{thm:Upper bound for the growth of Gamma}~\mbox{(-\hspace{2pt}-)}, as stated by Bartholdi and \v{S}uni\'{c}~(\textbf{---}) as well as the recursion of \cref{thm:Upper bound for the growth of Gamma}~($\bullet $)}
\label{g:Logarithmic growth}
\end{centering}
\end{figure}

Recall the mapping $\psi_\Gamma$ from \cref{thm:psi Gamma}. In the following we will generalize its definition to arbitrary stabilizer groups $\St[\Gamma]{k}$. For this purpose the bijection $\beta_k\colon L(k)\to\lbrace 1,2,\ldots 2^k\rbrace$ defined by
\begin{equation*}
\beta_k\left((b_i)_{i=1}^k\right)=1+\sum_{i=1}^{k} b_{i}2^{k-i}
\end{equation*}
will be useful. We have already encountered $\beta_k$ during the definition of $\St{k}$ on page \pageref{def:Stabilizer Group}.
\begin{defin}
Let $k\geq 1$ be an integer. We define $\psi_k\colon\St[\Gamma]{k}\to\Gamma^{2^k}$ by
\begin{equation*}
\gamma\mapsto\left(\gamma\big\vert_{\ST{\beta_k^{-1}(1)}},\gamma\big\vert_{\ST{\beta_k^{-1}(2)}},\ldots,\gamma\big\vert_{\ST{\beta_k^{-1}(2^k)}}\right).
\end{equation*}
\end{defin}
In the following we prove that $\psi_k$ is a well-defined monomorphism of groups.
\begin{thm}\label{thm:psi k}
The mapping $\psi_k\colon\St[\Gamma]{k}\to\Gamma^{2^k}$ is a monomorphism of groups for $k\geq1$, which satisfies the recursion
\begin{align}
\psi_1=&\psi_\Gamma\text{ and}\label{eq:psi1=psiGamma}\\
\psi_k(\gamma)=&\left((\pi_1\circ\psi_1)(\gamma_1),(\pi_2\circ\psi_1)(\gamma_1),\ldots\right.\\
	&\left.\ldots,(\pi_1\circ\psi_1)(\gamma_{2^{k-1}}),(\pi_2\circ\psi_1)(\gamma_{2^{k-1}})\right),\nonumber
\end{align}
when $\psi_{k-1}(\gamma)=(\gamma_1,\gamma_2,\ldots,\gamma_{2^{k-1}})$ and $\pi_i\colon\Gamma^2\to\Gamma$ denotes the projection defined by $(\alpha_1,\alpha_2)\mapsto\alpha_i$ for $i\in\lbrace 1,2\rbrace$.
\end{thm}
To increase readability we will denote the $2^k$-tupel
\begin{align*}
&\left((\pi_1\circ\psi_1)(\gamma_1),(\pi_2\circ\psi_1)(\gamma_1),\ldots\right.\\
&\left.\ldots,(\pi_1\circ\psi_1)(\gamma_{2^{k-1}}),(\pi_2\circ\psi_1)(\gamma_{2^{k-1}})\right)\in\Gamma^{2^k}
\end{align*}
by 
\begin{equation}\label{eq:Notation for projections}
\left(\psi_1(\gamma_1),\ldots,\psi_1(\gamma_{2^{k-1}})\right)_k
\end{equation}
if $(\gamma_1,\ldots,\gamma_{2^{k-1}})\in\St[\Gamma]{1}^{2^{k-1}}$. In this notation the sub-index $k$ shall remind the reader that the expression of Line (\ref{eq:Notation for projections}) represents a $2^k$-tupel. For example
\begin{equation*}
(\id,\id,a,c)=((\id,\id),(a,c))_2=(\psi_1(\id),\psi_1(b))_2
\end{equation*}
and $(\id,b)=\psi_1(d)$. Thus, \cref{thm:psi k} implies 
\begin{equation*}
\psi_2(d)=(\id,\id,a,c).
\end{equation*}
\cref{g:psi2(d)} illustrates the idea of the theorem. 
\begin{figure}
\begin{center}
\includegraphics[width=\textwidth]{./res/psi2}
%\mbox{
%\beginpgfgraphicnamed{psi2}
%\begin{tikzpicture}[level distance=7.5mm, baseline=(b.base)]
%\node (b) {$\emptyset$}
%  child foreach \i in {0,1} {node {\i}
% {child foreach \j in {0,1} {node {\j}
% {child foreach \k in {0,1} {node {\k}
% {child foreach \l in {0,1} {node {\l}
% {child foreach \m in {0,1} {coordinate edge from parent[dashed]}}}}}}}}};
%
%\begin{scope}[node distance=7mm, font=normall]
%\node[draw=none] [above of=b] (ST) {$d$};
%
%\node[draw=none] [above of=b-1] (ST0) {$\id$};
%\node[draw=none] [above of=b-2] (ST1) {$b$};
%\end{scope}
%
%\begin{scope}[bgn]
%\node[draw=none] [above of=b-1-1] (ST00) {$\id$};
%\node[draw=none] [above of=b-1-2] (ST01) {$\id$};
%\node[draw=none] [above of=b-2-1] (ST10) {$a$};
%\node[draw=none] [above of=b-2-2] (ST11) {$c$};
%\end{scope}
%
%\begin{scope}[node distance=3mm,font=\scriptsize, inner sep=1pt]
%\tikzstyle{every node}=[draw=none]
%\node  [below of=ST] {$\psi_1(d)=(\id,b)$};
%
%\node  [below of=ST0] {$\psi_1(\id)=(\id,\id)$};
%\node  [below of=ST1] {$\psi_1(b)=(a,c)$};
%\end{scope}
%
%\begin{pgfonlayer}{background}
%
%\filldraw[st] 	(ST0.north)  -- (ST01.north) -- (b-1-2-2.east) -- (b-1-2-2-2.east)-- (b-1-2-2-2-2) --
%				(b-1-1-1-1-1) -- (b-1-1-1-1.west) -- (b-1-1-1.west) -- (ST00.north) -- cycle;
%\filldraw[st] 	(ST1.north)  -- (ST11.north) -- (b-2-2-2.east) -- (b-2-2-2-2.east) -- (b-2-2-2-2-2) --
%				(b-2-1-1-1-1) -- (b-2-1-1-1.west) -- (b-2-1-1.west) -- (ST10.north) -- cycle;
%
%\filldraw[st1] 	(ST00.north)  -- (b-1-1-2.east) -- (b-1-1-2-2.east) -- (b-1-1-2-2-2.east) --
%				(b-1-1-1-1-1) -- (b-1-1-1-1.west) -- (b-1-1-1.west) -- cycle;
%\filldraw[st1] 	(ST01.north) -- (b-1-2-2.east) -- (b-1-2-2-2.east)-- (b-1-2-2-2-2) --
%				(b-1-2-1-1-1) -- (b-1-2-1-1.west) -- (b-1-2-1.west) -- cycle;
%\filldraw[st1] 	(ST10.north)  -- (b-2-1-1.west) -- (b-2-1-1-1.west) -- (b-2-1-1-1-1) -- 
%				(b-2-1-2-2-2) -- (b-2-1-2-2.east) -- (b-2-1-2.east) -- cycle;
%\filldraw[st1] 	(ST11.north) -- (b-2-2-1.west) -- (b-2-2-1-1.west) -- (b-2-2-1-1-1) -- 
%				(b-2-2-2-2-2) -- (b-2-2-2-2.east) -- (b-2-2-2.east) -- cycle;
%\end{pgfonlayer}
%\end{tikzpicture}
%\endpgfgraphicnamed
%}
\end{center}
\caption{$\psi_2(d)=(\id,\id,a,c)$}
\label{g:psi2(d)}
\end{figure}

\begin{proof}
\cref{eq:psi1=psiGamma} holds by definition. Let $\gamma\in\St[\Gamma]{k}$ be an arbitrary automorphism, then 
\begin{equation*}
\psi_{k-1}=(\gamma_1,\gamma_2,\ldots,\gamma_{2^{k-1}})\in\St[\Gamma]{k-1}^{2^{k-1}}.
\end{equation*}
If this were not the case, there would exist an index $1\leq j\leq 2^{k-1}$, such that $\gamma_j\not\in\St[\Gamma]{1}$. Hence, 
\begin{equation*}
\gamma(\beta_k^{-1}(j),0)=(\beta_k^{-1}(j),\gamma_j(0))=(\beta_k^{-1}(j),1)\neq(\beta_k^{-1}(j),0),
\end{equation*}
which contradicts the premises $\gamma\in\St[\Gamma]{k}$. Thus, it remains to show
\begin{equation*}
\psi_k(\gamma)=(\psi_1(\gamma_1),\ldots,\psi_1(\gamma_{2^{k-1}}))_k
\end{equation*}
if $\psi_{k-1}(\gamma)=\seq[2^{k-1}]{\gamma})$ and $k>1$. Assume indirectly there exists an automorphism $\gamma\in\St[\Gamma]{k}$ such that
\begin{equation*}
\eta_{2i+j-1}\neq\pi_j(\psi_1(\gamma_i)),
\end{equation*}
while $\psi_k(\gamma)=(\seq[2^k]{\eta})$ and $\psi_{k-1}(\gamma)=(\seq[2^{k-1}]{\gamma})$. Let $(\seq[k]{b})=\beta^{-1}(i)$ then by assumption there exists a vertex $v=(\seq[k]{b},\seq[l]{c})\in\ST{\beta_k^{-1}(i)}$, such that
\begin{align*}
\gamma(v)	&=		((\seq[k-1]{b},b_k),\eta_{2i+j-1}(\seq[l]{c}))\neq\\
			&\neq	((\seq[k-1]{b},b_k),(\pi_j(\psi_1(\gamma_i)))(\seq[l]{c}))=\\
			&=		((\seq[k-1]{b}),\gamma_i(b_k,\seq[l]{c}))=\\
			&=		\gamma(v).
\end{align*}
Thus, the assumption leads to a contradiction.\\
As a consequence, 
\begin{align*}
\psi_k(\gamma)=&\left((\pi_1\circ\psi_1)(\gamma_1),(\pi_2\circ\psi_1)(\gamma_1),\ldots\right.\\
	&\left.\ldots,(\pi_1\circ\psi_1)(\gamma_{2^{k-1}}),(\pi_2\circ\psi_1)(\gamma_{2^{k-1}})\right)
\end{align*}
is a well defined homomorphism. Since each pair of homomorphisms 
\begin{equation*}
\left((\pi_1\circ\psi_1)(\gamma_i),(\pi_2\circ\psi_1)(\gamma_i)\right)\colon\St[\Gamma]{1}\to\Gamma\times\Gamma
\end{equation*}
is injective, $\psi_k$ is injective as well.
\end{proof}
\begin{rem}
In the proof of \cref{thm:psi k} we have shown the following very useful identity
\begin{equation*}
\St[\Gamma]{k+1}=\psi_{k}^{-1}\left(\St[\Gamma]{1}^{2^k}\right).
\end{equation*}
\end{rem}
Remember the natural projection $p_k\colon\AutT\to\mathfrak{S}_{2^k}$ we encountered in the definition of $\St{k}=\ker(p_k)$. We consider the restriction of $p_k$
\begin{equation*}
\varpi_k\colon\Gamma\to\mathfrak{S}_{2^k},
\end{equation*}
defined by $\varpi_k(\gamma)=p_k(\gamma)$. Clearly, the identity $\ker(\varphi_k)=\St[\Gamma]{k}$ is satisfied. The subgroup $\im(\varpi_3)$ of $\mathfrak{S}_8$ is isomorphic to $\nicefrac{\Gamma}{\St[\Gamma]{3}}$. Next we shall compute its order.
\begin{pro}\label{thm:pi Gamma}
The group $\im(\varpi_3)\subseteq\mathfrak{S}_8$ is of order 128.
\end{pro}
\begin{proof}
Two automorphisms $\varphi,\eta\in\Gamma$ are in the same equivalence class of $\nicefrac{\Gamma}{\St[\Gamma]{3}}$ if and only if $\varphi\eta^{-1}\in\St[\Gamma]{3}$. Hence, their restrictions to the vertices on the first 3 levels must coincide. We can therefore identify $\nicefrac{\Gamma}{\St[\Gamma]{3}}$ with the automorphism group of the finite binary tree $\T_3$ of 15 vertices. The graph  $\T_3$ is visualized in the diagram below.

\begin{center}
\includegraphics[width=\textwidth]{./res/T32}
%\mbox{
%\beginpgfgraphicnamed{T32}
%\begin{tikzpicture}[level distance=7.5mm, baseline=(b.base)]
%\node (b) {$\emptyset$}
%  child foreach \i in {0,1} {node {\i}
% {child foreach \j in {0,1} {node {\j}
% {child foreach \k in {0,1} {node {\k}
% }}}}};
%
%\begin{scope}[bgn]
%\tikzstyle{every node}=[rectangle,draw=none,font=\small]
%\path[hlarrow] (b-1) .. controls +(3,-0.5) .. (b-2) node [draw=none, midway, above] {$a$};
%
%\path[hlarrow] (b-1-1) .. controls +(1.5,-0.5) .. (b-1-2) node [draw=none, midway, above] {$c$};
%\path[hlarrow] (b-2-1) .. controls +(1.5,-0.5) .. (b-2-2) node [draw=none, midway, above] {$aca$};
%
%\path[hlarrow] (b-1-1-1) .. controls +(0.75,-0.5) .. (b-1-1-2) node [draw=none, midway, below] {$ada$};
%\path[hlarrow] (b-1-2-1) .. controls +(0.75,-0.5) .. (b-1-2-2) node [draw=none, midway, below] {$cadac$};
%\path[hlarrow] (b-2-1-1) .. controls +(0.75,-0.5) .. (b-2-1-2) node [draw=none, midway, below] {$d$};
%\path[hlarrow] (b-2-2-1) .. controls +(0.75,-0.5) .. (b-2-2-2) node [midway, below] {$acadaca$};
%\end{scope}
%\end{tikzpicture}
%\endpgfgraphicnamed
%}
\end{center}

Since there are 7 pairs of subtrees (marked by a thick double-tipped arrow) that can switch there position independently, the automorphism group of $\T_3$ is of order $2^7=128$. Thus, we need to prove, that $\Gamma$ contains 7 automorphisms, which switch one of these pairs and leave the others fixed. In the diagram above those automorphisms can be found next to the pair they interchange.\\
As for automorphism $a$, this behaviour is clear from its definition. The action of $c$ on the first 3 levels was computed in the example following the definition of $a,b,c$ and $d$. As a consequence of \cref{thm:Conjugation with a inverts the performing of b c d}, automorphism $aca$ interchanges $\ST{(1,0)}$ and $\ST{(1,1)}$. In an example preceding the proof of \cref{thm:psi k} we have shown $\psi_2(d)=(\id,\id,a,b)$, which proves the claim for automorphism $d$ since $b\in\St[\Gamma]{1}$. Again \cref{thm:Conjugation with a inverts the performing of b c d} implies the statement for $ada$. Considering $cadac$ we find,
\begin{equation*}
\psi_1(cadac)=\psi_1(c)\psi_1(ada)\psi(c)=(aba,\id)
\end{equation*}
and by \cref{thm:psi k}
\begin{equation*}
\psi_2(cadac)=(\psi_1(aba),\psi_1(\id))_2=(c,a,\id,\id).
\end{equation*}
Applying \cref{thm:Conjugation with a inverts the performing of b c d} to $acadaca$ proves the claim.
\end{proof}
%\begin{lem}
%The composition of $\psi_2\colon\St[\Gamma]{2}\to\Gamma^4$ with any projection $\pi_i\colon\Gamma^4\to\Gamma$, defined by $(\gamma_1,\gamma_2,\gamma_3,\gamma_4)\mapsto\gamma_i$ for $1\leq i\leq 4$, is onto.
%\end{lem}
%\begin{proof}
%The formulas
%\begin{align*}
%\psi_1(cadab)	&=	\psi_1(c)\psi_1(ada)\psi_1(b)=(aba,dc)=(aba,b),\\
%\psi_1(acadaba)	&=	\psi_1(aca)\psi_1(d)\psi_1(aba)=(dc,aba)=(b,aba) \text{ and}\\
%\psi_1(acacacac)	&=	\psi_1(aca)\psi_1(c)\psi_1(aca)\psi_1(c)=(dada,adad)
%\end{align*} 
%imply by \cref{thm:psi k}
%\begin{align*}
%\psi_2(cadab)	&=	(\psi_1(aba),\psi_1(b))_2=(c,a,a,c),\\
%\psi_2(acadaba)	&=	(\psi_1(b),\psi_1(aba))_2=(a,c,c,a) \text{ and}\\
%\psi_2(acacacac)	&=	(\psi_1(dada),\psi_1(adad))_2=(\psi_1(d)\psi_1(ada),\psi_1(ada)\psi_1(d))_2=\\
%				&=	(b,b,b,b).
%\end{align*}
%The claim follows from \cref{thm:Three Generators Suffice}.
%\end{proof}
Let $B$ denote the \emph{normal} subgroup of $\Gamma$ generated by $b$. This is the intersection
\begin{equation*}
B=\bigcap_{\substack{N\unlhd\Gamma\\b\in N}}N
\end{equation*}
of all normal subgroups of $\Gamma$ containing $b$.
\begin{lem}\label{thm:B}
The group $B$ is of index $8$ in $\Gamma$ and is generated by the four elements
\begin{equation*}
b,\quad u=(bada)^2,\quad v=aba \text{ and } w=(abad)^2.
\end{equation*}
\end{lem}
\begin{proof}
Set $B_1=\langle b,u,v,w\rangle$ and consider the images of its generators with respect to the canonical projection $\pi_B\colon\Gamma\to\nicefrac{\Gamma}{B}$. We find
\begin{align*}
\pi_B(b)&=B,\\
\pi_B(u)	&=\pi_B((bada)(bada))=\\
				&=\pi_B(b)\pi_B(ada)\pi_B(b)\pi_B(ada)=\\
				&=\pi_B(ada)\pi_B(ada)=\\
				&=\pi_B((ada)(ada))=\pi_B(\id)=B,\\
\pi_B(v)	&=\pi_B(aba)^2=(\pi_B(a)\pi_B(b)\pi_B(a))^2=\\
				&=\pi_B(aa)^2=B \\
\intertext{ and}
\pi_B(w)	&=\pi_B(abad)^2=(\pi_B(a)\pi_B(b)\pi_B(ad))^2=\\
		&=\pi_B(aad)^2=\pi_B(d)^2=B.
\end{align*}
As a consequence, all $b,u,v,w$ are contained in $B$ and therefore $B_1\subseteq B$ holds. We continue by proving that $B_1$ is normal, so that $B_1=B$.\\
By \cref{thm:Three Generators Suffice} it suffices to show that the conjugates of $b,u,v,$ $w$ by $a,b$ and $c$ are contained in $B_1$. As for the conjugates of $b$, we find $aba=v, bbb=b$ and $cbc=b$. The last two identities follow directly from \cref{thm:Three Generators Suffice}. The conjugates by $a$ are
\begin{equation*}
aua=w, \quad ava=b \text{ and } awa=u.
\end{equation*}
Since $b$ is an element of $B_1$, conjugates by $b$ are clear. Thus, the claim is proven by the following three computations, were we use the identity $(ad)^4=\id$, which was proven in \cref{thm:D4}.
\begin{align*}
cuc	&=c(bada)(bada)c=(dada)b(ada)c=\\
	&=(dada)(dadaadad)b(ada)c=\\
	&=(adad)b(ada)c\overset{\text{\cref{thm:Three Generators Suffice}}}{=}(ada)bd(ada)c=\\
	&=(adab)(dada)(dadaadad)c=(adab)(adad)c=\\
	&=(adab)(adab)=u^{-1}\in B_1\\
cvc	&=c(aba)c=bd(aba)db=b(daba)(dabaabad)db=\\
	&=b(daba)(daba)(aba)b=bw^{-1}vb\in B_1\\
cwc	&=c(abad)(abad)c=bd(aba)(daba)b=bw^{-1}b\in B_1
\end{align*}
We will now consider the index of $B=B_1$ in $\Gamma$. In the canonical projection $\pi_B\colon\Gamma\to\nicefrac{\Gamma}{B}$ the image of $b$ is $B$. Thus, \cref{thm:Three Generators Suffice} implies, that $\pi_B(b)=\pi_B(c)$. As a consequence, $\nicefrac{\Gamma}{B}$ is generated by the images of $a$ and $d$. It follows from \cref{thm:D4}, that the order of $\nicefrac{\Gamma}{B}$ is bound by $\lvert\langle a,d\rangle\rvert=\lvert D_4\rvert=8$.\\
Remember the homomorphism $\varpi_3\colon\Gamma\to\mathfrak{S}_8$, which maps each automorphism to its permutation of the vertices on level 3. Since
\begin{equation*}
\psi_1(b)=(a,c)\text{ and }\psi_1(c)=(a,d)
\end{equation*}
hold, automorphism $b$ interchanges the subtrees $\ST{(0,0)}$ and $\ST{0,1}$ as well as $\ST{1,0,0}$ and $\ST{1,0,1}$. By computing the images of 
\begin{equation*}
b((0,0,0)),b((0,0,1)),\ldots,b((1,1,1))
\end{equation*}
 under $\beta_k$ we find
\begin{equation*}
\varpi_3(b)=(1\,3)(2\,4)(5\,6).
\end{equation*}
\cref{g:pi b} visualizes the computation above. Analogously one derives the images of the other generators under $\varpi_3$. As described in \cref{g:pi u,g:pi v,g:pi w} their images are
\begin{align*}
\varpi_3(u)	&=(1\,2)(3\,4),\\
\varpi_3(v)	&=(1\,2)(5\,7)(6\,8)\text{ and}\\
\varpi_3(w)	&=(5\,6)(7\,8).
\end{align*}
It is an easy but rather time consuming matter to find the subgroup of $\mathfrak{S}_8$ that is generated by these four elements. For this reason, I have used the program \textsc{Wolfram Mathematica$^\circledR$ 9} to obtain it. My program code can be found in \ref{sec:Appendix} Appendix. For this proof we require the result
\begin{equation*}
\lvert\varpi_3(B)\rvert=\lvert\langle\varpi_3(b),\varpi_3(u),\varpi_3(v),\varpi_3(w)\rangle\rvert=2^4=16.
\end{equation*}
Remember that $\lvert \varpi_3(\Gamma)\rvert=128$ holds by \cref{thm:pi Gamma}. As for the index of $B$ in $\Gamma$, we can now find an upper bound
\begin{equation*}
[\Gamma:B]\geq [\varpi_3(\Gamma):\varpi_3(B)]=\frac{2^7}{2^4}=2^3=8,
\end{equation*}
which proves the claim, as we have already shown, that $[\Gamma:B]\leq 8$.
\end{proof}


\begin{figure}
\begin{center}
\subfloat[$\varpi_3(b)=(1\,3)(2\,4)(5\,6)$\label{g:pi b}]{
\includegraphics[width=\textwidth]{./res/varpi_of_b}
%%\beginpgfgraphicnamed{varpi_of_b}
%\begin{tikzpicture}[level distance=7.5mm, baseline=(b.base)]
%\node (b) {$\emptyset$}
%  child foreach \i in {0,1} {node {\i}
% {child foreach \j in {0,1} {node {\j}
% {child foreach \k in {0,1} {node {\k}
% }}}}};
%
%\begin{scope}[node distance=7mm, font=normall]
%\node[draw=none,rectangle] [above of=b] (ST) {$b$};
%
%\node[draw=none] [above of=b-1] (ST0) {$a$};
%\node[draw=none] [above of=b-2] (ST1) {$c$};
%\end{scope}
%
%\begin{scope}[bgn]
%\node[draw=none] [above of=b-2-1] (ST10) {$a$};
%\node[draw=none] [above of=b-2-2] (ST11) {$d$};
%\end{scope}
%
%\begin{scope}[bgn]
%\tikzstyle{every node}=[rectangle,draw=none,font=\scriptsize]
%
%\path[hlarrow] (b-1-1) .. controls +(1.5,-0.5) .. (b-1-2);
%
%\path[hlarrow] (b-2-1-1) .. controls +(0.75,-0.5) .. (b-2-1-2);
%
%\path (b-1-1-1) ++(0,-8mm) -- ++(-155:18mm) node[midway,sloped] {$\beta_3(0,1,0)=3$};
%\path (b-1-1-2) ++(0,-8mm) -- ++(-155:18mm) node[midway,sloped] {$\beta_3(0,1,1)=4$};
%\path (b-1-2-1) ++(0,-8mm) -- ++(-155:18mm) node[midway,sloped] {$\beta_3(0,0,0)=1$};
%\path (b-1-2-2) ++(0,-8mm) -- ++(-155:18mm) node[midway,sloped] {$\beta_3(0,0,1)=2$};
%
%\path (b-2-1-1) ++(0,-8mm) -- ++(-155:18mm) node[midway,sloped] {$\beta_3(1,0,1)=6$};
%\path (b-2-1-2) ++(0,-8mm) -- ++(-155:18mm) node[midway,sloped] {$\beta_3(1,0,0)=5$};
%\path (b-2-2-1) ++(0,-8mm) -- ++(-155:18mm) node[midway,sloped] {$\beta_3(1,1,0)=7$};
%\path (b-2-2-2) ++(0,-8mm) -- ++(-155:18mm) node[midway,sloped] {$\beta_3(1,1,1)=8$};
%\end{scope}
%
%\node[fill=black!20] at (b-1-1) {1};
%\node[fill=black!20] at (b-1-2) {0};
%
%\node[fill=black!40] at (b-2-1-1) {1};
%\node[fill=black!40] at (b-2-1-2) {0};
%
%
%
%\begin{pgfonlayer}{background}
%
%%\filldraw[st] 	(ST0.north)  -- (b-1-2.north east) -- (b-1-2-2.east) to[out=-45,in=0] (b-1-2-2.south) --(b-1-1-1.south) to[out=-180,in=225] (b-1-1-1.west) -- (b-1-1.north west) -- cycle;
%\filldraw[black!20,line width=10mm, line join=round] 	(ST0.base)  -- (b-1-2.base) -- (b-1-2-2.base) --(b-1-1-1.base)  -- (b-1-1.base) -- cycle;
%\filldraw[black!20,line width=10mm, line join=round] 	(ST1.base)  -- (ST11.base) -- (b-2-2-2.base) -- (b-2-1-1.base) -- (ST10.base) -- cycle;
%
%\filldraw[black!40,line width=10mm, line join=round] 	(ST10.base) -- (b-2-1-1.base)  -- (b-2-1-2.base)  -- cycle;
%\filldraw[black!40,line width=10mm, line join=round] 	(ST11.base) -- (b-2-2-1.base)  -- (b-2-2-2.base)  -- cycle;
%\end{pgfonlayer}
%\end{tikzpicture}
%%\endpgfgraphicnamed

}

\subfloat[$\varpi_3(u)=(1\,2)(3\,4)$\label{g:pi u}]{
\includegraphics[width=\textwidth]{./res/varpi_of_u}
%\begin{tikzpicture}[level distance=7.5mm, baseline=(b.base)]
%\node (b) {$\emptyset$}
%  child foreach \i in {0,1} {node {\i}
% {child foreach \j in {0,1} {node {\j}
% {child foreach \k in {0,1} {node {\k}
% }}}}};
%
%\begin{scope}[node distance=7mm, font=normall]
%\node[draw=none,rectangle] [above of=b] (ST) {$u=b(ada)b(ada)$};
%
%\node[draw=none] [above of=b-1] (ST0) {$(aba)b$};
%\node[draw=none] [above of=b-2] (ST1) {$\id$};
%\end{scope}
%
%\begin{scope}[bgn]
%\node[draw=none] [above of=b-1-1] (ST00) {$ca$};
%\node[draw=none] [above of=b-1-2] (ST01) {$ac$};
%\end{scope}
%
%\begin{scope}[bgn]
%\tikzstyle{every node}=[rectangle,draw=none,font=\scriptsize]
%
%\path[hlarrow] (b-1-1-1) .. controls +(0.75,-0.5) .. (b-1-1-2);
%
%\path[hlarrow] (b-1-2-1) .. controls +(0.75,-0.5) .. (b-1-2-2);
%
%%\path (b-1-1-1) ++(0,-8mm) -- ++(-155:18mm) node[midway,sloped] {$\beta_3(0,0,1)=2$};
%%\path (b-1-1-2) ++(0,-8mm) -- ++(-155:18mm) node[midway,sloped] {$\beta_3(0,0,0)=1$};
%%\path (b-1-2-1) ++(0,-8mm) -- ++(-155:18mm) node[midway,sloped] {$\beta_3(0,1,1)=4$};
%%\path (b-1-2-2) ++(0,-8mm) -- ++(-155:18mm) node[midway,sloped] {$\beta_3(0,1,0)=3$};
%%
%%\path (b-2-1-1) ++(0,-8mm) -- ++(-155:18mm) node[midway,sloped] {$\beta_3(1,0,0)=5$};
%%\path (b-2-1-2) ++(0,-8mm) -- ++(-155:18mm) node[midway,sloped] {$\beta_3(1,0,1)=6$};
%%\path (b-2-2-1) ++(0,-8mm) -- ++(-155:18mm) node[midway,sloped] {$\beta_3(1,1,0)=7$};
%%\path (b-2-2-2) ++(0,-8mm) -- ++(-155:18mm) node[midway,sloped] {$\beta_3(1,1,1)=8$};
%
%\path (b-1-1-1) ++(0,-8mm) node {$2$} -- ++(-16.3135mm,0);
%\path (b-1-1-2) ++(0,-8mm) node {$1$};
%\path (b-1-2-1) ++(0,-8mm) node {$4$};
%\path (b-1-2-2) ++(0,-8mm) node {$3$};
%
%\path (b-2-1-1) ++(0,-8mm) node {$5$};
%\path (b-2-1-2) ++(0,-8mm) node {$6$};
%\path (b-2-2-1) ++(0,-8mm) node {$7$};
%\path (b-2-2-2) ++(0,-8mm) node {$8$};
%\end{scope}
%
%\node[fill=black!40] at (b-1-1-1) {1};
%\node[fill=black!40] at (b-1-1-2) {0};
%
%\node[fill=black!40] at (b-1-2-1) {1};
%\node[fill=black!40] at (b-1-2-2) {0};
%
%
%
%\begin{pgfonlayer}{background}
%
%\filldraw[black!20,line width=10mm, line join=round] 	(ST0.base)  -- (ST01.base) -- (b-1-2-2.base) --(b-1-1-1.base)  -- (ST00.base) -- cycle;
%\filldraw[black!20,line width=10mm, line join=round] 	(ST1.base)  -- (b-2-2.base) -- (b-2-2-2.base) -- (b-2-1-1.base) -- (b-2-1.base) -- cycle;
%
%\filldraw[black!40,line width=10mm, line join=round] 	(ST00.base) -- (b-1-1-1.base)  -- (b-1-1-2.base)  -- cycle;
%\filldraw[black!40,line width=10mm, line join=round] 	(ST01.base) -- (b-1-2-1.base)  -- (b-1-2-2.base)  -- cycle;
%\end{pgfonlayer}
%\end{tikzpicture}
}

\subfloat[$\varpi_3(v)=(1\,2)(5\,7)(6\,8)$\label{g:pi v}]{
\includegraphics[width=\textwidth]{./res/varpi_of_v}
%\begin{tikzpicture}[level distance=7.5mm, baseline=(b.base)]
%\node (b) {$\emptyset$}
%  child foreach \i in {0,1} {node {\i}
% {child foreach \j in {0,1} {node {\j}
% {child foreach \k in {0,1} {node {\k}
% }}}}};
%
%\begin{scope}[node distance=7mm, font=normall]
%\node[draw=none,rectangle] [above of=b] (ST) {$v=aba$};
%
%\node[draw=none] [above of=b-1] (ST0) {$c$};
%\node[draw=none] [above of=b-2] (ST1) {$a$};
%\end{scope}
%
%\begin{scope}[bgn]
%\node[draw=none] [above of=b-1-1] (ST00) {$a$};
%\node[draw=none] [above of=b-1-2] (ST01) {$d$};
%\end{scope}
%
%\begin{scope}[bgn]
%\tikzstyle{every node}=[rectangle,draw=none,font=\scriptsize]
%
%\path[hlarrow] (b-2-1) .. controls +(1.5,-0.5) .. (b-2-2);
%
%\path[hlarrow] (b-1-1-1) .. controls +(0.75,-0.5) .. (b-1-1-2);
%
%%\path (b-1-1-1) ++(0,-8mm) -- ++(-155:18mm) node[midway,sloped] {$\beta_3(0,0,1)=2$};
%%\path (b-1-1-2) ++(0,-8mm) -- ++(-155:18mm) node[midway,sloped] {$\beta_3(0,0,0)=1$};
%%\path (b-1-2-1) ++(0,-8mm) -- ++(-155:18mm) node[midway,sloped] {$\beta_3(0,1,0)=3$};
%%\path (b-1-2-2) ++(0,-8mm) -- ++(-155:18mm) node[midway,sloped] {$\beta_3(0,1,1)=4$};
%%
%%\path (b-2-1-1) ++(0,-8mm) -- ++(-155:18mm) node[midway,sloped] {$\beta_3(1,1,0)=7$};
%%\path (b-2-1-2) ++(0,-8mm) -- ++(-155:18mm) node[midway,sloped] {$\beta_3(1,1,1)=8$};
%%\path (b-2-2-1) ++(0,-8mm) -- ++(-155:18mm) node[midway,sloped] {$\beta_3(1,0,0)=5$};
%%\path (b-2-2-2) ++(0,-8mm) -- ++(-155:18mm) node[midway,sloped] {$\beta_3(1,0,1)=6$};
%
%\path (b-1-1-1) ++(0,-8mm) node {$2$} -- ++(-16.3135mm,0);
%\path (b-1-1-2) ++(0,-8mm) node {$1$};
%\path (b-1-2-1) ++(0,-8mm) node {$3$};
%\path (b-1-2-2) ++(0,-8mm) node {$4$};
%
%\path (b-2-1-1) ++(0,-8mm) node {$7$};
%\path (b-2-1-2) ++(0,-8mm) node {$8$};
%\path (b-2-2-1) ++(0,-8mm) node {$5$};
%\path (b-2-2-2) ++(0,-8mm) node {$6$};
%\end{scope}
%
%\node[fill=black!20] at (b-2-1) {1};
%\node[fill=black!20] at (b-2-2) {0};
%
%\node[fill=black!40] at (b-1-1-1) {1};
%\node[fill=black!40] at (b-1-1-2) {0};
%
%
%
%\begin{pgfonlayer}{background}
%
%\filldraw[black!20,line width=10mm, line join=round] 	(ST0.base)  -- (ST01.base) -- (b-1-2-2.base) --(b-1-1-1.base)  -- (ST00.base) -- cycle;
%\filldraw[black!20,line width=10mm, line join=round] 	(ST1.base)  -- (b-2-2.base) -- (b-2-2-2.base) -- (b-2-1-1.base) -- (b-2-1.base) -- cycle;
%
%\filldraw[black!40,line width=10mm, line join=round] 	(ST00.base) -- (b-1-1-1.base)  -- (b-1-1-2.base)  -- cycle;
%\filldraw[black!40,line width=10mm, line join=round] 	(ST01.base) -- (b-1-2-1.base)  -- (b-1-2-2.base)  -- cycle;
%\end{pgfonlayer}
%\end{tikzpicture}
}

\subfloat[$\varpi_3(w)=(5\,6)(7\,8)$\label{g:pi w}]{
\includegraphics[width=\textwidth]{./res/varpi_of_w}
%\begin{tikzpicture}[level distance=7.5mm, baseline=(b.base)]
%\node (b) {$\emptyset$}
%  child foreach \i in {0,1} {node {\i}
% {child foreach \j in {0,1} {node {\j}
% {child foreach \k in {0,1} {node {\k}
% }}}}};
%
%\begin{scope}[node distance=7mm, font=normall]
%\node[draw=none,rectangle] [above of=b] (ST) {$w=(aba)d(aba)d$};
%
%\node[draw=none] [above of=b-1] (ST0) {$\id$};
%\node[draw=none] [above of=b-2] (ST1) {$(aba)b$};
%\end{scope}
%
%\begin{scope}[bgn]
%\node[draw=none] [above of=b-2-1] (ST10) {$ca$};
%\node[draw=none] [above of=b-2-2] (ST11) {$ac$};
%\end{scope}
%
%\begin{scope}[bgn]
%\tikzstyle{every node}=[rectangle,draw=none,font=\scriptsize]
%
%\path[hlarrow] (b-2-1-1) .. controls +(0.75,-0.5) .. (b-2-1-2);
%
%\path[hlarrow] (b-2-2-1) .. controls +(0.75,-0.5) .. (b-2-2-2);
%
%%\path (b-1-1-1) ++(0,-8mm) -- ++(-155:18mm) node[midway,sloped] {$\beta_3(0,0,0)=1$};
%%\path (b-1-1-2) ++(0,-8mm) -- ++(-155:18mm) node[midway,sloped] {$\beta_3(0,0,1)=2$};
%%\path (b-1-2-1) ++(0,-8mm) -- ++(-155:18mm) node[midway,sloped] {$\beta_3(0,1,0)=3$};
%%\path (b-1-2-2) ++(0,-8mm) -- ++(-155:18mm) node[midway,sloped] {$\beta_3(0,1,1)=4$};
%%
%%\path (b-2-1-1) ++(0,-8mm) -- ++(-155:18mm) node[midway,sloped] {$\beta_3(1,0,1)=6$};
%%\path (b-2-1-2) ++(0,-8mm) -- ++(-155:18mm) node[midway,sloped] {$\beta_3(1,0,0)=5$};
%%\path (b-2-2-1) ++(0,-8mm) -- ++(-155:18mm) node[midway,sloped] {$\beta_3(1,1,1)=8$};
%%\path (b-2-2-2) ++(0,-8mm) -- ++(-155:18mm) node[midway,sloped] {$\beta_3(1,1,0)=7$};
%
%\path (b-1-1-1) ++(0,-8mm) node {$1$} -- ++(-16.3135mm,0);
%\path (b-1-1-2) ++(0,-8mm) node {$2$};
%\path (b-1-2-1) ++(0,-8mm) node {$3$};
%\path (b-1-2-2) ++(0,-8mm) node {$4$};
%
%\path (b-2-1-1) ++(0,-8mm) node {$6$};
%\path (b-2-1-2) ++(0,-8mm) node {$5$};
%\path (b-2-2-1) ++(0,-8mm) node {$8$};
%\path (b-2-2-2) ++(0,-8mm) node {$7$};
%\end{scope}
%
%\node[fill=black!40] at (b-2-2-1) {1};
%\node[fill=black!40] at (b-2-2-2) {0};
%
%\node[fill=black!40] at (b-2-1-1) {1};
%\node[fill=black!40] at (b-2-1-2) {0};
%
%
%
%\begin{pgfonlayer}{background}
%
%%\filldraw[st] 	(ST0.north)  -- (b-1-2.north east) -- (b-1-2-2.east) to[out=-45,in=0] (b-1-2-2.south) --(b-1-1-1.south) to[out=-180,in=225] (b-1-1-1.west) -- (b-1-1.north west) -- cycle;
%\filldraw[black!20,line width=10mm, line join=round] 	(ST0.base)  -- (b-1-2.base) -- (b-1-2-2.base) --(b-1-1-1.base)  -- (b-1-1.base) -- cycle;
%\filldraw[black!20,line width=10mm, line join=round] 	(ST1.base)  -- (ST11.base) -- (b-2-2-2.base) -- (b-2-1-1.base) -- (ST10.base) -- cycle;
%
%\filldraw[black!40,line width=10mm, line join=round] 	(ST10.base) -- (b-2-1-1.base)  -- (b-2-1-2.base)  -- cycle;
%\filldraw[black!40,line width=10mm, line join=round] 	(ST11.base) -- (b-2-2-1.base)  -- (b-2-2-2.base)  -- cycle;
%\end{pgfonlayer}
%\end{tikzpicture}
}
\caption{Projection of $B$ on $\St[\Gamma]{3}$}
\label{g:B on St Gamma 3}
\end{center}
\end{figure}

Let $K$ denote the \emph{normal} subgroup of $\Gamma$ generated by $(ab)^2$. The following lemma is taken from \cite{de2000topics}. It will provide essential tools for the proof of the main result stated in \cref{thm:infinite growth}.
\begin{lem}
\begin{thmlist}
\item The group $K$ is generated by the elements
\begin{equation*}
t=(ab)^2,\quad u=(bada)^2\text{ and }w=(abad)^2.
\end{equation*}
\item The index of $K$ in $\Gamma$ is 16.
\item $K$ is a subgroup of $\St[\Gamma]{1}$ and $\St[\Gamma]{3}$ is a subgroup of $K$.
\item The Cartesian product $K\times K$ is contained in the image $\psi_1(K).$
\item \label{thm:psi k(K)} For each integer $k\geq 1$, the image $\psi_k\left(\St[\Gamma]{k}\right)$ contains $K^{2^k}$.
\item \label{thm:psi4((ab)8)} The element $(ab)^8\in K$ is contained in $\St[\Gamma]{4}$. It fulfils
\begin{align*}
\psi_3((ab)^8)&=(b,b,b,b,b,b,b,b)\text{ and by \cref{thm:psi k}}\\
\psi_4((ab)^8)&=(a,c,a,c,a,c,a,c,a,c,a,c,a,c,a,c).
\end{align*}
\end{thmlist}
\end{lem}
\begin{proof}
\begin{plist}
\item Let $K_1=\langle t,u,w\rangle$. By construction, $t$ is an element of $K$. Since $K$ is a normal subgroup of $\Gamma$, we find
\begin{align*}
td^{-1}t^{-1}d=(abab)d(baba)d\overset{\text{\cref{thm:Three Generators Suffice}}}{=}&\\
(abad)(abad)=w&\in K\\
\intertext{and}
a^{-1}wa=a(abad)(abad)a=v&\in K.
\end{align*}
As a consequence, $K_1\subseteq K$. We prove claim (i) by showing, that $K_1$ is a normal subgroup of $\Gamma$, so that $K_1=K$. This follows from the computations below.
\begin{align*}
ata	&=a(ab)(ab)a=baba=t^{-1}\in K_1\\
btb	&=b(ab)(ab)b=baba=t^{-1}\in K_1\\
dtd	&=d(ab)(ab)d\overset{\text{\cref{thm:Three Generators Suffice}}}{=}\\
		&=dabadb=dabad(abaaba)b=w^{-1}t\in K_1\\[1em]
aua	&=a(bada)(bada)a=(abad)(abad)=w\in K_1\\
bub	&=b(bada)(bada)b=(adab)(adab)=u^{-1}\in K_1\\
dud	&=d(bada)(bada)d\overset{\text{\cref{thm:Three Generators Suffice}}}{=}b(dada)b(ada)d=\\
	&=b(dada)(dadaadad)b(ada)d=\\
	&=b(adad)b(ada)d\overset{\text{\cref{thm:Three Generators Suffice}}}{=}(bada)b(dada)d=\\
	&=(bada)b(adaddada)(dada)d=\\
	&=(bada)b(adad)d=(bada)(bada)=u\in K_1\\[1em]
awa	&=a(abad)(abad)a=(bada)(bada)=u\in K_1\\
bwb	&=b(abad)(abad)b=ba(bada)(bada)ab=\\
		&=bab(adab)(adab)bab=\\
		&=(baba)(daba)(daba)(abab)=t^{-1}w^{-1}t\in K_1\\
dwd	&=d(abad)(abad)d=(daba)(daba)=w^{-1}\in K_1
\end{align*}
\item\label{p:Index of K in Gamma} By \cref{thm:B} the quotient group $\nicefrac{B}{K}$ is generated by the equivalence classes of $b$ and $v=aba$. As these automorphisms are self-inverse and $(aba)b\in K$, by the uniqueness of the inverse element their equivalence classes  coincide. As a result we get $\lvert\nicefrac{B}{K}\rvert=[B:K]\leq 2$. Since the index of $B$ in $\Gamma$ equals 8 by \cref{thm:B}, it suffices to prove that $B\neq K$.\\
For this purpose, we consider the subgroup $\varpi_3(K)\subseteq \mathfrak{S}_8$. Using \textsc{Wolfram Mathematica$^\circledR$ 9} and the routine of \ref{sec:Appendix} Appendix, one finds $\lvert\varpi_3(K)\rvert=8$. In the proof of \cref{thm:B} we have seen, that $\lvert\varpi_3(B)\rvert=16$, so that the claim is proven.
\item Since $\lbrace t,u,w\rbrace\subseteq\St[\Gamma]{1}$ the inclusion $\St[\Gamma]{1}\subseteq K$ is trivial.\\
Consider the canonical projection $\pi\colon\Gamma\to\St[\Gamma]{3}$. There exists an isomorphism between $\nicefrac{K}{\St[\Gamma]{3}}=\pi(K)$ and $\nicefrac{K}{(K\cap\St[\Gamma]{3})}$. This follows from the fact, that two elements $\gamma,\eta$ of $K$ are in the same equivalence class of the group $\nicefrac{K}{\St[\Gamma]{3}}$ if and only if $\gamma\eta^{-1}\in K$ is contained in $\St[\Gamma]{3}$. Therefore, the composition $\gamma\eta^{-1}$ is an element of $K\cap\St[\Gamma]{3}$. The quotient group $\nicefrac{K}{K\cap\St[\Gamma]{3}}$ is isomorph to $\nicefrac{K\St[\Gamma]{3}}{\St[\Gamma]{3}}$ by the first isomorphism theorem , so that
\begin{equation*}
[K\St[\Gamma]{3}:\St[\Gamma]{3}]=[K:K\cap\St[\Gamma]{3}]=\lvert \pi(K)\rvert=\lvert\varpi_3(K)\rvert=2^3=8.
\end{equation*}
It follows that
\begin{equation*}
[\Gamma:K\St[\Gamma]{3}]=\frac{[\Gamma:\St[\Gamma]{3}]}{[K\St[\Gamma]{3}:\St[\Gamma]{3}]}=\frac{2^7}{2^3}=2^4,
\end{equation*}
while \ref{p:Index of K in Gamma} implies $[\Gamma:K]=2^4.$ Thus, the obvious inclusion $K\subseteq K\St[\Gamma]{3}$ is in fact an equality.
\item \label{p:Base Case psi(K) contains K} \cref{thm:Conjugation with a inverts the performing of b c d} implies the identity
\begin{align*}
\psi_1(w)	&=\psi_1(aba)\psi_1(d)\psi_1(aba)\psi_1(d)=\\
			&=(c,a)(\id,b)(c,a)(\id,b)=(\id,t)\in\psi_1(K).
\end{align*}
Because $K$ is a normal subgroup of $\Gamma$, the automorphism $wcw^{-1}c$ is an element of $K$. Therefore, the equation above implies
\begin{align*}
\psi_1(wcw^{-1}c)	&=\psi_1(w)\psi_1(c)\psi_1(w)^{-1}\psi_1(c)=\\
					&=(\id,t)(a,d)(\id,t^{-1})(a,d)=\\
					&=(\id,t)(\id,dt^{-1}d)\overset{(i)}{=}(\id,t)(\id,t^{-1}w)=\\
					&=(\id,w)\in\psi_1(K).
\end{align*}
The group $\id\times K$ is contained in $\psi_1(K)$, since
\begin{align*}
\psi_1(aca(wcw^{-1}c)aca)	&=\psi(aca)\psi_1(wcw^{-1}c)\psi_1(aca)=\\
							&=(d,a)(\id,w)(d,a)=(\id,awa)\overset{(i)}{=}\\
							&=(\id,u)\in\psi_1(K).
\end{align*}
As a consequence of \cref{thm:Conjugation with a inverts the performing of b c d}, the images of $awa,awcw^{-1}ca$ and $ca(wcw^{-1}c)ac\in K$ generate $K\times\id$. Thus, $K\times K\subseteq\psi_1(K)$.
\item We prove the claim by induction on $k$. The case $k=1$ was shown in \ref{p:Base Case psi(K) contains K}. Hence, we assume the claim to be proven for all mappings $\psi_l\colon\St[\Gamma]{l}\to\Gamma^{2^l}$, where $l<k$. We have seen, that the homomorphisms
\begin{align*}
\psi_{\text{left}}\colon \St[\Gamma]{1}\to \Gamma,&\text{ defined by }\gamma\mapsto(\pi_1\circ\psi_1)(\gamma),\\
\psi_{\text{right}}\colon \St[\Gamma]{1}\to \Gamma,&\text{ defined by }\gamma\mapsto(\pi_2\circ\psi_1)(\gamma),
\end{align*}
fulfil $K\subseteq\psi_{\text{left}}(K)$ and $K\subseteq\psi_{\text{right}}(K)$ in the proof of \ref{p:Base Case psi(K) contains K}. As $K^{2^{k-1}}\subseteq\St[\Gamma]{1}^{2^{k-1}}$, it follows that
\begin{equation*}
\psi_{k-1}^{-1}\left(K^{2^{k-1}}\right)\subseteq \psi_{k-1}^{-1}\left(\St[\Gamma]{1}^{2^{k-1}}\right)=\St[\Gamma]{k}
\end{equation*}
from \cref{thm:psi k}. By the induction hypothesis $K^{2^{k-1}}$ is a subgroup of $\psi_{k-1}(\St[\Gamma]{k-1})$. As a result of the equation above $K^{2^{k-1}}$ is in fact contained in $\psi_{k-1}(\St[\Gamma]{k})$, so that the identity 
\begin{align*}
\psi_{k}(\St[\Gamma]{k})	&=(\psi_{1}(\psi_{k}(\St[\Gamma]{k})),\ldots\psi_{1}(\psi_{k}(\St[\Gamma]{k})))_k\subseteq\\
						&\subseteq(\psi_{1}(K),\ldots,\psi_{1}(K))_K\subseteq K^{2^k}.
\end{align*}
holds by \cref{thm:psi k}.
\item We apply the formula obtained in \cref{thm:psi k} to compute $\psi_3((ab)^8)$. Since the number of occurrences of automorphism $a$ in $(ab)^8$ is even, $(ab)^8\in\St[\Gamma]{1}$ by \cref{thm:StGamma1 is a proper subset of Gamma}. We find
\begin{align*}
\psi_1((ab)^8)&=\psi_1(abab)^4=(\psi_1(aba)\psi_1(b))^4=\\
						&=((c,a)(a,c))^4=((ca)^4,(ac)^4).
\end{align*}
We may apply $\psi_2$ to $(ab)^8$ because $(ca)^4$ and $(ac)^4$ leave the first level of $\T$ invariant. Now \cref{thm:psi k} implies
\begin{align*}
\psi_2((ab)^8)	&=\left(\psi_1(caca)^2,\psi_1(acac)^2\right)_2=\\
				&=\left(((a,d)(d,a))^2,((d,a)(a,d))^2\right)_2=\\
				&=(adad,dada,dada,adad).
\end{align*}
It follows that 
\begin{align*}
\psi_3	&((ab)^8)=(\psi_1(adad),\psi_1(dada),\psi_1(dada),\psi_1(adad))_3=\\
			&=((b,\id)(\id,b),(\id,b)(b,\id),(\id,b)(b,\id),(b,\id)(\id,b))_3=\\
			&=(b,b,b,b,b,b,b,b).
\end{align*}
As a consequence, $(ab)^8$ is an element of $\St[\Gamma]{4}$.
\end{plist}
\end{proof}

The following and final lemma will provide the inductive step for \cref{thm:infinite growth}.

\begin{lem}\label{thm:Inductive Step Infinite Growth}
Let $n$ be a positive integer. Given an automorphism $\gamma\in K$, such that $\ord(\gamma)\geq 2^n$, we find an automorphism $\eta\in K$ satisfying
\begin{equation*}
\ord(\eta)\geq 2^{n+1}. 
\end{equation*}
\end{lem}
\begin{proof}
By \cref{thm:psi k(K)} there exists an automorphism 
\begin{equation*}
h\in\St[\Gamma]{5}\subseteq K,
\end{equation*}
 such that
\begin{equation*}
\psi_5(h)=(\gamma,\id,\ldots,\id),
\end{equation*}
where on the right hand side $31=2^5-1$ entries equal the identity. We may rewrite this expression in the form
\begin{equation*}
\psi_5(h)=((\gamma,\id),\psi_1(\id),\ldots,\psi_1(\id))_5,
\end{equation*}
with $15=2^4-1$ coordinates equal to $\psi_1(\id)$. We set $\eta=(ab)^8h$, which is an automorphism of $K$ since $(ab)^8$ and $h$ are.\\

Since $\psi_1$ is injective, the preimage $\psi_1^{-1}(\lbrace(\vartheta_1,\vartheta_2)\rbrace)$ of the sing\-le\-ton $\lbrace(\vartheta_1,\vartheta_2)\rbrace\subseteq K^2$ is a well-defined singleton $\lbrace\vartheta\rbrace\subseteq K$ by \cref{thm:psi k(K)}. Thus, the homomorphism $\psi_1^{-1}\colon K\times K\to K$ is well-defined by $(\vartheta_1,\vartheta_2)\mapsto\vartheta$. As for automorphism $\eta$ it follows, that
\begin{align*}
\psi_4\left(\eta^2\right)	&=\left(\psi_4\left((ab)^8\right)\psi_4(h)\right)^2\overset{\text{\cref{thm:psi4((ab)8)}}}{=}\\
				&=\left((a,c,\ldots,a,c)\left(\psi_1^{-1}(\gamma,\id),\id,\ldots,\id\right)\right)^2=\\
				&=\left(a\,\psi_1^{-1}(\gamma,\id)\,a\,\psi_1^{-1}(\gamma,\id),\id,\ldots,\id\right)\overset{\text{\cref{thm:Conjugation with a inverts the performing of b c d} }}{=}\\
				&=\left(\psi_1^{-1}(\id,\gamma)\,\psi_1^{-1}(\gamma,\id),\id,\ldots,\id\right)=\\
				&=\left(\psi_1^{-1}(\gamma,\gamma),\id,\ldots,\id\right)\in\St[\Gamma]{1}^{16}.
\end{align*}
As a consequence of \cref{thm:psi k}, the  image of $\eta^2$ under $\psi_5$ equals
\begin{equation*}
\psi_5\left(\eta^2\right)=(\gamma,\gamma,\id,\ldots,\id)\in\St[\Gamma]{1}^{32},
\end{equation*}
which has an order greater than $2^n$. As homomorphism $\psi_5$ is a monomorphism of groups, it follows that
\begin{equation*}
\ord(\eta)\geq 2^{n+1},
\end{equation*}
so that the claim is proven.
\end{proof}

\begin{thm}\label{thm:infinite growth}
Let $n$ be a positive integer. There exists an automorphism $\gamma_{n}\in K\subseteq\Gamma$, such that
\begin{equation*}
\ord(\gamma)\geq 2^n.
\end{equation*}
\end{thm}

\begin{proof}
We prove the theorem by induction on $n$. The case $n=1$ is clear since $\id\in K$.\\
As induction hypothesis we assume the claim to be proven for all positive integers smaller than or equal to $n$, so that there exists an automorphism $\gamma_n\in K$, satisfying $\ord(\gamma_n)\geq 2^n$. \cref{thm:Inductive Step Infinite Growth} proves the existence of an automorphism $\gamma_{n+1}$ with an order exceeding $2^{n+1}$. Hence, the induction is finished.
\end{proof}
