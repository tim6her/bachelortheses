\thispagestyle{empty}

\begin{de}
\section*{Abriss}
\sloppypar%
Die vorliegende Arbeit beschäftigt sich mit der ersten Gri\-gor\-chuk-Grup\-pe, die 1980 von Rostislav Grigorchuk eingef\"uhrt wurde. Da es sich bei der ersten Grigorchuk-Gruppe um eine Untergruppe der Automorphismengruppe eines vollst\"andigen Bin\"arbaumes handelt, werden im ersten Abschnitt Grundz\"uge der Graphentheorie vorgestellt. Im zweiten Teil wird gezeigt, dass die erste Grigorchuk-Gruppe ein Gegenbeispiel zum \emph{Unbounded Burnside Problem} darstellt. Das 1902 von William Burnside formulierte \emph{Unbounded Burnside Problem} fragt nach der Existenz von  endlich erzeugten, unendlichen, periodischen Gruppen. Es gilt als wegbereitend f\"ur die Entwicklung der Gruppentheorie im 20. Jahrhundert. Der letzte Abschnitt dieser Arbeit ist dem Wachstum der ersten Grigorchuk-Gruppe gewidmet. Es wird bewiesen, dass diese Gruppe Automorphismen beliebiger Ordnung enth\"alt, sodass die erste Grigorchuk-Gruppe kein Gegenbeispiel f\"ur das \emph{Bounded Burnside Problem} ist.\par
\end{de}
\newpage

\section*{Abstract}
The subject of this bachelor thesis is the first Grigorchuk group, which was introduced by Rostislav Grigorchuk in 1980. Since the first Grigorchuk group is a subgroup of the automorphism group of a binary tree, the first section of this study is dedicated to elementary graph theory. In the second part it will be proven that the first Grigorchuk group is a counterexample to the \emph{unbounded Burnside problem}. This problem, which first occurred in a paper by William Burnside in 1902, asks whether each finitely generated, periodic group is finite. Burnside question had significant impact on the whole field of group theory throughout the 20th century. In the last section we will study the growth of the first Grigorchuk group. We will see, that it contains automorphisms of arbitrary order, so that it is not a counterexample to the \emph{bounded Burnside problem}.

\newpage
\chapter*{\contentsname}

\printcontents[parts]{1}{}{}