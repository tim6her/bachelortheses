% ****************************************************************************************************
% classicthesis-config.tex 
% formerly known as loadpackages.sty, classicthesis-ldpkg.sty, and classicthesis-preamble.sty 
% Use it at the beginning of your ClassicThesis.tex, or as a LaTeX Preamble 
% in your ClassicThesis.{tex,lyx} with % ****************************************************************************************************
% classicthesis-config.tex 
% formerly known as loadpackages.sty, classicthesis-ldpkg.sty, and classicthesis-preamble.sty 
% Use it at the beginning of your ClassicThesis.tex, or as a LaTeX Preamble 
% in your ClassicThesis.{tex,lyx} with % ****************************************************************************************************
% classicthesis-config.tex 
% formerly known as loadpackages.sty, classicthesis-ldpkg.sty, and classicthesis-preamble.sty 
% Use it at the beginning of your ClassicThesis.tex, or as a LaTeX Preamble 
% in your ClassicThesis.{tex,lyx} with % ****************************************************************************************************
% classicthesis-config.tex 
% formerly known as loadpackages.sty, classicthesis-ldpkg.sty, and classicthesis-preamble.sty 
% Use it at the beginning of your ClassicThesis.tex, or as a LaTeX Preamble 
% in your ClassicThesis.{tex,lyx} with \input{classicthesis-config}
% ****************************************************************************************************  
% If you like the classicthesis, then I would appreciate a postcard. 
% My address can be found in the file ClassicThesis.pdf. A collection 
% of the postcards I received so far is available online at 
% http://postcards.miede.de
% ****************************************************************************************************

% ****************************************************************************************************
% 1. Configure classicthesis for your needs here, e.g., remove "drafting" below 
% in order to deactivate the time-stamp on the pages
% ****************************************************************************************************
\PassOptionsToPackage{eulerchapternumbers,listings,drafting,%
				 pdfspacing,%floatperchapter,%linedheaders,%
				 subfig,beramono,eulermath,parts, a5paper}{classicthesis}										
% ********************************************************************
% Available options for classicthesis.sty 
% (see ClassicThesis.pdf for more information):
% drafting
% parts nochapters linedheaders
% eulerchapternumbers beramono eulermath pdfspacing minionprospacing
% tocaligned dottedtoc manychapters
% listings floatperchapter subfig
% ********************************************************************

% ********************************************************************
% Triggers for this config
% ******************************************************************** 
\usepackage{ifthen}
\newboolean{enable-backrefs} % enable backrefs in the bibliography
\setboolean{enable-backrefs}{false} % true false
% ****************************************************************************************************


% ****************************************************************************************************
% 2. Personal data and user ad-hoc commands
% ****************************************************************************************************
\newcommand{\myTitle}{Bachelortheses\xspace}
\newcommand{\mySubtitle}{Katze \and Hund\xspace}
\newcommand{\myDegree}{\xspace}
\newcommand{\myName}{Tim Benedikt Herbstrith\xspace}
\newcommand{\myProf}{Put name here\xspace}
\newcommand{\myOtherProf}{Put name here\xspace}
\newcommand{\mySupervisor}{Put name here\xspace}
\newcommand{\myFaculty}{Put data here\xspace}
\newcommand{\myDepartment}{Put data here\xspace}
\newcommand{\myUni}{Put data here\xspace}
\newcommand{\myLocation}{Vienna\xspace}
\newcommand{\myTime}{July 2015\xspace}
\newcommand{\myVersion}{version 0.1.1\xspace}

\newcommand{\ra}{\rightarrow}
\newcommand{\mt}{\mapsto}

% ********************************************************************
% Setup, finetuning, and useful commands
% ********************************************************************
\newcounter{dummy} % necessary for correct hyperlinks (to index, bib, etc.)
\newlength{\abcd} % for ab..z string length calculation
\providecommand{\mLyX}{L\kern-.1667em\lower.25em\hbox{Y}\kern-.125emX\@}
\newcommand{\ie}{i.\,e.}
\newcommand{\Ie}{I.\,e.}
\newcommand{\eg}{e.\,g.}
\newcommand{\Eg}{E.\,g.}

% New- and Renewcommands
\newcommand{\seq}[2][n]{#2_{1},\ldots,#2_{#1}}
\newcommand{\nicefrac}[2]{#1/#2}
\newcommand{\MtN}{\nicefrac{M}{N}}
\newcommand{\MTN}{\faktor{M}{N}}
\newcommand{\T}{T^{(2)}}
\newcommand{\VT}{V\left(\T\right)}
\newcommand{\ET}{E\left(\T\right)}
\newcommand{\ST}[1]{\T_{#1}}
\newcommand{\VST}[1]{V\left(\ST{#1}\right)}
\newcommand{\EST}[1]{E\left(\ST{#1}\right)}
\newcommand{\AutT}{\Aut\left(\T\right)}
\newcommand{\St}[2][]{\mathrm{St}_{#1}(#2)}
% ****************************************************************************************************


% ****************************************************************************************************
% 3. Loading some handy packages
% ****************************************************************************************************
% ******************************************************************** 
% Packages with options that might require adjustments
% ******************************************************************** 
\PassOptionsToPackage{utf8}{inputenc}	% latin9 (ISO-8859-9) = latin1+"Euro sign"
 \usepackage{inputenc}				

%\PassOptionsToPackage{ngerman,american}{babel}   % change this to your language(s)
% Spanish languages need extra options in order to work with this template
%\PassOptionsToPackage{spanish,es-lcroman}{babel}
 \usepackage{babel}
 \usepackage{csquotes}				

% BibLaTex
\usepackage[style=numeric,backend=bibtex]{biblatex}
\addbibresource{./parts/content/References2012}
\addbibresource{./parts/content/References2013}		

\PassOptionsToPackage{fleqn}{amsmath}		% math environments and more by the AMS 
 \usepackage{amsmath,amsfonts,amssymb,amsthm}

% ********************************************************************
% Math, Operators and Theorems
\usepackage{thmtools}	% More Flexibility in Theorem Styles + Provides in Combination with ref-Packages the 
						% Possibility to Refer to Theorem Style in Reference.
\usepackage{faktor}		% Displays factor groups

\numberwithin{equation}{section}

\DeclareMathOperator{\N}{\mathbb{N}}
\DeclareMathOperator{\Z}{\mathbb{Z}}
\DeclareMathOperator{\Q}{\mathbb{Q}}
\DeclareMathOperator{\R}{\mathbb{R}}
\DeclareMathOperator{\C}{\mathbb{C}}
\DeclareMathOperator{\F}{\mathbb{F}}

\DeclareMathOperator{\Aut}{Aut}
\DeclareMathOperator{\id}{id}

\DeclareMathOperator{\kernel}{ker}
\DeclareMathOperator{\im}{im}
\DeclareMathOperator{\End}{\mathrm{End}}
\DeclareMathOperator{\Hom}{\mathrm{Hom}}
\DeclareMathOperator{\Mod}{\mathrm{mod}}
\DeclareMathOperator{\D}{\mathrm{D}}
\DeclareMathOperator{\lcm}{\mathrm{lcm}}
\DeclareMathOperator{\ord}{\mathrm{ord}}



\declaretheorem[
	name=Theorem,
  	refname={theorem,theorems},		%Lower Case Versions of Theorem Type
  	Refname={Theorem,Theorems},
  	numberwithin=section]{thm}
\declaretheorem[
	name=Lemma,
	refname={lemma,lemmas},
	Refname={Lemma,Lemmas},
	sibling=thm]{lem}
\declaretheorem[
	name=Proposition,
	refname={proposition,propositions},
	Refname={Proposition,Propositions},
	sibling=thm]{pro}
\declaretheorem[
	name=Corollary,
	refname={corollary,corollarys},
	Refname={Corollary,Corollarys},
	sibling=thm]{cor}

\declaretheorem[
	name=Definition,
	style=definition,
	numbered=no]{defin}
\declaretheorem[
	name=Example,
	style=definition,
	numbered=no]{exam}
	
\declaretheorem[
	name=Remark,
	style=remark,
	numbered=no]{rem}

% ********************************************************************
%TikZ and TikZ-Styles
\usepackage[dvipsnames]{xcolor}
\usepackage{tikz}
%\pgfrealjobname{HerbstrithBachelorthesis}
%\usetikzlibrary{external}
%\tikzexternalize[prefix=figures/]

\usepackage{pgfcore}
\usetikzlibrary{arrows}
%\usepackage{pgfbaselayers}
%\usepackage{pgfbaseimage} %Externalize Graphics

\tikzstyle{every node}=[circle,draw=black,font=\small,text=black,inner sep=1pt, minimum size=5mm]
\tikzstyle{edge from parent}=[draw=black]
\tikzstyle{st}=[black!20, line join=round, line width=2mm]
\tikzstyle{st1}=[black!40, line join=round, line width=2mm]

\tikzstyle{bgn}=[node distance=6mm, font=normal]
\tikzstyle{hlarrow}=[<->, shorten <=1pt, shorten >=1pt, draw, thick]

\tikzstyle{level 1}=[sibling distance=60mm]
\tikzstyle{level 2}=[sibling distance=30mm]
\tikzstyle{level 3}=[sibling distance=15mm]
\tikzstyle{level 4}=[sibling distance=7.5mm]
\tikzstyle{level 5}=[sibling distance=3.75mm, level distance=5mm]


% ******************************************************************** 
% General useful packages
% ******************************************************************** 
\PassOptionsToPackage{T1}{fontenc} % T2A for cyrillics
	\usepackage{fontenc}     
\usepackage{textcomp} % fix warning with missing font shapes
\usepackage{scrhack} % fix warnings when using KOMA with listings package          
\usepackage{xspace} % to get the spacing after macros right  
\usepackage{mparhack} % get marginpar right
\usepackage{fixltx2e} % fixes some LaTeX stuff 
\PassOptionsToPackage{printonlyused,smaller}{acronym}
	\usepackage{acronym} % nice macros for handling all acronyms in the thesis
%\renewcommand*{\acsfont}[1]{\textssc{#1}} % for MinionPro
\renewcommand{\bflabel}[1]{{#1}\hfill} % fix the list of acronyms
% ****************************************************************************************************


% ****************************************************************************************************
% 4. Setup floats: tables, (sub)figures, and captions
% ****************************************************************************************************
\usepackage{tabularx} % better tables
	\setlength{\extrarowheight}{3pt} % increase table row height
\newcommand{\tableheadline}[1]{\multicolumn{1}{c}{\spacedlowsmallcaps{#1}}}
\newcommand{\myfloatalign}{\centering} % to be used with each float for alignment
\usepackage{caption}
\captionsetup{format=hang,font=small}
\usepackage{subfig} 

\usepackage{enumitem}
\newlist{thmlist}{enumerate}{1}			% thmlist-s may only be used in theorem environments
\setlist[thmlist]{label=(\roman{thmlisti}), ref=\thethm.(\roman{thmlisti}),noitemsep}
\newlist{exlist}{enumerate}{1}
\setlist[exlist]{label=(\arabic{exlisti}), ref=(\arabic{exlisti}),noitemsep,leftmargin=0pt,itemindent=2\parindent}
\newlist{plist}{enumerate}{1}
\setlist[plist]{label=(\roman{plisti}), ref=(\roman{plisti}),noitemsep,leftmargin=0pt,itemindent=2\parindent}
\newlist{clist}{enumerate}{2}
\setlist[clist]{label*=(\alph*), ref=(\alph*),noitemsep,leftmargin=0pt,itemindent=2\parindent} 

\usepackage{multienum}
% ****************************************************************************************************


% ****************************************************************************************************
% 5. Setup code listings
% ****************************************************************************************************
\usepackage{listings} 
%\lstset{emph={trueIndex,root},emphstyle=\color{BlueViolet}}%\underbar} % for special keywords
\lstset{language=[5.2]Mathematica,%C++,
    keywordstyle=\color{RoyalBlue},%\bfseries,
    basicstyle=\small\ttfamily,
    %identifierstyle=\color{NavyBlue},
    commentstyle=\color{Green}\ttfamily,
    stringstyle=\rmfamily,
    numbers=none,%left,%
    numberstyle=\scriptsize,%\tiny
    stepnumber=5,
    numbersep=8pt,
    showstringspaces=false,
    breaklines=true,
    frameround=ftff,
    frame=single,
    belowcaptionskip=.75\baselineskip
    %frame=L
} 
% ****************************************************************************************************    		   


% ****************************************************************************************************
% 6. PDFLaTeX, hyperreferences and citation backreferences
% ****************************************************************************************************
% ********************************************************************
% Using PDFLaTeX
% ********************************************************************
\PassOptionsToPackage{hyperfootnotes=false,pdfpagelabels}{hyperref}
	\usepackage{hyperref}  % backref linktocpage pagebackref
\pdfcompresslevel=9
\pdfadjustspacing=1 
\PassOptionsToPackage{pdftex}{graphicx}
	\usepackage{graphicx} 

% ********************************************************************
% Setup the style of the backrefs from the bibliography
% (translate the options to any language you use)
% ********************************************************************
\newcommand{\backrefnotcitedstring}{\relax}%(Not cited.)
\newcommand{\backrefcitedsinglestring}[1]{(Cited on page~#1.)}
\newcommand{\backrefcitedmultistring}[1]{(Cited on pages~#1.)}
\ifthenelse{\boolean{enable-backrefs}}%
{%
		\PassOptionsToPackage{hyperpageref}{backref}
		\usepackage{backref} % to be loaded after hyperref package 
		   \renewcommand{\backreftwosep}{ and~} % separate 2 pages
		   \renewcommand{\backreflastsep}{, and~} % separate last of longer list
		   \renewcommand*{\backref}[1]{}  % disable standard
		   \renewcommand*{\backrefalt}[4]{% detailed backref
		      \ifcase #1 %
		         \backrefnotcitedstring%
		      \or%
		         \backrefcitedsinglestring{#2}%
		      \else%
		         \backrefcitedmultistring{#2}%
		      \fi}%
}{\relax}    

% ********************************************************************
% Hyperreferences
% ********************************************************************
\hypersetup{%
    %draft,	% = no hyperlinking at all (useful in b/w printouts)
    colorlinks=false, linktocpage=true, pdfstartpage=3, pdfstartview=FitV,%
    % uncomment the following line if you want to have black links (e.g., for printing)
    %colorlinks=false, linktocpage=false, pdfborder={0 0 0}, pdfstartpage=3, pdfstartview=FitV,% 
    breaklinks=true, pdfpagemode=UseNone, pageanchor=true, pdfpagemode=UseOutlines,%
    plainpages=false, bookmarksnumbered, bookmarksopen=true, bookmarksopenlevel=1,%
    hypertexnames=true, pdfhighlight=/O,%nesting=true,%frenchlinks,%
    urlcolor=webbrown, linkcolor=RoyalBlue, citecolor=webgreen, %pagecolor=RoyalBlue,%
    %urlcolor=Black, linkcolor=Black, citecolor=Black, %pagecolor=Black,%
    pdftitle={\myTitle},%
    pdfauthor={\textcopyright\ \myName, \myUni, \myFaculty},%
    pdfsubject={},%
    pdfkeywords={},%
    pdfcreator={pdfLaTeX},%
    pdfproducer={LaTeX with hyperref and classicthesis}%
}   

% ********************************************************************
% Setup autoreferences
% ********************************************************************
\usepackage[capitalize]{cleveref}

\Crefname{thm}{Thm.}{Thms.}
\Crefname{lem}{Lemma}{Lemmas.}
\Crefname{pro}{Prop.}{Props.}
\Crefname{cor}{Cor.}{Cors.}

\addtotheorempostheadhook[thm]{\crefalias{thmlisti}{thm}}
\addtotheorempostheadhook[lem]{\crefalias{thmlisti}{lem}}
\addtotheorempostheadhook[pro]{\crefalias{thmlisti}{pro}}
\addtotheorempostheadhook[cor]{\crefalias{thmlisti}{cor}}



% ****************************************************************************************************
% 7. Last calls before the bar closes
% ****************************************************************************************************
% ********************************************************************
% Development Stuff
% ********************************************************************
\listfiles
%\PassOptionsToPackage{l2tabu,orthodox,abort}{nag}
%	\usepackage{nag}
%\PassOptionsToPackage{warning, all}{onlyamsmath}
%	\usepackage{onlyamsmath}

% ********************************************************************
% Last, but not least...
% ********************************************************************
\usepackage{classicthesis} 
% ****************************************************************************************************


% ****************************************************************************************************
% 8. Further adjustments (experimental)
% ****************************************************************************************************
% ********************************************************************
% Changing the text area
% ********************************************************************
%\linespread{1.05} % a bit more for Palatino
%\areaset[current]{312pt}{761pt} % 686 (factor 2.2) + 33 head + 42 head \the\footskip
%\setlength{\marginparwidth}{7em}%
%\setlength{\marginparsep}{2em}%

% ********************************************************************
% Using different fonts
% ********************************************************************
%\usepackage[oldstylenums]{kpfonts} % oldstyle notextcomp
%\usepackage[osf]{libertine}
%\usepackage{hfoldsty} % Computer Modern with osf
%\usepackage[light,condensed,math]{iwona}
%\renewcommand{\sfdefault}{iwona}
%\usepackage{lmodern} % <-- no osf support :-(
%\usepackage[urw-garamond]{mathdesign} <-- no osf support :-(
% ****************************************************************************************************

% ****************************************************************************************************  
% If you like the classicthesis, then I would appreciate a postcard. 
% My address can be found in the file ClassicThesis.pdf. A collection 
% of the postcards I received so far is available online at 
% http://postcards.miede.de
% ****************************************************************************************************

% ****************************************************************************************************
% 1. Configure classicthesis for your needs here, e.g., remove "drafting" below 
% in order to deactivate the time-stamp on the pages
% ****************************************************************************************************
\PassOptionsToPackage{eulerchapternumbers,listings,drafting,%
				 pdfspacing,%floatperchapter,%linedheaders,%
				 subfig,beramono,eulermath,parts, a5paper}{classicthesis}										
% ********************************************************************
% Available options for classicthesis.sty 
% (see ClassicThesis.pdf for more information):
% drafting
% parts nochapters linedheaders
% eulerchapternumbers beramono eulermath pdfspacing minionprospacing
% tocaligned dottedtoc manychapters
% listings floatperchapter subfig
% ********************************************************************

% ********************************************************************
% Triggers for this config
% ******************************************************************** 
\usepackage{ifthen}
\newboolean{enable-backrefs} % enable backrefs in the bibliography
\setboolean{enable-backrefs}{false} % true false
% ****************************************************************************************************


% ****************************************************************************************************
% 2. Personal data and user ad-hoc commands
% ****************************************************************************************************
\newcommand{\myTitle}{Bachelortheses\xspace}
\newcommand{\mySubtitle}{Katze \and Hund\xspace}
\newcommand{\myDegree}{\xspace}
\newcommand{\myName}{Tim Benedikt Herbstrith\xspace}
\newcommand{\myProf}{Put name here\xspace}
\newcommand{\myOtherProf}{Put name here\xspace}
\newcommand{\mySupervisor}{Put name here\xspace}
\newcommand{\myFaculty}{Put data here\xspace}
\newcommand{\myDepartment}{Put data here\xspace}
\newcommand{\myUni}{Put data here\xspace}
\newcommand{\myLocation}{Vienna\xspace}
\newcommand{\myTime}{July 2015\xspace}
\newcommand{\myVersion}{version 0.1.1\xspace}

\newcommand{\ra}{\rightarrow}
\newcommand{\mt}{\mapsto}

% ********************************************************************
% Setup, finetuning, and useful commands
% ********************************************************************
\newcounter{dummy} % necessary for correct hyperlinks (to index, bib, etc.)
\newlength{\abcd} % for ab..z string length calculation
\providecommand{\mLyX}{L\kern-.1667em\lower.25em\hbox{Y}\kern-.125emX\@}
\newcommand{\ie}{i.\,e.}
\newcommand{\Ie}{I.\,e.}
\newcommand{\eg}{e.\,g.}
\newcommand{\Eg}{E.\,g.}

% New- and Renewcommands
\newcommand{\seq}[2][n]{#2_{1},\ldots,#2_{#1}}
\newcommand{\nicefrac}[2]{#1/#2}
\newcommand{\MtN}{\nicefrac{M}{N}}
\newcommand{\MTN}{\faktor{M}{N}}
\newcommand{\T}{T^{(2)}}
\newcommand{\VT}{V\left(\T\right)}
\newcommand{\ET}{E\left(\T\right)}
\newcommand{\ST}[1]{\T_{#1}}
\newcommand{\VST}[1]{V\left(\ST{#1}\right)}
\newcommand{\EST}[1]{E\left(\ST{#1}\right)}
\newcommand{\AutT}{\Aut\left(\T\right)}
\newcommand{\St}[2][]{\mathrm{St}_{#1}(#2)}
% ****************************************************************************************************


% ****************************************************************************************************
% 3. Loading some handy packages
% ****************************************************************************************************
% ******************************************************************** 
% Packages with options that might require adjustments
% ******************************************************************** 
\PassOptionsToPackage{utf8}{inputenc}	% latin9 (ISO-8859-9) = latin1+"Euro sign"
 \usepackage{inputenc}				

%\PassOptionsToPackage{ngerman,american}{babel}   % change this to your language(s)
% Spanish languages need extra options in order to work with this template
%\PassOptionsToPackage{spanish,es-lcroman}{babel}
 \usepackage{babel}
 \usepackage{csquotes}				

% BibLaTex
\usepackage[style=numeric,backend=bibtex]{biblatex}
\addbibresource{./parts/content/References2012}
\addbibresource{./parts/content/References2013}		

\PassOptionsToPackage{fleqn}{amsmath}		% math environments and more by the AMS 
 \usepackage{amsmath,amsfonts,amssymb,amsthm}

% ********************************************************************
% Math, Operators and Theorems
\usepackage{thmtools}	% More Flexibility in Theorem Styles + Provides in Combination with ref-Packages the 
						% Possibility to Refer to Theorem Style in Reference.
\usepackage{faktor}		% Displays factor groups

\numberwithin{equation}{section}

\DeclareMathOperator{\N}{\mathbb{N}}
\DeclareMathOperator{\Z}{\mathbb{Z}}
\DeclareMathOperator{\Q}{\mathbb{Q}}
\DeclareMathOperator{\R}{\mathbb{R}}
\DeclareMathOperator{\C}{\mathbb{C}}
\DeclareMathOperator{\F}{\mathbb{F}}

\DeclareMathOperator{\Aut}{Aut}
\DeclareMathOperator{\id}{id}

\DeclareMathOperator{\kernel}{ker}
\DeclareMathOperator{\im}{im}
\DeclareMathOperator{\End}{\mathrm{End}}
\DeclareMathOperator{\Hom}{\mathrm{Hom}}
\DeclareMathOperator{\Mod}{\mathrm{mod}}
\DeclareMathOperator{\D}{\mathrm{D}}
\DeclareMathOperator{\lcm}{\mathrm{lcm}}
\DeclareMathOperator{\ord}{\mathrm{ord}}



\declaretheorem[
	name=Theorem,
  	refname={theorem,theorems},		%Lower Case Versions of Theorem Type
  	Refname={Theorem,Theorems},
  	numberwithin=section]{thm}
\declaretheorem[
	name=Lemma,
	refname={lemma,lemmas},
	Refname={Lemma,Lemmas},
	sibling=thm]{lem}
\declaretheorem[
	name=Proposition,
	refname={proposition,propositions},
	Refname={Proposition,Propositions},
	sibling=thm]{pro}
\declaretheorem[
	name=Corollary,
	refname={corollary,corollarys},
	Refname={Corollary,Corollarys},
	sibling=thm]{cor}

\declaretheorem[
	name=Definition,
	style=definition,
	numbered=no]{defin}
\declaretheorem[
	name=Example,
	style=definition,
	numbered=no]{exam}
	
\declaretheorem[
	name=Remark,
	style=remark,
	numbered=no]{rem}

% ********************************************************************
%TikZ and TikZ-Styles
\usepackage[dvipsnames]{xcolor}
\usepackage{tikz}
%\pgfrealjobname{HerbstrithBachelorthesis}
%\usetikzlibrary{external}
%\tikzexternalize[prefix=figures/]

\usepackage{pgfcore}
\usetikzlibrary{arrows}
%\usepackage{pgfbaselayers}
%\usepackage{pgfbaseimage} %Externalize Graphics

\tikzstyle{every node}=[circle,draw=black,font=\small,text=black,inner sep=1pt, minimum size=5mm]
\tikzstyle{edge from parent}=[draw=black]
\tikzstyle{st}=[black!20, line join=round, line width=2mm]
\tikzstyle{st1}=[black!40, line join=round, line width=2mm]

\tikzstyle{bgn}=[node distance=6mm, font=normal]
\tikzstyle{hlarrow}=[<->, shorten <=1pt, shorten >=1pt, draw, thick]

\tikzstyle{level 1}=[sibling distance=60mm]
\tikzstyle{level 2}=[sibling distance=30mm]
\tikzstyle{level 3}=[sibling distance=15mm]
\tikzstyle{level 4}=[sibling distance=7.5mm]
\tikzstyle{level 5}=[sibling distance=3.75mm, level distance=5mm]


% ******************************************************************** 
% General useful packages
% ******************************************************************** 
\PassOptionsToPackage{T1}{fontenc} % T2A for cyrillics
	\usepackage{fontenc}     
\usepackage{textcomp} % fix warning with missing font shapes
\usepackage{scrhack} % fix warnings when using KOMA with listings package          
\usepackage{xspace} % to get the spacing after macros right  
\usepackage{mparhack} % get marginpar right
\usepackage{fixltx2e} % fixes some LaTeX stuff 
\PassOptionsToPackage{printonlyused,smaller}{acronym}
	\usepackage{acronym} % nice macros for handling all acronyms in the thesis
%\renewcommand*{\acsfont}[1]{\textssc{#1}} % for MinionPro
\renewcommand{\bflabel}[1]{{#1}\hfill} % fix the list of acronyms
% ****************************************************************************************************


% ****************************************************************************************************
% 4. Setup floats: tables, (sub)figures, and captions
% ****************************************************************************************************
\usepackage{tabularx} % better tables
	\setlength{\extrarowheight}{3pt} % increase table row height
\newcommand{\tableheadline}[1]{\multicolumn{1}{c}{\spacedlowsmallcaps{#1}}}
\newcommand{\myfloatalign}{\centering} % to be used with each float for alignment
\usepackage{caption}
\captionsetup{format=hang,font=small}
\usepackage{subfig} 

\usepackage{enumitem}
\newlist{thmlist}{enumerate}{1}			% thmlist-s may only be used in theorem environments
\setlist[thmlist]{label=(\roman{thmlisti}), ref=\thethm.(\roman{thmlisti}),noitemsep}
\newlist{exlist}{enumerate}{1}
\setlist[exlist]{label=(\arabic{exlisti}), ref=(\arabic{exlisti}),noitemsep,leftmargin=0pt,itemindent=2\parindent}
\newlist{plist}{enumerate}{1}
\setlist[plist]{label=(\roman{plisti}), ref=(\roman{plisti}),noitemsep,leftmargin=0pt,itemindent=2\parindent}
\newlist{clist}{enumerate}{2}
\setlist[clist]{label*=(\alph*), ref=(\alph*),noitemsep,leftmargin=0pt,itemindent=2\parindent} 

\usepackage{multienum}
% ****************************************************************************************************


% ****************************************************************************************************
% 5. Setup code listings
% ****************************************************************************************************
\usepackage{listings} 
%\lstset{emph={trueIndex,root},emphstyle=\color{BlueViolet}}%\underbar} % for special keywords
\lstset{language=[5.2]Mathematica,%C++,
    keywordstyle=\color{RoyalBlue},%\bfseries,
    basicstyle=\small\ttfamily,
    %identifierstyle=\color{NavyBlue},
    commentstyle=\color{Green}\ttfamily,
    stringstyle=\rmfamily,
    numbers=none,%left,%
    numberstyle=\scriptsize,%\tiny
    stepnumber=5,
    numbersep=8pt,
    showstringspaces=false,
    breaklines=true,
    frameround=ftff,
    frame=single,
    belowcaptionskip=.75\baselineskip
    %frame=L
} 
% ****************************************************************************************************    		   


% ****************************************************************************************************
% 6. PDFLaTeX, hyperreferences and citation backreferences
% ****************************************************************************************************
% ********************************************************************
% Using PDFLaTeX
% ********************************************************************
\PassOptionsToPackage{hyperfootnotes=false,pdfpagelabels}{hyperref}
	\usepackage{hyperref}  % backref linktocpage pagebackref
\pdfcompresslevel=9
\pdfadjustspacing=1 
\PassOptionsToPackage{pdftex}{graphicx}
	\usepackage{graphicx} 

% ********************************************************************
% Setup the style of the backrefs from the bibliography
% (translate the options to any language you use)
% ********************************************************************
\newcommand{\backrefnotcitedstring}{\relax}%(Not cited.)
\newcommand{\backrefcitedsinglestring}[1]{(Cited on page~#1.)}
\newcommand{\backrefcitedmultistring}[1]{(Cited on pages~#1.)}
\ifthenelse{\boolean{enable-backrefs}}%
{%
		\PassOptionsToPackage{hyperpageref}{backref}
		\usepackage{backref} % to be loaded after hyperref package 
		   \renewcommand{\backreftwosep}{ and~} % separate 2 pages
		   \renewcommand{\backreflastsep}{, and~} % separate last of longer list
		   \renewcommand*{\backref}[1]{}  % disable standard
		   \renewcommand*{\backrefalt}[4]{% detailed backref
		      \ifcase #1 %
		         \backrefnotcitedstring%
		      \or%
		         \backrefcitedsinglestring{#2}%
		      \else%
		         \backrefcitedmultistring{#2}%
		      \fi}%
}{\relax}    

% ********************************************************************
% Hyperreferences
% ********************************************************************
\hypersetup{%
    %draft,	% = no hyperlinking at all (useful in b/w printouts)
    colorlinks=false, linktocpage=true, pdfstartpage=3, pdfstartview=FitV,%
    % uncomment the following line if you want to have black links (e.g., for printing)
    %colorlinks=false, linktocpage=false, pdfborder={0 0 0}, pdfstartpage=3, pdfstartview=FitV,% 
    breaklinks=true, pdfpagemode=UseNone, pageanchor=true, pdfpagemode=UseOutlines,%
    plainpages=false, bookmarksnumbered, bookmarksopen=true, bookmarksopenlevel=1,%
    hypertexnames=true, pdfhighlight=/O,%nesting=true,%frenchlinks,%
    urlcolor=webbrown, linkcolor=RoyalBlue, citecolor=webgreen, %pagecolor=RoyalBlue,%
    %urlcolor=Black, linkcolor=Black, citecolor=Black, %pagecolor=Black,%
    pdftitle={\myTitle},%
    pdfauthor={\textcopyright\ \myName, \myUni, \myFaculty},%
    pdfsubject={},%
    pdfkeywords={},%
    pdfcreator={pdfLaTeX},%
    pdfproducer={LaTeX with hyperref and classicthesis}%
}   

% ********************************************************************
% Setup autoreferences
% ********************************************************************
\usepackage[capitalize]{cleveref}

\Crefname{thm}{Thm.}{Thms.}
\Crefname{lem}{Lemma}{Lemmas.}
\Crefname{pro}{Prop.}{Props.}
\Crefname{cor}{Cor.}{Cors.}

\addtotheorempostheadhook[thm]{\crefalias{thmlisti}{thm}}
\addtotheorempostheadhook[lem]{\crefalias{thmlisti}{lem}}
\addtotheorempostheadhook[pro]{\crefalias{thmlisti}{pro}}
\addtotheorempostheadhook[cor]{\crefalias{thmlisti}{cor}}



% ****************************************************************************************************
% 7. Last calls before the bar closes
% ****************************************************************************************************
% ********************************************************************
% Development Stuff
% ********************************************************************
\listfiles
%\PassOptionsToPackage{l2tabu,orthodox,abort}{nag}
%	\usepackage{nag}
%\PassOptionsToPackage{warning, all}{onlyamsmath}
%	\usepackage{onlyamsmath}

% ********************************************************************
% Last, but not least...
% ********************************************************************
\usepackage{classicthesis} 
% ****************************************************************************************************


% ****************************************************************************************************
% 8. Further adjustments (experimental)
% ****************************************************************************************************
% ********************************************************************
% Changing the text area
% ********************************************************************
%\linespread{1.05} % a bit more for Palatino
%\areaset[current]{312pt}{761pt} % 686 (factor 2.2) + 33 head + 42 head \the\footskip
%\setlength{\marginparwidth}{7em}%
%\setlength{\marginparsep}{2em}%

% ********************************************************************
% Using different fonts
% ********************************************************************
%\usepackage[oldstylenums]{kpfonts} % oldstyle notextcomp
%\usepackage[osf]{libertine}
%\usepackage{hfoldsty} % Computer Modern with osf
%\usepackage[light,condensed,math]{iwona}
%\renewcommand{\sfdefault}{iwona}
%\usepackage{lmodern} % <-- no osf support :-(
%\usepackage[urw-garamond]{mathdesign} <-- no osf support :-(
% ****************************************************************************************************

% ****************************************************************************************************  
% If you like the classicthesis, then I would appreciate a postcard. 
% My address can be found in the file ClassicThesis.pdf. A collection 
% of the postcards I received so far is available online at 
% http://postcards.miede.de
% ****************************************************************************************************

% ****************************************************************************************************
% 1. Configure classicthesis for your needs here, e.g., remove "drafting" below 
% in order to deactivate the time-stamp on the pages
% ****************************************************************************************************
\PassOptionsToPackage{eulerchapternumbers,listings,drafting,%
				 pdfspacing,%floatperchapter,%linedheaders,%
				 subfig,beramono,eulermath,parts, a5paper}{classicthesis}										
% ********************************************************************
% Available options for classicthesis.sty 
% (see ClassicThesis.pdf for more information):
% drafting
% parts nochapters linedheaders
% eulerchapternumbers beramono eulermath pdfspacing minionprospacing
% tocaligned dottedtoc manychapters
% listings floatperchapter subfig
% ********************************************************************

% ********************************************************************
% Triggers for this config
% ******************************************************************** 
\usepackage{ifthen}
\newboolean{enable-backrefs} % enable backrefs in the bibliography
\setboolean{enable-backrefs}{false} % true false
% ****************************************************************************************************


% ****************************************************************************************************
% 2. Personal data and user ad-hoc commands
% ****************************************************************************************************
\newcommand{\myTitle}{Bachelortheses\xspace}
\newcommand{\mySubtitle}{Katze \and Hund\xspace}
\newcommand{\myDegree}{\xspace}
\newcommand{\myName}{Tim Benedikt Herbstrith\xspace}
\newcommand{\myProf}{Put name here\xspace}
\newcommand{\myOtherProf}{Put name here\xspace}
\newcommand{\mySupervisor}{Put name here\xspace}
\newcommand{\myFaculty}{Put data here\xspace}
\newcommand{\myDepartment}{Put data here\xspace}
\newcommand{\myUni}{Put data here\xspace}
\newcommand{\myLocation}{Vienna\xspace}
\newcommand{\myTime}{July 2015\xspace}
\newcommand{\myVersion}{version 0.1.1\xspace}

\newcommand{\ra}{\rightarrow}
\newcommand{\mt}{\mapsto}

% ********************************************************************
% Setup, finetuning, and useful commands
% ********************************************************************
\newcounter{dummy} % necessary for correct hyperlinks (to index, bib, etc.)
\newlength{\abcd} % for ab..z string length calculation
\providecommand{\mLyX}{L\kern-.1667em\lower.25em\hbox{Y}\kern-.125emX\@}
\newcommand{\ie}{i.\,e.}
\newcommand{\Ie}{I.\,e.}
\newcommand{\eg}{e.\,g.}
\newcommand{\Eg}{E.\,g.}

% New- and Renewcommands
\newcommand{\seq}[2][n]{#2_{1},\ldots,#2_{#1}}
\newcommand{\nicefrac}[2]{#1/#2}
\newcommand{\MtN}{\nicefrac{M}{N}}
\newcommand{\MTN}{\faktor{M}{N}}
\newcommand{\T}{T^{(2)}}
\newcommand{\VT}{V\left(\T\right)}
\newcommand{\ET}{E\left(\T\right)}
\newcommand{\ST}[1]{\T_{#1}}
\newcommand{\VST}[1]{V\left(\ST{#1}\right)}
\newcommand{\EST}[1]{E\left(\ST{#1}\right)}
\newcommand{\AutT}{\Aut\left(\T\right)}
\newcommand{\St}[2][]{\mathrm{St}_{#1}(#2)}
% ****************************************************************************************************


% ****************************************************************************************************
% 3. Loading some handy packages
% ****************************************************************************************************
% ******************************************************************** 
% Packages with options that might require adjustments
% ******************************************************************** 
\PassOptionsToPackage{utf8}{inputenc}	% latin9 (ISO-8859-9) = latin1+"Euro sign"
 \usepackage{inputenc}				

%\PassOptionsToPackage{ngerman,american}{babel}   % change this to your language(s)
% Spanish languages need extra options in order to work with this template
%\PassOptionsToPackage{spanish,es-lcroman}{babel}
 \usepackage{babel}
 \usepackage{csquotes}				

% BibLaTex
\usepackage[style=numeric,backend=bibtex]{biblatex}
\addbibresource{./parts/content/References2012}
\addbibresource{./parts/content/References2013}		

\PassOptionsToPackage{fleqn}{amsmath}		% math environments and more by the AMS 
 \usepackage{amsmath,amsfonts,amssymb,amsthm}

% ********************************************************************
% Math, Operators and Theorems
\usepackage{thmtools}	% More Flexibility in Theorem Styles + Provides in Combination with ref-Packages the 
						% Possibility to Refer to Theorem Style in Reference.
\usepackage{faktor}		% Displays factor groups

\numberwithin{equation}{section}

\DeclareMathOperator{\N}{\mathbb{N}}
\DeclareMathOperator{\Z}{\mathbb{Z}}
\DeclareMathOperator{\Q}{\mathbb{Q}}
\DeclareMathOperator{\R}{\mathbb{R}}
\DeclareMathOperator{\C}{\mathbb{C}}
\DeclareMathOperator{\F}{\mathbb{F}}

\DeclareMathOperator{\Aut}{Aut}
\DeclareMathOperator{\id}{id}

\DeclareMathOperator{\kernel}{ker}
\DeclareMathOperator{\im}{im}
\DeclareMathOperator{\End}{\mathrm{End}}
\DeclareMathOperator{\Hom}{\mathrm{Hom}}
\DeclareMathOperator{\Mod}{\mathrm{mod}}
\DeclareMathOperator{\D}{\mathrm{D}}
\DeclareMathOperator{\lcm}{\mathrm{lcm}}
\DeclareMathOperator{\ord}{\mathrm{ord}}



\declaretheorem[
	name=Theorem,
  	refname={theorem,theorems},		%Lower Case Versions of Theorem Type
  	Refname={Theorem,Theorems},
  	numberwithin=section]{thm}
\declaretheorem[
	name=Lemma,
	refname={lemma,lemmas},
	Refname={Lemma,Lemmas},
	sibling=thm]{lem}
\declaretheorem[
	name=Proposition,
	refname={proposition,propositions},
	Refname={Proposition,Propositions},
	sibling=thm]{pro}
\declaretheorem[
	name=Corollary,
	refname={corollary,corollarys},
	Refname={Corollary,Corollarys},
	sibling=thm]{cor}

\declaretheorem[
	name=Definition,
	style=definition,
	numbered=no]{defin}
\declaretheorem[
	name=Example,
	style=definition,
	numbered=no]{exam}
	
\declaretheorem[
	name=Remark,
	style=remark,
	numbered=no]{rem}

% ********************************************************************
%TikZ and TikZ-Styles
\usepackage[dvipsnames]{xcolor}
\usepackage{tikz}
%\pgfrealjobname{HerbstrithBachelorthesis}
%\usetikzlibrary{external}
%\tikzexternalize[prefix=figures/]

\usepackage{pgfcore}
\usetikzlibrary{arrows}
%\usepackage{pgfbaselayers}
%\usepackage{pgfbaseimage} %Externalize Graphics

\tikzstyle{every node}=[circle,draw=black,font=\small,text=black,inner sep=1pt, minimum size=5mm]
\tikzstyle{edge from parent}=[draw=black]
\tikzstyle{st}=[black!20, line join=round, line width=2mm]
\tikzstyle{st1}=[black!40, line join=round, line width=2mm]

\tikzstyle{bgn}=[node distance=6mm, font=normal]
\tikzstyle{hlarrow}=[<->, shorten <=1pt, shorten >=1pt, draw, thick]

\tikzstyle{level 1}=[sibling distance=60mm]
\tikzstyle{level 2}=[sibling distance=30mm]
\tikzstyle{level 3}=[sibling distance=15mm]
\tikzstyle{level 4}=[sibling distance=7.5mm]
\tikzstyle{level 5}=[sibling distance=3.75mm, level distance=5mm]


% ******************************************************************** 
% General useful packages
% ******************************************************************** 
\PassOptionsToPackage{T1}{fontenc} % T2A for cyrillics
	\usepackage{fontenc}     
\usepackage{textcomp} % fix warning with missing font shapes
\usepackage{scrhack} % fix warnings when using KOMA with listings package          
\usepackage{xspace} % to get the spacing after macros right  
\usepackage{mparhack} % get marginpar right
\usepackage{fixltx2e} % fixes some LaTeX stuff 
\PassOptionsToPackage{printonlyused,smaller}{acronym}
	\usepackage{acronym} % nice macros for handling all acronyms in the thesis
%\renewcommand*{\acsfont}[1]{\textssc{#1}} % for MinionPro
\renewcommand{\bflabel}[1]{{#1}\hfill} % fix the list of acronyms
% ****************************************************************************************************


% ****************************************************************************************************
% 4. Setup floats: tables, (sub)figures, and captions
% ****************************************************************************************************
\usepackage{tabularx} % better tables
	\setlength{\extrarowheight}{3pt} % increase table row height
\newcommand{\tableheadline}[1]{\multicolumn{1}{c}{\spacedlowsmallcaps{#1}}}
\newcommand{\myfloatalign}{\centering} % to be used with each float for alignment
\usepackage{caption}
\captionsetup{format=hang,font=small}
\usepackage{subfig} 

\usepackage{enumitem}
\newlist{thmlist}{enumerate}{1}			% thmlist-s may only be used in theorem environments
\setlist[thmlist]{label=(\roman{thmlisti}), ref=\thethm.(\roman{thmlisti}),noitemsep}
\newlist{exlist}{enumerate}{1}
\setlist[exlist]{label=(\arabic{exlisti}), ref=(\arabic{exlisti}),noitemsep,leftmargin=0pt,itemindent=2\parindent}
\newlist{plist}{enumerate}{1}
\setlist[plist]{label=(\roman{plisti}), ref=(\roman{plisti}),noitemsep,leftmargin=0pt,itemindent=2\parindent}
\newlist{clist}{enumerate}{2}
\setlist[clist]{label*=(\alph*), ref=(\alph*),noitemsep,leftmargin=0pt,itemindent=2\parindent} 

\usepackage{multienum}
% ****************************************************************************************************


% ****************************************************************************************************
% 5. Setup code listings
% ****************************************************************************************************
\usepackage{listings} 
%\lstset{emph={trueIndex,root},emphstyle=\color{BlueViolet}}%\underbar} % for special keywords
\lstset{language=[5.2]Mathematica,%C++,
    keywordstyle=\color{RoyalBlue},%\bfseries,
    basicstyle=\small\ttfamily,
    %identifierstyle=\color{NavyBlue},
    commentstyle=\color{Green}\ttfamily,
    stringstyle=\rmfamily,
    numbers=none,%left,%
    numberstyle=\scriptsize,%\tiny
    stepnumber=5,
    numbersep=8pt,
    showstringspaces=false,
    breaklines=true,
    frameround=ftff,
    frame=single,
    belowcaptionskip=.75\baselineskip
    %frame=L
} 
% ****************************************************************************************************    		   


% ****************************************************************************************************
% 6. PDFLaTeX, hyperreferences and citation backreferences
% ****************************************************************************************************
% ********************************************************************
% Using PDFLaTeX
% ********************************************************************
\PassOptionsToPackage{hyperfootnotes=false,pdfpagelabels}{hyperref}
	\usepackage{hyperref}  % backref linktocpage pagebackref
\pdfcompresslevel=9
\pdfadjustspacing=1 
\PassOptionsToPackage{pdftex}{graphicx}
	\usepackage{graphicx} 

% ********************************************************************
% Setup the style of the backrefs from the bibliography
% (translate the options to any language you use)
% ********************************************************************
\newcommand{\backrefnotcitedstring}{\relax}%(Not cited.)
\newcommand{\backrefcitedsinglestring}[1]{(Cited on page~#1.)}
\newcommand{\backrefcitedmultistring}[1]{(Cited on pages~#1.)}
\ifthenelse{\boolean{enable-backrefs}}%
{%
		\PassOptionsToPackage{hyperpageref}{backref}
		\usepackage{backref} % to be loaded after hyperref package 
		   \renewcommand{\backreftwosep}{ and~} % separate 2 pages
		   \renewcommand{\backreflastsep}{, and~} % separate last of longer list
		   \renewcommand*{\backref}[1]{}  % disable standard
		   \renewcommand*{\backrefalt}[4]{% detailed backref
		      \ifcase #1 %
		         \backrefnotcitedstring%
		      \or%
		         \backrefcitedsinglestring{#2}%
		      \else%
		         \backrefcitedmultistring{#2}%
		      \fi}%
}{\relax}    

% ********************************************************************
% Hyperreferences
% ********************************************************************
\hypersetup{%
    %draft,	% = no hyperlinking at all (useful in b/w printouts)
    colorlinks=false, linktocpage=true, pdfstartpage=3, pdfstartview=FitV,%
    % uncomment the following line if you want to have black links (e.g., for printing)
    %colorlinks=false, linktocpage=false, pdfborder={0 0 0}, pdfstartpage=3, pdfstartview=FitV,% 
    breaklinks=true, pdfpagemode=UseNone, pageanchor=true, pdfpagemode=UseOutlines,%
    plainpages=false, bookmarksnumbered, bookmarksopen=true, bookmarksopenlevel=1,%
    hypertexnames=true, pdfhighlight=/O,%nesting=true,%frenchlinks,%
    urlcolor=webbrown, linkcolor=RoyalBlue, citecolor=webgreen, %pagecolor=RoyalBlue,%
    %urlcolor=Black, linkcolor=Black, citecolor=Black, %pagecolor=Black,%
    pdftitle={\myTitle},%
    pdfauthor={\textcopyright\ \myName, \myUni, \myFaculty},%
    pdfsubject={},%
    pdfkeywords={},%
    pdfcreator={pdfLaTeX},%
    pdfproducer={LaTeX with hyperref and classicthesis}%
}   

% ********************************************************************
% Setup autoreferences
% ********************************************************************
\usepackage[capitalize]{cleveref}

\Crefname{thm}{Thm.}{Thms.}
\Crefname{lem}{Lemma}{Lemmas.}
\Crefname{pro}{Prop.}{Props.}
\Crefname{cor}{Cor.}{Cors.}

\addtotheorempostheadhook[thm]{\crefalias{thmlisti}{thm}}
\addtotheorempostheadhook[lem]{\crefalias{thmlisti}{lem}}
\addtotheorempostheadhook[pro]{\crefalias{thmlisti}{pro}}
\addtotheorempostheadhook[cor]{\crefalias{thmlisti}{cor}}



% ****************************************************************************************************
% 7. Last calls before the bar closes
% ****************************************************************************************************
% ********************************************************************
% Development Stuff
% ********************************************************************
\listfiles
%\PassOptionsToPackage{l2tabu,orthodox,abort}{nag}
%	\usepackage{nag}
%\PassOptionsToPackage{warning, all}{onlyamsmath}
%	\usepackage{onlyamsmath}

% ********************************************************************
% Last, but not least...
% ********************************************************************
\usepackage{classicthesis} 
% ****************************************************************************************************


% ****************************************************************************************************
% 8. Further adjustments (experimental)
% ****************************************************************************************************
% ********************************************************************
% Changing the text area
% ********************************************************************
%\linespread{1.05} % a bit more for Palatino
%\areaset[current]{312pt}{761pt} % 686 (factor 2.2) + 33 head + 42 head \the\footskip
%\setlength{\marginparwidth}{7em}%
%\setlength{\marginparsep}{2em}%

% ********************************************************************
% Using different fonts
% ********************************************************************
%\usepackage[oldstylenums]{kpfonts} % oldstyle notextcomp
%\usepackage[osf]{libertine}
%\usepackage{hfoldsty} % Computer Modern with osf
%\usepackage[light,condensed,math]{iwona}
%\renewcommand{\sfdefault}{iwona}
%\usepackage{lmodern} % <-- no osf support :-(
%\usepackage[urw-garamond]{mathdesign} <-- no osf support :-(
% ****************************************************************************************************

% ****************************************************************************************************  
% If you like the classicthesis, then I would appreciate a postcard. 
% My address can be found in the file ClassicThesis.pdf. A collection 
% of the postcards I received so far is available online at 
% http://postcards.miede.de
% ****************************************************************************************************

% ****************************************************************************************************
% 1. Configure classicthesis for your needs here, e.g., remove "drafting" below 
% in order to deactivate the time-stamp on the pages
% ****************************************************************************************************
\PassOptionsToPackage{eulerchapternumbers,listings,drafting,%
				 pdfspacing,%linedheaders,%
				 subfig,beramono,eulermath,parts, a5paper}{classicthesis}										
% ********************************************************************
% Available options for classicthesis.sty 
% (see ClassicThesis.pdf for more information):
% drafting
% parts nochapters linedheaders
% eulerchapternumbers beramono eulermath pdfspacing minionprospacing
% tocaligned dottedtoc manychapters
% listings floatperchapter subfig
% ********************************************************************

% ********************************************************************
% Triggers for this config
% ******************************************************************** 
\usepackage{ifthen}
\newboolean{enable-backrefs} % enable backrefs in the bibliography
\setboolean{enable-backrefs}{false} % true false
% ****************************************************************************************************


% ****************************************************************************************************
% 2. Personal data and user ad-hoc commands
% ****************************************************************************************************
\newcommand*{\myTitle}{Bachelortheses}
\newcommand*{\mySubtitle}{Katze \and Hund}
\newcommand*{\myDegree}{Bachelor of Science}
\newcommand*{\myName}{Tim Benedikt Herbstrith}
\newcommand*{\myProf}{Put name here\xspace}
\newcommand*{\myOtherProf}{Put name here\xspace}
\newcommand*{\mySupervisor}{Put name here\xspace}
\newcommand*{\myFaculty}{Put data here\xspace}
\newcommand*{\myDepartment}{Put data here\xspace}
\newcommand*{\myUni}{Put data here\xspace}
\newcommand*{\myLocation}{Vienna}
\newcommand*{\myTime}{February 2017}
\newcommand*{\myVersion}{version 0.1.1}

\newcommand*{\ra}{\rightarrow}
\newcommand*{\mt}{\mapsto}

% ********************************************************************
% Setup, finetuning, and useful commands
% ********************************************************************
\newcounter{dummy} % necessary for correct hyperlinks (to index, bib, etc.)
\newlength{\abcd} % for ab..z string length calculation
\providecommand{\mLyX}{L\kern-.1667em\lower.25em\hbox{Y}\kern-.125emX\@}
\newcommand{\ie}{i.\,e.}
\newcommand{\Ie}{I.\,e.}
\newcommand{\eg}{e.\,g.}
\newcommand{\Eg}{E.\,g.}

% New- and Renewcommands
\newcommand{\seq}[2][n]{#2_{1},\ldots,#2_{#1}}
\newcommand{\nicefrac}[2]{#1/#2}
\newcommand{\MtN}{\nicefrac{M}{N}}
\newcommand{\MTN}{\faktor{M}{N}}
\newcommand{\T}{T^{(2)}}
\newcommand{\VT}{V\left(\T\right)}
\newcommand{\ET}{E\left(\T\right)}
\newcommand{\ST}[1]{\T_{#1}}
\newcommand{\VST}[1]{V\left(\ST{#1}\right)}
\newcommand{\EST}[1]{E\left(\ST{#1}\right)}
\newcommand{\AutT}{\Aut\left(\T\right)}
\newcommand{\St}[2][]{\mathrm{St}_{#1}(#2)}
% ****************************************************************************************************



% ****************************************************************************************************
% 3. Loading some handy packages
% ****************************************************************************************************
% ******************************************************************** 
% Packages with options that might require adjustments
% ******************************************************************** 
\PassOptionsToPackage{utf8}{inputenc}	% latin9 (ISO-8859-9) = latin1+"Euro sign"
 \usepackage{inputenc}				

%\PassOptionsToPackage{ngerman,american}{babel}   % change this to your language(s)
% Spanish languages need extra options in order to work with this template
%\PassOptionsToPackage{spanish,es-lcroman}{babel}
 \usepackage{babel}
 \babeltags{de = ngerman}
 
 \usepackage{csquotes}				

% BibLaTex
\usepackage[style=numeric,backend=biber,url=false]{biblatex}
\addbibresource{./parts/content/References2012.bib}
\addbibresource{./parts/content/References2013.bib}		

\PassOptionsToPackage{fleqn}{amsmath}		% math environments and more by the AMS 
 \usepackage{amsmath,amsfonts,amssymb,amsthm}

% ********************************************************************
% Math, Operators and Theorems
\usepackage{thmtools}	% More Flexibility in Theorem Styles + Provides in Combination with ref-Packages the 
						% Possibility to Refer to Theorem Style in Reference.
\usepackage{faktor}		% Displays factor groups

\numberwithin{equation}{section}

\DeclareMathOperator{\N}{\mathbb{N}}
\DeclareMathOperator{\Z}{\mathbb{Z}}
\DeclareMathOperator{\Q}{\mathbb{Q}}
\DeclareMathOperator{\R}{\mathbb{R}}
\DeclareMathOperator{\C}{\mathbb{C}}
\DeclareMathOperator{\F}{\mathbb{F}}

\DeclareMathOperator{\Aut}{Aut}
\DeclareMathOperator{\id}{id}

\DeclareMathOperator{\kernel}{ker}
\DeclareMathOperator{\im}{im}
\DeclareMathOperator{\End}{\mathrm{End}}
\DeclareMathOperator{\Hom}{\mathrm{Hom}}
\DeclareMathOperator{\Mod}{\mathrm{mod}}
\DeclareMathOperator{\D}{\mathrm{D}}
\DeclareMathOperator{\lcm}{\mathrm{lcm}}
\DeclareMathOperator{\ord}{\mathrm{ord}}


\usepackage{mathtools}

\usepackage{letltxmacro}
\LetLtxMacro\orgvdots\vdots
\LetLtxMacro\orgddots\ddots

\makeatletter
\DeclareRobustCommand\vdots{%
    \mathpalette\@vdots{}%
}
\newcommand*{\@vdots}[2]{%
    % #1: math style
    % #2: unused
    \sbox0{$#1\cdotp\cdotp\cdotp\m@th$}%
    \sbox2{$#1.\m@th$}%
    \vbox{%
        \dimen@=\wd0 %
        \advance\dimen@ -3\ht2 %
        \kern.5\dimen@
        % remove side bearings
        \dimen@=\wd2 %
        \advance\dimen@ -\ht2 %
        \dimen2=\wd0 %
        \advance\dimen2 -\dimen@
        \vbox to \dimen2{%
            \offinterlineskip
            \copy2 \vfill\copy2 \vfill\copy2%
        }%
    }%
}

\declaretheoremstyle[
    spaceabove=6pt, spacebelow=6pt,
    headfont=\normalfont\scshape,
    notefont=\mdseries, notebraces={(}{)},
    bodyfont=\normalfont\itshape,
    postheadspace=1em,
    ]{scthm}
\declaretheoremstyle[
    spaceabove=6pt, spacebelow=6pt,
    headfont=\normalfont\scshape,
    notefont=\mdseries, notebraces={(}{)},
    bodyfont=\normalfont,
    postheadspace=1em,
    ]{scdef}
	
\declaretheorem[
	name=theorem,
  	refname={theorem,theorems},		%Lower Case Versions of Theorem Type
  	Refname={Theorem,Theorems},
    style=scthm,
  	numberwithin=section]{thm}
\declaretheorem[
	name=lemma,
	refname={lemma,lemmas},
	Refname={Lemma,Lemmas},
    style=scthm,
	sibling=thm]{lem}
\declaretheorem[
	name=proposition,
	refname={proposition,propositions},
	Refname={Proposition,Propositions},
    style=scthm,
	sibling=thm]{pro}
\declaretheorem[
	name=corollary,
	refname={corollary,corollarys},
	Refname={Corollary,Corollarys},
    style=scthm,
	sibling=thm]{cor}

\declaretheorem[
	name=definition,
	style=scdef,
	sibling=thm]{defin}
\declaretheorem[
	name=example,
	style=scdef,
	sibling=thm]{exam}
	
\declaretheorem[
	name=Remark,
	style=remark,
	numbered=no]{rem}

% ********************************************************************
%TikZ and TikZ-Styles
\usepackage[dvipsnames]{xcolor}
\usepackage{tikz}
%\pgfrealjobname{HerbstrithBachelorthesis}
%\usetikzlibrary{external}
%\tikzexternalize[prefix=figures/]

\usepackage{pgfcore}
\usetikzlibrary{arrows}
%\usepackage{pgfbaselayers}
%\usepackage{pgfbaseimage} %Externalize Graphics

%\tikzstyle{every node}=[circle,draw=black,font=\small,text=black,inner sep=1pt, minimum size=5mm]
%\tikzstyle{edge from parent}=[draw=black]
%\tikzstyle{st}=[black!20, line join=round, line width=2mm]
%\tikzstyle{st1}=[black!40, line join=round, line width=2mm]
%
%\tikzstyle{bgn}=[node distance=6mm, font=normal]
%\tikzstyle{hlarrow}=[<->, shorten <=1pt, shorten >=1pt, draw, thick]
%
%\tikzstyle{level 1}=[sibling distance=60mm]
%\tikzstyle{level 2}=[sibling distance=30mm]
%\tikzstyle{level 3}=[sibling distance=15mm]
%\tikzstyle{level 4}=[sibling distance=7.5mm]
%\tikzstyle{level 5}=[sibling distance=3.75mm, level distance=5mm]
\tikzstyle{comment}=[font=\small]


% ******************************************************************** 
% General useful packages
% ******************************************************************** 
\PassOptionsToPackage{T1}{fontenc} % T2A for cyrillics
	\usepackage{fontenc}     
\usepackage{textcomp} % fix warning with missing font shapes
\usepackage{scrhack} % fix warnings when using KOMA with listings package          
\usepackage{xspace} % to get the spacing after macros right  
\usepackage{mparhack} % get marginpar right
\usepackage{fixltx2e} % fixes some LaTeX stuff 
\PassOptionsToPackage{printonlyused,smaller}{acronym}
	\usepackage{acronym} % nice macros for handling all acronyms in the thesis
%\renewcommand*{\acsfont}[1]{\textssc{#1}} % for MinionPro
%\renewcommand{\bflabel}[1]{{#1}\hfill} % fix the list of acronyms
% ****************************************************************************************************


% ****************************************************************************************************
% 4. Setup floats: tables, (sub)figures, and captions
% ****************************************************************************************************
\usepackage{tabularx} % better tables
	\setlength{\extrarowheight}{3pt} % increase table row height
\newcommand{\tableheadline}[1]{\multicolumn{1}{c}{\spacedlowsmallcaps{#1}}}
\newcommand{\myfloatalign}{\centering} % to be used with each float for alignment
\usepackage{caption}
\captionsetup{format=hang,font=small}
\usepackage{subfig} 

\usepackage{enumitem}
\newlist{thmlist}{enumerate}{1}			% thmlist-s may only be used in theorem environments
\setlist[thmlist]{label=(\roman{thmlisti}), ref=\thethm.(\roman{thmlisti}),noitemsep}
\newlist{exlist}{enumerate}{2}
\setlist[exlist]{label=(\arabic{exlisti}), ref=\thethm.(\arabic{exlisti}),noitemsep,leftmargin=0pt,itemindent=2\parindent}
\newlist{plist}{enumerate}{1}
\setlist[plist]{label=(\roman{plisti}), ref=(\roman{plisti}),noitemsep,leftmargin=0pt,itemindent=2\parindent}
\newlist{clist}{enumerate}{2}
\setlist[clist]{label*=(\alph*), ref=(\alph*),noitemsep,leftmargin=0pt,itemindent=2\parindent} 

% TODO: Indent vor Absätzen

\usepackage{multienum}
% ****************************************************************************************************


% ****************************************************************************************************
% 5. Setup code listings
% ****************************************************************************************************
\usepackage{listings} 
%\lstset{emph={trueIndex,root},emphstyle=\color{BlueViolet}}%\underbar} % for special keywords
\lstset{language=[5.2]Mathematica,%C++,
    keywordstyle=\color{RoyalBlue},%\bfseries,
    basicstyle=\small\ttfamily,
    %identifierstyle=\color{NavyBlue},
    commentstyle=\color{Green}\ttfamily,
    stringstyle=\rmfamily,
    numbers=none,%left,%
    numberstyle=\scriptsize,%\tiny
    stepnumber=5,
    numbersep=8pt,
    showstringspaces=false,
    breaklines=true,
    frameround=ftff,
    frame=single,
    belowcaptionskip=.75\baselineskip
    %frame=L
} 
% ****************************************************************************************************    		   


% ****************************************************************************************************
% 6. PDFLaTeX, hyperreferences and citation backreferences
% ****************************************************************************************************
% ********************************************************************
% Using PDFLaTeX
% ********************************************************************
\PassOptionsToPackage{hyperfootnotes=false,pdfpagelabels}{hyperref}
	\usepackage{hyperref}  % backref linktocpage pagebackref
\pdfcompresslevel=9
\pdfadjustspacing=1 
\PassOptionsToPackage{pdftex}{graphicx}
	\usepackage{graphicx} 

% ********************************************************************
% Setup the style of the backrefs from the bibliography
% (translate the options to any language you use)
% ********************************************************************
\newcommand{\backrefnotcitedstring}{\relax}%(Not cited.)
\newcommand{\backrefcitedsinglestring}[1]{(Cited on page~#1.)}
\newcommand{\backrefcitedmultistring}[1]{(Cited on pages~#1.)}
\ifthenelse{\boolean{enable-backrefs}}%
{%
		\PassOptionsToPackage{hyperpageref}{backref}
		\usepackage{backref} % to be loaded after hyperref package 
		   \renewcommand{\backreftwosep}{ and~} % separate 2 pages
		   \renewcommand{\backreflastsep}{, and~} % separate last of longer list
		   \renewcommand*{\backref}[1]{}  % disable standard
		   \renewcommand*{\backrefalt}[4]{% detailed backref
		      \ifcase #1 %
		         \backrefnotcitedstring%
		      \or%
		         \backrefcitedsinglestring{#2}%
		      \else%
		         \backrefcitedmultistring{#2}%
		      \fi}%
}{\relax}    

% ********************************************************************
% Hyperreferences
% ********************************************************************
\hypersetup{%
    %draft,	% = no hyperlinking at all (useful in b/w printouts)
    colorlinks=true, linktocpage=true, pdfstartpage=3, pdfstartview=FitV,%
    % uncomment the following line if you want to have black links (e.g., for printing)
    %colorlinks=false, linktocpage=false, pdfborder={0 0 0}, pdfstartpage=3, pdfstartview=FitV,% 
    breaklinks=true, pdfpagemode=UseNone, pageanchor=true, pdfpagemode=UseOutlines,%
    plainpages=false, bookmarksnumbered, bookmarksopen=true, bookmarksopenlevel=1,%
    hypertexnames=true, pdfhighlight=/O,%nesting=true,%frenchlinks,%
    urlcolor=webbrown, linkcolor=RoyalBlue, citecolor=webgreen, %pagecolor=RoyalBlue,%
    %urlcolor=Black, linkcolor=Black, citecolor=Black, %pagecolor=Black,%
    pdftitle={\myTitle},%
    pdfauthor={\textcopyright\ \myName, \myUni, \myFaculty},%
    pdfsubject={},%
    pdfkeywords={},%
    pdfcreator={pdfLaTeX},%
    pdfproducer={LaTeX with hyperref and classicthesis}%
}   

% ********************************************************************
% Setup autoreferences
% ********************************************************************
\usepackage[capitalize]{cleveref}

\Crefname{thm}{Thm.}{Thms.}
\Crefname{lem}{Lem.}{Lemmata}
\Crefname{pro}{Prop.}{Props.}
\Crefname{cor}{Cor.}{Cors.}
\Crefname{exam}{Example}{Examples}

\addtotheorempostheadhook[thm]{\crefalias{thmlisti}{thm}}
\addtotheorempostheadhook[lem]{\crefalias{thmlisti}{lem}}
\addtotheorempostheadhook[pro]{\crefalias{thmlisti}{pro}}
\addtotheorempostheadhook[cor]{\crefalias{thmlisti}{cor}}
\addtotheorempostheadhook[exam]{\crefalias{exlisti}{exam}}



% ****************************************************************************************************
% 7. Last calls before the bar closes
% ****************************************************************************************************
% ********************************************************************
% Development Stuff
% ********************************************************************
\listfiles
%\PassOptionsToPackage{l2tabu,orthodox,abort}{nag}
%	\usepackage{nag}
%\PassOptionsToPackage{warning, all}{onlyamsmath}
%	\usepackage{onlyamsmath}

% ********************************************************************
% Last, but not least...
% ********************************************************************
\usepackage{classicthesis} 
% ****************************************************************************************************


% ****************************************************************************************************
% 8. Further adjustments (experimental)
% ****************************************************************************************************
% ********************************************************************
% Changing the text area
% ********************************************************************
%\linespread{1.05} % a bit more for Palatino
%\areaset[current]{312pt}{761pt} % 686 (factor 2.2) + 33 head + 42 head \the\footskip
%\setlength{\marginparwidth}{7em}%
%\setlength{\marginparsep}{2em}%

% ********************************************************************
% Using different fonts
% ********************************************************************
%\usepackage[oldstylenums]{kpfonts} % oldstyle notextcomp
%\usepackage[osf]{libertine}
%\usepackage{hfoldsty} % Computer Modern with osf
%\usepackage[light,condensed,math]{iwona}
%\renewcommand{\sfdefault}{iwona}
%\usepackage{lmodern} % <-- no osf support :-(
%\usepackage[urw-garamond]{mathdesign} <-- no osf support :-(
% ****************************************************************************************************
